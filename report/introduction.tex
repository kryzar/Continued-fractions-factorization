% introduction.tex

La méthode de factorisation par fractions continues est une variante de la
méthode de Fermat-Kraitchik dont les idées (recherche de congruences de carrés
$\pmod{N}$ — $N$ étant le nombre à factoriser — et recherche de relations
linéaires sur $\mathbb{F}_2$) demeurent centrales dans les algorithmes de
factorisation actuels. Cette méthode fut suggérée en 1931 par D. H. Lehmer et
R. E. Powers et permet de manier des quantités plus petites que dans la méthode
de Kraitchik originelle. C'est aussi l'une des premières méthodes de
factorisation dont la complexité soit sous exponentielle. Bien que la méthode
de crible quadratique lui soit aujourd'hui préférée, la méthode des fractions
continues obtint des succès d'importance, notamment la première factorisation
du nombre de Fermat $F_7 = 2^{2^7} + 1$ le 13 septembre 1970. Cet exploit est
dû à M. A. Morrison et J. Brillhart qui implémentèrent la méthode pour la
première fois. Dans ce projet, nous implémentons à notre tour la méthode des
fractions continues, en langage C, et dans une version très proche de celle de
Morrison et Brillhart. La première section de ce texte est donc consacrée à la
théorie mathématique sous-jacente ; la deuxième présente l'implémentation en
détails ; la troisième expose les résultats numériques obtenus.
