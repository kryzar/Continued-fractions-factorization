\subsection{Collecte des paires $(A,Q)$ : \texttt{step\_1.c et lp\_var.c}}

Décrivons à présent la phase de collecte des données. Concernant les vecteurs
historiques, il suffit d'initialiser à la fin de la collecte
\texttt{hist\_vects[i]} pour \texttt{0 <= i < nb\_AQp}. C'est ce que fait la
fonction \texttt{init\_hist\_vects}. La collecte des autres données requiert un
peu plus d'explications. 

\subsubsection{La fonction \texttt{create\_AQ\_pairs}}

Sachant que seules les paires $(A,Q)$ dont on a pu factoriser $Q_n$ nous
intéressent pour la seconde phase, nous avons décidé de ne stocker que
celles-ci. Ce choix a un autre avantage : ayant fixé \texttt{nb\_want\_AQp} le
nombre de paires $(A, Q)$ factorisées désirées, on peut arrêter le
développement en fraction continue dès que ce nombre est atteint. Au fur à
mesure du développement de $\sqrt{kN}$ en fraction continue, il faut donc
tester si le $Q_n$ qui vient d'être calculé est factorisable. Si c'est le cas,
on crée son vecteur exposant et ajoute les données de la paire aux tableaux
\texttt{Ans}, \texttt{Qns} et \texttt{exp\_vects}. Tout ceci est géré dans la
longue fonction \texttt{create\_AQ\_pairs}, qui utilise les sous-routines
\texttt{is\_Qn\_factorisable} et \texttt{init\_exp\_vect}.


\subsubsection{La \textit{early abort strategy}}

La fonction \texttt{is\_Qn\_factorisable} teste si un $Q_n$ est friable
\footnote{Ou presque friable, voir ¶ suivant.} par divisions successives avec
les premiers de la base de factorisation. Il est possible d'améliorer les
performances de la fonction en l'empêchant de poursuivre les divisions 
successives si, après un certain nombre de tentatives, on juge la partie non 
factorisée de $Q_n$ trop grande (\cite{PW}). On fixe pour cela une borne
\texttt{eas\_bound\_div} (proportionnelle à la borne déjà connue $2\sqrt{kN}$),
un palier \texttt{eas\_cut} et l'on arrête les divisions successives sur $Q_n$ 
si $Q_n$ reste plus grand que \texttt{eas\_bound\_div} après \texttt{eas\_cut}
tentatives de divisions.

\subsubsection{La \textit{large prime variation }}

Étant donnée une base de factorisation $B = \{ p_1, \cdots, p_m\}$, la
\textit{large prime variation } (\cite{PW}) consiste à accepter lors de la collecte, non
seulement des $Q_n$ $B$-friables mais aussi des $Q_n$ produits d'un entier
$B$-friable et d'un entier $lp_n$ inférieur à $p_m^2$. On dira que $Q_n$ est
\emph{presque friable} et l'on appellera \emph{grand premier (large prime)} le
premier $lp_n$ en question. \\

Pour que des $Q_n$ presque friables soient exploitables, il faut qu'ils aient
un grand premier $lp$ en commun. En effet, si on trouve deux entiers presque
friables $Q_{n_1} = X_{n_1}lp $ et $Q_{n_2} =  X_{n_2}lp $, on peut former une
nouvelle paire $(A,Q)$ avec laquelle on peut travailler pour chercher une 
congruence de carrés. Remarquons pour cela que :
\[\begin{cases}
	A_{n_1 -1}^2 \equiv (-1)^{n_1} X_{n_1}lp\pmod{N}, \\
	A_{n_2 -1}^2 \equiv (-1)^{n_2} X_{n_2}lp\pmod{N}.
\end{cases}\]

Multiplier entre elles ces deux identités donne :
\[(A_{n_1 -1} A_{n_2 -1})^2 \equiv 
     \underbrace{(-1)^{n_1 + n_2} X_{n_1} X_{n_2}}_
            {\begin{subarray}{c}\text{associé au vecteur exposant}\\
             v_{B} \big( (-1 )^{n_1} X_{n_1} \big)
             + v_{B} \big( (-1 )^{n_2} X_{n_2} \big) \end{subarray}
             }
    \underbrace{lp^2}_{\text{carré qui ne pose pas problème}}
    \pmod{N}.
 \]
  
On forme donc la nouvelle paire $ (A_{n_1-1}A_{n_2 -1} \pmod{N},
Q_{n_1}Q_{n_2}) $ associée au vecteur exposant $v_{B} \big( (-1 )^{n_1} X_{n_1}
\big)+ v_{B} \big( (-1 )^{n_2} X_{n_2} \big) $.  Elle sera traitée lors de la
deuxième phase exactement de la même manière que les paires \og classiques
\fg{}.\\

En pratique, pour repérer les paires qui ont le même grand premier, nous
constituons au fur et à mesure de la collecte une liste chainée dont les nœuds
stockent les données d'une paire dont le $Q_n$ est presque friable (les entiers
$Q_n$, $A_{n-1}$, le vecteur exposant et le grand premier associé à $Q_n$).
Nous maintenons cette liste triée par taille des grands premiers. Lorsque
survient un $Q_n$ presque friable, il est repéré par la fonction
\texttt{is\_Qn\_factorisable} qui fournit également son grand premier $lp$. La
liste chainée est alors parcourue pour savoir si l'on a déjà rencontré ce $lp$.
Deux cas se présentent alors. Si $lp$ est absent de la liste, on crée à la
bonne place un nœud. Si $lp$ est déjà présent dans la liste, au lieu de
rajouter un nœud, on utilise le nœud possédant ce $lp$ pour obtenir une
nouvelle paire $(A,Q)$ selon la méthode énoncée plus haut et ajoute ses
composantes aux tableaux \texttt{Ans}, \texttt{Qns} et \texttt{exp\_vects}. La
fonction \texttt{insert\_or\_elim\_lp} se charge de cela. \\

