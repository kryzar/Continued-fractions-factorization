\subsection{Collecte des paires $(A,Q)$ : \texttt{step\_1.c et lp\_var.c}}

Décrivons à présent la phase de collecte des données. Concernant les vecteurs
historiques, il suffit d'initialiser à la fin de la 
collecte \texttt{hist\_vects[i]} pour \texttt{0 <= i < nb\_AQp}. C'est ce que fait 
la fonction \texttt{init\_hist\_vects}. La collecte des autres données 
requiert un peu plus d'explications. 

\subsubsection{La fonction \texttt{create\_AQ\_pairs}}

Sachant que seules les paires $(A,Q)$ dont on a pu factoriser $Q_n$ nous
intéressent pour la seconde phase, nous avons décidé de ne stocker que celles-ci.
Ce choix a en outre un avantage : étant donné un nombre \texttt{nb\_want\_AQp}
représentant le nombre voulu de telles paires, il est possible d'arrêter le 
développement en fraction continue dès que ce nombre est atteint. Cela évite
d'avoir à stocker toutes les paires $(A,Q)$, pour ensuite sélectionner celles 
qui nous intéressent, en courant le risque d'en avoir trop ou pas assez. \\

Ce choix amène à avoir une grande fonction, en l'occurence 
\texttt{create\_AQ\_pairs}, qui au fur à mesure du développement de $\sqrt{kN}$
en fraction continue, teste si le $Q_n$ qui vient d'être calculé est factorisable.
Si c'est le cas, on crée son vecteur exposant et ajoute les données de la paire
aux tableaux \texttt{Ans}, \texttt{Qns} et \texttt{exp\_vects}. Pour ce faire, la
fonction utilise les sous-routines \texttt{is\_Qn\_factorisable} et 
\texttt{init\_exp\_vect}.

\subsubsection{La \textit{early abort strategy}}

La fonction \texttt{is\_Qn\_factorisable} teste si un $Q_n$ est friable 
\footnote {ou presque friable, voir paragraphe suivant.} par divisions successives
avec les premiers de la base de factorisation.  Un moyen d'améliorer les 
performances de la méthode est de décider de ne pas poursuivre les divisions 
successives si après un nombre \texttt{eas\_cut} de divisions la partie non
factorisée de $Q_n$ est trop grande (supérieure à une borne 
\texttt{eas\_bound\_div} proportionnelle à la borne déjà connue $\sqrt{kN}$). 

\subsubsection{La \textit{large prime variation }}

Etant donnée une base de factorisation $B = \{ p_1, \cdots, p_m\}$, la \textit{large
prime variation } consiste à accepter lors de la collecte, non seulement des 
$Q_n$ $B$-friables mais aussi des $Q_n$ produits d'un entier $B$-friable et d'un
entier $lp_n$ inférieur à $p_m^2$. On dira que $Q_n$ est \emph{presque friable}
et l'on appelera \emph{grand premier (large prime)} le premier $lp_n$ en question. \\

Pour que des $Q_n$ presque friables soient exploitables, il faut qu'ils aient
un grand premier $lp$ en commun. En effet, si on trouve deux entiers presque
friables $Q_{n_1} = X_{n_1}lp $ et $Q_{n_2} =  X_{n_2}lp $, on peut former une
nouvelle paire $(A,Q)$ avec laquelle on peut travailler pour chercher une 
congruence de carrés. \\ 

Remarquons pour cela qu'on a les conguences :
\begin{equation*}
  \left\{
      \begin{aligned}
          A_{n_1 -1}^2 &\equiv (-1)^{n_1} X_{n_1}lp\pmod{N} \\
          A_{n_2 -1}^2 &\equiv (-1)^{n_2} X_{n_2}lp\pmod{N}\\
        \end{aligned}
    \right.
\end{equation*}

En les multipliant, on obtient : 
\[  (A_{n_1 -1} A_{n_2 -1})^2 \equiv 
     \underbrace{(-1)^{n_1 + n_2} X_{n_1} X_{n_2}}_
            {\begin{subarray}{c}\text{associé au vecteur exposant}\\
             v_{B} \big( (-1 )^{n_1} X_{n_1} \big)
             + v_{B} \big( (-1 )^{n_2} X_{n_2} \big) \end{subarray}
             }
    \underbrace{lp^2}_{\text{carré qui ne pose pas problème}}
    \pmod{N}
 \]
  
On forme donc la nouvelle paire $ (A_{n_1-1}A_{n_2 -1} \pmod{N}, Q_{n_1}Q_{n_2}) $
associée au vecteur exposant $v_{B} \big( (-1 )^{n_1} X_{n_1} \big)+
v_{B} \big( (-1 )^{n_2} X_{n_2} \big) $.  Elle sera traitée lors de la 
deuxième phase exactement de la même manière que les paires \og classiques \fg{}.\\

En pratique, pour repérer les paires qui ont le même grand premier, nous
constituons au fur et à mesure de la collecte une liste chainée dont les noeuds
stockent les données d'une paire dont le $Q_n$ est presque friable (les entiers
$Q_n$, $A_{n-1}$, le vecteur exposant et le grand premier associé à $Q_n$). Nous
maintenons cette liste triée par taille des grands premiers. Lorsque que survient
un $Q_n$ presque friable, il est repéré par la fonction
\texttt{is\_Qn\_factorisable} qui fournit également son grand premier $lp$. La 
liste chainée est alors parcourue pour savoir si l'on a déjà rencontré ce $lp$.
Deux cas se présentent alors. Si $lp$ est absent de la liste, on crée à la bonne
place un noeud. Si $lp$ est déjà présent dans la liste, au lieu de rajouter un
noeud, on utilise le noeud possédant ce $lp$ pour obtenir une nouvelle paire
$(A,Q)$ selon la méthode énoncée plus haut et ajoute ses composantes aux tableaux
\texttt{Ans}, \texttt{Qns} et \texttt{exp\_vects}. La fonction
\texttt{insert\_or\_elim\_lp} se charge de cela. \\

