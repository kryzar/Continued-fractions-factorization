\subsection{Architecture du programme}

\subsubsection{Terminologie}

Énonçons pour commencer quelques définitions qui seront utiles pour décrire le
code, en reprenant les notations de la sous-§ précédente.

\begin{definition}
	On nomme \emph{paire $(A, Q)$} tout couple $(A_{n-1}, Q_n), n\geq 1$.
\end{definition}

\begin{definition}
	Un ensemble de paires $(A, Q)$ indexé par $n_1, \dots, n_r$ est dit
	\emph{valide} si le produit $\prod_{i=1}^r (-1)^{n_i} Q_{n_i}$ est un carré
	(dans $\Z$).
\end{definition}

\begin{definition}
    Si $B$ est la base de factorisation utilisée par le programme, on 
    désignera par l'expression \emph{vecteur exposant associé à $Q_n$} 
    le $B$-vecteur exposant $v_B((-1)^n Q_n)$.
\end{definition}

\subsubsection{Structure générale}

Le programme comprend deux étapes principales. La première consiste à générer,
à partir du développement en fractions continues de $\sqrt{kN}$ (voir
\ref{ss-irrquad}), des paires $(A, Q)$ avec $Q_n$ friable pour une base de
factorisation préalablement déterminée et fixée. On associe à chaque $Q_n$
ainsi produit son vecteur exposant \texttt{mpz\_t exp\_vect}. Ce vecteur permet
de retenir les nombres premiers qui interviennent dans la factorisation de
$Q_n$ avec une valuation impaire. Dans le but d'augmenter le nombre de paires
$(A,Q)$ acceptées lors de cette étape, nous avons implémenté la \textit{large
prime variation}. Cette variante permet d'accepter une paire si son $Q_n$ se
factorise grâce aux premiers de la base de factorisation fixée et à un nombre
premier supplémentaire. Les fonctions de cette phase de collecte sont
rassemblées dans le fichier \texttt{step\_1.c}. Elles font appel, pour mettre
en oeuvre la \textit{large prime variation}, aux fonctions du fichier
\texttt{lp\_var.c}. \\
 
Ces données sont traitées lors de la seconde phase dans l'espoir de trouver un
facteur non trivial de $N$. Il s'agit de trouver des ensembles valides de
paires $(A, Q)$ parmi les paires trouvées en première phase. Cela se fait par
pivot de Gau\ss{} sur la matrice dont les lignes sont formées des vecteurs
exposants (voir \ref{matrice-gauss}). Chaque ensemble de paires $(A, Q)$ valide
trouvé fournit une congruence de la forme $A^2 \equiv Q^2$ (mod $N$). Cette
dernière permet éventuellement de trouver un facteur non trivial de $N$. Les
fonctions de cette phase sont regroupées dans le fichier \texttt{step\_2.c}. \\

Avant d'effectuer la première étape, il faut se doter d'une base de factorisation,
ce qui est fait avec les fonctions de \texttt{init\_algo.c}. Ces dernières se
chargent plus généralement de l'initialisation et du choix par défaut des
paramètres, comme la taille de la base de factorisation ou l'entier $k$. \\

Finalement, en mettant bout à bout les deux étapes, la fonction 
\texttt{contfract\_factor} du fichier \texttt{fact.c} recherche un facteur
non trivial de $N$ et \texttt{print\_results} affiche les résultats. 

\subsubsection{Entrées et sorties}

Nous avons regroupé dans une structure \texttt{Params} les paramètres d'entrée
de la fonction de factorisation, à savoir :

\begin{itemize}
	\item \texttt{N} : le nombre à factoriser, supposé produit de deux grands
	nombres premiers.
    \item \texttt{k} : le coefficient multiplicateur.
	\item \texttt{n\_lim} : le nombre maximal de paires $(A,Q)$ que l'on
	s'autorise à calculer. Ce nombre prend en compte toutes les paires
	produites, non uniquement les paires dont le $Q_n$ est friable ou résultant
	de la \textit{large prime variation}.
    \item \texttt{s\_fb} : la taille de la base de factorisation. 
	\item \texttt{nb\_want\_AQp} : le nombre désiré de paires $(A,Q)$ avec
	$Q_n$ friable ou résultant de la \textit{large prime variation}.
	\item des booléens indiquant si la \textit{early abort strategy} ou la
	\textit{large prime variation} doivent être utilisées et des paramètres s'y
	rapportant.
\end{itemize}

Le programme stocke dans une structure \texttt{Results} un facteur non trivial
de \texttt{N} trouvé (si tel est le cas) ainsi que des données permettant 
l'analyse des performances de la méthode.  

\begin{remarque}
	L'efficatité de la méthode dépend du choix des paramètres ci-dessus. Pour
	avoir plus de latitude dans les tests, nous les considérons comme des
	paramètres d'entrée du programme. C'est pourquoi notre programme ne
	s'attèle pas à la factorisation complète d'un entier, qui aurait nécessité
	une sous-routine déterminant des paramètres optimaux en fonction de la
	taille de l'entier dont on cherche un facteur. 
\end{remarque}

\begin{remarque}
	Notre programme n'est pas supposé prendre en entrée un nombre admettant un
	petit facteur premier (inférieur aux premiers de la base de factorisation
	par exemple).  En effet, comme il ne teste pas au préalable si \texttt{N}
	est divisible par de petits facteurs, il mettra autant de temps à trouver
	un petit facteur qu'un grand facteur.
\end{remarque}
