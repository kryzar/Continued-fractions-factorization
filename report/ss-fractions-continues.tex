\subsection{Fractions continues}
\subsubsection{Intuition}

Intuitivement, une fraction continue est une expression — finie ou infinie — de
la forme suivante : \[a_0 + \cfrac{1}{a_1 + \cfrac{1}{a_2 + \cfrac{1}{a_3 +
\dots}}}\] telle que $a_0\in \Z$ et $a_i \in \N^*$ pour tout $i\in \N^*$.  Si
la fraction continue est finie, elle est un rationnel; si la fraction continue
est infinie, on lui associe une valeur en calculant $a_0$, puis $a_0 +
\frac{1}{a_1}$, puis $a_0 + \frac{1}{a_1 +\frac{1}{a_2}}$, et continuant une
infinité de fois. La limite de la suite générée est la \og{} valeur \fg{} de la
fraction continue. Nous ferons sens plus précis de l'intuition dans la
prochaine sous-section. \\

Les fractions continues permettent d'approcher des réels irrationels par une
suite de fractions d'entiers.  Par exemple, la fraction $\frac{103 993}{33
102}$ approche $\pi$ avec une précision meilleure que le milliardième. Pour
générer une telle fraction continue, on part de l'identité $x = x + \ent{x} -
\ent{x}$ et l'on écrit \[x = \ent{x} + \cfrac{1}{\cfrac{1}{x - \ent{x}}}.\] On
pose $x_0 = x$ et $x_1 = \frac{1}{x_0 - \ent{x_0}}$ (bien défini par
irrationalité de $x$) et l'on répète la première étape sur $x_1$ : \[x =
\ent{x_0} + \cfrac{1}{x_1} = \ent{x_0} + \cfrac{1}{\ent{x_1} +
\cfrac{1}{\cfrac{1}{x_1 - \ent{x_1}}}}.\] Comme le réel $x$ est irrationnel, on
peut répéter ce procédé indéfiniment. Nous construisons alors la suite
d'éléments \emph{irrationnels} de terme général \[x_n = \frac{1}{x_{n-1} -
\ent{x_{n-1}}}, \quad \forall n \geq 1.\] On associe alors à
\emph{l'irrationnel} $x$ la fraction continue \emph{infinie}\footnote{Lorsque
nous aurons correctement défini la notion de fraction continue, cette fraction
continue canoniquement associée à $x$ sera notée $\hat{x}$.} \[\hat{x}_0 +
\cfrac{1}{\hat{x}_1 + \cfrac{1}{\hat{x}_2 + \cfrac{1}{\hat{x}_3 + \dots}}},\]
où l'on a posé \[\hat{x}_i = \ent{x_i}\] pour tout $i\in \N$.

\begin{notation}\label{notations}
	Soit $x\in \R$ un élément irrationel. Notons $x_0 = x$, \[x_n :=
	\frac{1}{x_{n-1} - \ent{x_n}}, \quad \forall n \geq 1,\] puis \[\hat{x}_n
	:= \ent{x_n}, \quad \forall n\in \N.\]
\end{notation}

\begin{remarque}
	La méthode de construction d'une fraction continue \emph{finie} pour un
	rationnel est la même : il faut simplement s'arrêter lorsque l'on tombe sur
	un $\hat{x}_n$ vérifiant $\hat{x}_n = \ent{\hat{x}_n}$. Cet algorithme
	termine et s'exécute plus simplement en utilisant l'algorithme d'Euclide
	(\cite{wikiu} ¶ \emph{Les deux fractions continues (finies) d'un rationnel}).
\end{remarque}

\subsubsection{Définition, réduites}

Formellement, on peut définir\footnote{La définition mathématique est
descriptive et non prescriptive.} une fraction continue ainsi :

\begin{definition}[Fraction continue]\label{def-fracont}
	On appelle \emph{fraction continue} toute suite non vide (finie ou infinie)
	$(a_i)_{i\in U} \in \Z^U$, $U\subset \N$, d'entiers qui vérifie \[a_i
	\geq 1, \quad \forall i\in U\setminus\{0\}.\] Cette suite est alors notée
	\[a_0 + \cfrac{1}{a_1 + \cfrac{1}{a_2 + \cfrac{1}{a_3 + \dots}}}.\]
\end{definition}

\begin{notation}
	Soit $x\in \R$ un irrationel. On note $\hat{x}$ la fraction continue
	infinie canoniquement associée à $x$ par la méthode exposée dans le premier
	paragraphe.
\end{notation}

À toute fraction continue est associée une suite (finie ou infinie) de
fractions \og intermédiaires \fg{} appelées
\emph{réduites}\footnote{L'utilisation d'indéterminées formelles permet dans la
définition d'éviter les divisions par zéro.}.

\begin{definition}[Réduites formelles]
	Soit $(X_i)_{i\in \N}$ une suite (infinie) d'indeterminées sur le corps
	$\Q$. On définit \[[X_0] = X_0\] puis par récurrence \[[X_0, \dots,
	X_n] = X_0 + \frac{1}{[X_1, \dots, X_n]}.\]
\end{definition}

\begin{definition}[Réduites d'une fraction continue]
	Soit $f$ une fraction continue.
	\begin{itemize}
		\item Si $f$ est donnée par la suite finie $(a_0, \dots, a_n)$, pour
		tout $k\in [\![0, n]\!]$ on appelle \emph{$k$-ième réduite de $f$}
		l'élément $[a_0, \dots, a_k]$.
		\item Si $f$ est donnée par la suite infinie $(a_i)_{i\in \N}$, pour
		tout $k\in \N$ on appelle \emph{$k$-ième réduite de $f$} l'élément
		$[a_0, \dots, a_k]$.
	\end{itemize}
\end{definition}

\begin{exemple}
	Soit $f$ la fraction continue infinie donnée par la suite $(1)_{i\in \N}$.
	La première réduite est $[1] = 1$, la deuxième est \[[1, 1] = 1 +
	\frac{1}{[1]} = 1 + \frac{1}{1}.\] Plus généralement, la $k$-ième réduite
	de $f$ est de la forme \[[1, 1, \dots, 1] = 1 + \cfrac{1}{1 +
	\cfrac{1}{\dots \\ \cfrac{1}{1 + \cfrac{1}{1}}}}.\]
\end{exemple}

Les réduites de toute fraction continue sont des éléments rationnels, y compris
celles de la forme $\hat{x}$ pour un certain irrationel $x$. De fait, $x$ n'est
égal à aucune des réduites de $\hat{x}$. On a toutefois (voir notations
\ref{notations}) :

\begin{equation} \label{egalite-reduite}
	x = [\hat{x}_1, \dots, \hat{x}_{n-1}, x_n], \quad \forall x\in \N.
\end{equation}
Cette égalité est cruciale dans notre algorithme de factorisation. \\

Si formellement les fractions continues sont des suites (déf.
\ref{def-fracont}), leur représentation graphique permet de les voir
trivialement comme des éléments du corps $\Q$. Si $f$ est une fraction continue
finie de suite $(a_0, \dots, a_n)$, le rationnel \[a_0 + \cfrac{1}{a_1 +
\cfrac{1}{a_2 + \dots}}\] est égal à la dernière réduite $[a_0, \dots, a_n]$.
On dit que $f$ est égale au rationnel $[a_0, \dots, a_n]$. Pour les fractions
continues infinies, ce n'est pas aussi simple.

\begin{definition}
	Soient $l$ un réel et $f$ une fraction continue donnée par la suite infinie
	$(a_i)_{i\in \N}$. On dit que \emph{$f$ converge vers $l$} ou que \emph{$f$
	est le développement en fraction continue de $l$} et l'on note $f = l$ si
	la suite des réduites de $f$ converge vers $l$. Si une fraction continue
	infinie est égale à un certain réel, on dit qu'elle converge.
\end{definition}

\begin{exemple}[Nombre d'or]
	On appelle \emph{nombre d'or} et l'on note $\varphi$ l'unique racine réelle
	positive du polynôme $X^2 - X - 1 \in \Z[X]$. On a $\varphi =\frac{1 +
	\sqrt{5}}{2} \simeq 1, 618$. Comme $\varphi^2 = \varphi + 1$ et que
	$\varphi \neq 0$, on a $\varphi = 1 + \frac{1}{\varphi} = 1 + \cfrac{1}{1 +
	\cfrac{1}{\varphi}}$. En réalité, $\varphi$ admet un développement en
	fraction continue donné par \[\varphi = 1 + \cfrac{1}{1 + \cfrac{1}{1 +
	\cfrac{1}{1 + \dots}}}.\]
\end{exemple}

Des raisonnements d'analyse élémentaire (\cite{wikiu} ¶ \emph{Bijection entre
irrationnels et fractions continues infinies}) permettent de montrer que toute
fraction continue infinie converge, et qu'elle converge vers un irrationnel.

\begin{theoreme}
	L'application canonique \[x \mapsto \cfrac{1}{\hat{x}_0 +
	\cfrac{1}{\hat{x}_1 + \cfrac{1}{\hat{x}_2 + \dots}}}\] établit une
	\emph{bijection} entre l'ensemble des nombres réels irrationnels et
	l'ensemble des fractions continues infinies.
\end{theoreme}

En particulier, un réel une fraction continue converge vers un réel $x$ \ssi
$f=\hat{x}$. Attention, les réels tout entier ne sont pas en bijection avec les
fractions continues (finies ou infinies). En effet, un rationnel est égal à
exactement deux fractions continues car $[a_0, \dots, a_n] = [a_0, \dots, a_n -
1, 1]$ (\cite{wikiu} ¶ \emph{Les deux fractions continues (finies) d'un
rationnel}).

\subsubsection{Irrationels quadratiques}
\label{ss-irrquad}

L'adaptation de l'algorithme de Fermat-Kraitchik avec les fractions continues
que nous verrons plus tard utilise crucialement le développement en fraction
continue de $\sqrt{kN}$, où $N$ est le nombre à factoriser et $k\in \N*$ un
entier arbitraire. Intéréssons nous aux fractions continues des éléments de
cette forme.

\begin{definition}[Irrationel quadratique]
	On appelle \emph{irrationel quadratique} tout nombre réel, algébrique sur
	$\Q$, de degré $2$. Un irrationel quadratique est dit \emph{réduit} si son
	conjugué est dans l'intervalle $]-1, 0[$.
\end{definition}

Les fractions continues d'irrationnels quaratiques sont sujettes à des
phénomènes de périodicité.

\begin{definition}
	Soit $f = (a_i)_{i\in \N}$ une fraction continue. On dit que $f$ est
	\emph{périodique} si la suite l'est à partir d'un certain rang. Il existe
	alors un rang $n_0\in \N$ et une période $p\in \N^*$ tels que \[a_{i} =
	a_{i+p}, \quad \forall i\geq n_0.\] On note alors \[f = [a_0, \dots, a_{n_0
	- 1}, \overline{a_{n_0}, \dots, a_{n_0 + p -1}}].\] On dit que $f$ est
	\emph{purement périodique} si $n_0 = 0$.
\end{definition}

\begin{exemple}
	La fraction continue du nombre d'or est purement périodique de période $1$.
	La fraction continue de l'irrationel $\sqrt{14}$ vaut \[\sqrt{14} = [4,
	\overline{2, 1, 3, 1, 2, 8}].\]
\end{exemple}

Les résultats suivants sont fondamentaux (voir \cite{wikiu} ¶ \emph{Fraction
continue d'un irrationnel quadratique} pour les preuves).

\begin{theoreme}[Lagrange, 1770]
	Un réel irrationel est un irrationel quadratique \ssi son développement en
	fraction continue est périodique.
\end{theoreme}

\begin{theoreme}[Galois, 1829]
	Un irrationnel quadratique est réduit \ssi son développement en fraction
	continue est purement périodique.
\end{theoreme}

\begin{theoreme}[Legendre, 1798]
	Un réel irrationel est la racine carrée d'un entier $>1$ \ssi son
	développement en fraction continue est de la forme \[[a_0, \overline{a_1,
	a_2, \dots, a_2, a_1, 2a_0}].\]
\end{theoreme}

Ces phénomènes de périodicité devront être pris en compte dans les paramètres
d'entrée de l'algorithme de factorisation. En plus de la suite des réduites,
nous aurons besoin d'une autre suite importante.

\begin{lemme}
	Soit $x$ un irrationel quadratique. Alors l'élément $\frac{1}{x - \ent{x}}$
	est lui aussi un irrationel quadratique.
\end{lemme}

Voir \cite{Lauritzen} prop. 2.5.2. Fixons $N$ l'entier à factoriser, $k\in \N*$
tel que $\sqrt{kN}$ est un irrationel quadratique et $x :=
\sqrt{kN}$. D'après l'identité \ref{egalite-reduite} et en
reprenant les notations \ref{notations}, nous pouvons donner un développement
partiel (jusqu'à un rang donné $n\in \N$) de $x$ en fraction continue \[x =
[\hat{x}_1, \dots, \hat{x}_{n-1}, x_n].\] D'après le lemme précédent, $x_n$ est
lui aussi un irrationel quadratique, i.e.  il est solution d'une équation
quadratique. On peut (\cite{Lauritzen} dém. de 2.5.8) de fait l'écrire de
manière unique 

\begin{equation}
	x_n = \frac{P_n + x}{Q_n}, \quad P_n, Q_n \in \Z.
\end{equation}

La $n$-ième réduite de $x$ étant un nombre rationnel, on peut l'écrire sous la
forme $\frac{A_n}{B_n}$, où $A_n, B_n$ sont entiers et la fraction est
irréductible bien définie. Pour tout $n\geqslant 1$, on a (\cite{Lauritzen} §
2.7) \[A_{n-1}^2 - kN B_{n-1}^2 = (-1)^n Q_n\] et donc
\begin{equation}
	A_{n-1}^2 \equiv (-1)^n Q_n.
\end{equation}
On a également (\cite{Lauritzen} dém. de 2.5.8)
\begin{equation}\label{inegalite}
	\begin{cases}
		P_n < \sqrt{kN} \\
		Q_n < 2\sqrt{kN}.
	\end{cases}
\end{equation}

Ces notations et identitées seront cruciales dans la suite.
