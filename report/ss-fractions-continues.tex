\subsection{Fraction continues}
\subsubsection{Intuition}

Intuitivement, une fraction continue est une expression — finie ou infinie — de
la forme suivante\footnote{Notez que nous ne nous autorisons que des $1$ aux
numérateurs.} : \[a_0 + \cfrac{1}{a_1 + \cfrac{1}{a_2 + \cfrac{1}{a_3 +
\dots}}}\] telle que $a_0\in \Z$ et $a_i \in \N^*$ pour tout $i\in \N^*$.
Toujours intuitivement, nous voulons affûbler cette fraciton continue d'une
valeur. Si la fraction continue est finie, cette une bonne vieille fraciton,
c'est à dire un élément du corps $\Q$~; si la fraction continue est infinie, on
calcule d'abord $a_0$, puis $a_0 + \frac{1}{a_1}$, puis $a_0 + \frac{1}{a_1
+\frac{1}{a_2}}$, et on continue une infinité de fois. La limite de la suite
générée est la \og{} valeur \fg{} de la fraction continue. Nous ferons sens
plus précis de l'intuition dans le prochain paragraphe. \\

Les fractions continues émanent de la volonté d'approcher des réels irrationels
par des fractions d'entiers. Par exemple, la fraction $\frac{103 993}{33 102}$
approche $\pi$ avec une précision meilleure que le milliardième.  Comment
générer une telle fraction continue pour un réel irrationnel $x$ ?  On part de
l'identité $x = x + \ent{x} - \ent{x}$ et l'on écrit \[x = \ent{x} +
\cfrac{1}{\cfrac{1}{x - \ent{x}}}.\] On pose désormais $a_1 = \frac{1}{x -
\ent{x}}$ (qui est bien défini par irrationalité de $x$) et l'on répète la
première étape sur $a_1$ : \[x = \ent{x} + \cfrac{1}{a_1} = \ent{x} +
\cfrac{1}{\ent{a_1} + \cfrac{1}{\cfrac{1}{a_1 - \ent{a_1}}}}.\] On recommence :
on pose $a_2 = \frac{1}{a_1 - \ent{a_1}}$ pour obtenir \[x = \ent{x} +
\cfrac{1}{a_1 + \cfrac{1}{a_2 + \cfrac{1}{\cfrac{1}{a_2 - \ent{a_2}}}}}.\]
Comme le réel $x$ est irrationnel, ce procédé ne s'arrête jamais. Nous
construisons alors la suite de terme général \[a_n = \frac{1}{a_{n-1} -
\ent{a_{n-1}}}, \quad \forall n \geq 1\] et associons canoniquement à
\emph{l'irrationnel} $x$ la fraction continue \emph{infinie} \[a_0 +
\cfrac{1}{a_1 + \cfrac{1}{a_2 + \cfrac{1}{a_3 + \dots}}}.\]

\begin{remarque}
	La méthode de construction d'une fraction continue \emph{finie} pour un
	rationnel est la même : il faut simplement s'arrêter lorsque l'on tombe sur
	un $a_n$ vérifiant $a_n = \ent{a_n}$. Cet algorithme termine (\NTS{ref}) et
	s'exécute plus simplement en utilisant… l'algorithme d'Euclide. Par
	ailleurs, réaffirmons que la fraction continue d'un irrationel (encore une
	fois, dans un sens qui sera précisé au prochain paragraphe) est forcément
	infinie.
\end{remarque}

\subsubsection{Formalisation}

\subsubsection{Irrationels quadratiques}
