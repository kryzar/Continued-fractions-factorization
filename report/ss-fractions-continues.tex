\subsection{Fraction continues}
\subsubsection{Intuition}

Intuitivement, une fraction continue est une expression — finie ou infinie — de
la forme suivante\footnote{Notez que nous ne nous autorisons que des $1$ aux
numérateurs.} : \[a_0 + \cfrac{1}{a_1 + \cfrac{1}{a_2 + \cfrac{1}{a_3 +
\dots}}}\] telle que $a_0\in \Z$ et $a_i \in \N^*$ pour tout $i\in \N^*$.
Toujours intuitivement, nous voulons affûbler cette fraciton continue d'une
valeur. Si la fraction continue est finie, cette une bonne vieille fraciton,
c'est à dire un élément du corps $\Q$~; si la fraction continue est infinie, on
calcule d'abord $a_0$, puis $a_0 + \frac{1}{a_1}$, puis $a_0 + \frac{1}{a_1
+\frac{1}{a_2}}$, et on continue une infinité de fois. La limite de la suite
générée est la \og{} valeur \fg{} de la fraction continue. Nous ferons sens
plus précis de l'intuition dans la prochaine sous-section. \\

Les fractions continues émanent de la volonté d'approcher des réels irrationels
par des fractions d'entiers. Par exemple, la fraction $\frac{103 993}{33 102}$
approche $\pi$ avec une précision meilleure que le milliardième.  Comment
générer une telle fraction continue pour un réel irrationnel $x$ ?  On part de
l'identité $x = x + \ent{x} - \ent{x}$ et l'on écrit \[x = \ent{x} +
\cfrac{1}{\cfrac{1}{x - \ent{x}}}.\] On pose $x_0 = x$ et $x_1 = \frac{1}{x_0 -
\ent{x_0}}$ (qui est bien défini par irrationalité de $x$) et l'on répète la
première étape sur $x_1$ : \[x = \ent{x_0} + \cfrac{1}{x_1} = \ent{x_0} +
\cfrac{1}{\ent{x_1} + \cfrac{1}{\cfrac{1}{x_1 - \ent{x_1}}}}.\] Comme le réel
$x$ est irrationnel, on peut répéter ce procédé indéfiniment. Nous construisons
alors la suite d'éléments \emph{irrationnels} de terme général \[x_n =
\frac{1}{x_{n-1} - \ent{x_{n-1}}}, \quad \forall n \geq 1.\] On associe alors à
\emph{l'irrationnel} $x$ la fraction continue \emph{infinie}\footnote{Lorsque
nous aurons correctement défini la notion de fraction continue, cette fraction
continue canoniquement associée à $x$ sera notée $\hat{x}$.} \[\hat{x}_0 +
\cfrac{1}{\hat{x}_1 + \cfrac{1}{\hat{x}_2 + \cfrac{1}{\hat{x}_3 + \dots}}},\]
où l'on a posé \[\hat{x}_i = \ent{x_i}\] pour tout $i\in \N$. Fixons ces
notations :

\begin{notation}\label{notations}
	Soit $x\in \R$ un élément irrationel. Notons $x_0 = x$ puis \[x_n :=
	\frac{1}{x_{n-1} - \ent{x_n}}, \quad \forall n \geq 1.\] Par ailleurs,
	notons \[\hat{x}_n := \ent{x_n}, \quad \forall n\in \N.\]
\end{notation}

\begin{remarque}
	La méthode de construction d'une fraction continue \emph{finie} pour un
	rationnel est la même : il faut simplement s'arrêter lorsque l'on tombe sur
	un $\hat{x}_n$ vérifiant $\hat{x}_n = \ent{\hat{x}_n}$. Cet algorithme
	termine (\NTS{ref}) et s'exécute plus simplement en utilisant… l'algorithme
	d'Euclide.
\end{remarque}

\subsubsection{Formalisation}

Formellement, on peut définir\footnote{La définition mathématique est
descriptive et non prescriptive.} une fraction continue ainsi :

\begin{definition}[Fraction continue]\label{def-fracont}
	On appelle \emph{fraction continue} toute suite non vide (finie ou infinie)
	$(a_i)_{i\in U} \in \N^\N$, $U\subset \N$, d'entiers qui vérifie \[a_i
	\geq 1, \quad \forall i\in U\setminus\{0\}.\] Cette suite est alors notée
	\[a_0 + \cfrac{1}{a_1 + \cfrac{1}{a_2 + \cfrac{1}{a_3 + \dots}}}.\]
\end{definition}

\begin{notation}
	Soit $x\in \R$ un élément irrationel. On note $\hat{x}$ la fraction
	continue infinie canoniquement associée à $x$ par la méthode exposée dans
	le premier paragraphe. Autrement dit, $\hat{x}$ est la fraction continue
	donnée par la suite infinie (voir \ref{notations}) $(\hat{x}_i)_{i\in
	\N})$.
\end{notation}

Il est naturel d'associer à une fraction continue (finie ou infinie) une suite
(finie ou infinie) de fractions \og intermédiaires \fg{} appelées
\emph{réduites}. Pour n'avoir aucun problème de division par zéro, nous nous
plaçons temporairement dans un corps de fractions rationnelles en $\N$
indeterminées.

\begin{definition}[Réduites formelles]
	Soit $(X_i)_{i\in \N}$ une suite (infinie) d'indeterminées sur le corps
	$\Q$. On définit \[[X_0] = X_0]\] puis par récurrence \[[X_1, \dots,
	X_n] = X_0 + \frac{1}{[X_1, \dots, X_n]}.\] Ces éléments sont dans
	$\Q((X_i)_{i\in \N})$.
\end{definition}

\begin{definition}[Réduites d'une fraction continue]
	Distinguons les cas finis et infinis. Soit $f$ une fraction
	continue.
	\begin{itemize}
		\item Si $f$ est donnée par la suite finie $(a_0, \dots, a_n)$, pour
		tout $k\in [\![0, n]\!]$ on appelle \emph{$k$-ième réduite de $f$}
		l'élément $[a_0, \dots, a_k]$.
		\item Si $f$ est donnée par la suite infinie $(a_i)_{i\in \N}$, pour
		tout $k\in \N$ on appelle \emph{$k$-ième réduite de $f$} l'élément
		$[a_0, \dots, a_k]$.
	\end{itemize}
\end{definition}

\begin{exemple}
	Soit $f$ la fraction continue infinie donnée par la suite $(1)_{i\in \N}$.
	La première réduite est $[1] = 1$, la deuxième est \[[1, 1] = 1 +
	\frac{1}{[1]} = 1 + \frac{1}{1},\] la troisième est \[[1, 1, 1] = 1 +
	\frac{1}{[1, 1]} = 1 + \frac{1}{1 + \cfrac{1}{1}}.\] Plus généralement, la
	$k$-ième réduite de $f$ est de la forme \[[1, 1, \dots, 1] = 1 +
	\cfrac{1}{1 + \cfrac{1}{\dots \\ \cfrac{1}{1 +
	\cfrac{1}{1}}}}.\]
\end{exemple}

Remarquons que les réduites de toute fraction continue sont des éléments
rationnels, ce même si la fraction continue est égale à $\hat{x}$ pour un
certain irrationel $x$. De fait, $x$ n'est égal à aucune des réduites de
$\hat{x}$. Mais en reprenant les notations \ref{notations}, on a toutefois
\begin{equation}
	x = [\hat{x}_1, \dots, \hat{x}_{n-1}, x_n], \quad \forall x\in \N.
\end{equation}
Cette égalité sera cruciale dans notre algorithme de factorisation. \\

Même si les fractions continues finie restent des suites (déf.
\ref{def-fracont}), leur représentation graphique permet de les voir
trivialement comme des éléments du corps $\Q$. En effet, en représentant \[a_0
+ \cfrac{1}{a_1 + \cfrac{1}{a_2 + \dots}}\] la fraction continue finie $f$
associée à la suite finie $(a_0, \dots, a_n)$, on peut la voir comme l'élément
rationnel \[a_0 + \cfrac{1}{a_1 + \cfrac{1}{a_2 + \dots}}.\] Cet élément n'est
autre que sa dernière réduite $[a_0, \dots, a_n]$ et on dit que $f$ est égale à
l'élément rationnel $[a_0, \dots, a_n]$. Pour les fractions continues infinies,
ce n'est pas aussi simple.

\begin{definition}
	Soient $l$ un réel et $f$ une fraction continue donnée par la suite infinie
	$(a_i)_{i\in \N}$. On dit que \emph{$f$ est égale à $l$} ou que \emph{$f$
	converge vers $l$} et l'on note $f = l$ si la suite des réduits de $f$
	converge vers $l$. Si une fraction continue infinie est égale à un certain
	réel, on dit qu'elle converge.
\end{definition}

\begin{exemple}[Nombre d'or]
	On appelle \emph{nombre d'or} et l'on note $\varphi$ l'unique racine réelle
	positive du polynôme $X^2 - X - 1 \in \Z[X]$. On a $\varphi =\frac{1 +
	\sqrt{5}}{2} \simeq 1, 618$. Comme $\varphi^2 = \varphi + 1$ et que
	$\varphi \neq 0$, on a $\varphi = 1 + \frac{1}{\varphi} = 1 + \cfrac{1}{1 +
	\cfrac{1}{\varphi}}$. En réalité, $\varphi$ est égal à une fraction
	continue : \[\varphi = 1 + \cfrac{1}{1 + \cfrac{1}{1 + \cfrac{1}{1 +
	\dots}}}.\]
\end{exemple}

Dans quelle mesure une fraction continue converge-t-elle ? Des raisonnements
d'analyse élémentaire (\NTS{réf}) permettent de montrer que toute fraction
continue infinie converge, et qu'elle converge vers un irrationnel !

\begin{theoreme}
	La fonction canonique \[x \mapsto \cfrac{1}{\hat{x}_0 +
	\cfrac{1}{\hat{x}_1 + \cfrac{1}{\hat{x}_2 + \dots}}}\] est une
	\emph{bijection} entre l'ensemble des nombres réels irrationnels et des
	fractions continues infinies.
\end{theoreme}

En particulier, un réel $x$ et une fraction continue $f$ sont égaux \ssi $f$
est la fraction continue donnée par la suite $(\hat{x}_{i\in \N})$. Attention,
les réels tout entier ne sont pas en bijection avec les fractions continues
(finies ou infinies). En effet, un rationnel est égal (au sens donné dans les
paragraphes précédents) à exactement deux fractions continues : si un rationnel
est égal à $[a_0, \dots, a_n]$, il est aussi égal à $[a_0, \dots, a_n - 1, 1]$
et n'est égal à aucune autre fraction continue (\NTS{ref}).

\subsubsection{Irrationels quadratiques}
