\subsection{Méthodes de factorisation de Fermat-Kraitchik et utilisation des
fractions continues}

Dans toute section, $N$ désigne un entier naturel composé impair.

\subsubsection{Méthodes de Fermat et Kraitchik}

La méthode de factorisation de Fermat par du constat suivant.

\begin{lemme}
	Factoriser $N$ est équivalent à l'exprimer comme différence de deux carrés
	d'entiers.
\end{lemme}

\begin{proof}
	En effet, si $N = u^2 - v^2, u, v\in \Z$ alors $N = (u-v)(u + v)$.
	Réciproquement si l'on a une factorisation $N = ab$, alors \[N =
	\left(\frac{a+b}{2}\right)^2 - \left(\frac{a-b}{2}\right)^2.\]
\end{proof}

La méthode de Fermat cherche donc à exploiter cette propriété en exprimant $N$
comme la différence de deux carrés et en déduire une factorisation. Celle-ci se
montre particulièrement efficace lorsque $N$ est le produit de deux entiers
proches l'un de l'autre. Notons $N=ab$ une factorisation de $N$,
$r=\frac{a+b}{2}$ et $s=\frac{a-b}{2}$. On a \[N = r^2 - s^2\] et que l'entier
$r$ est donc plus grand que $\sqrt{N}$ tout en lui étant proche. Il existe donc
un entier positif $u$ \emph{pas trop grand} tel que \[\ent{\sqrt{N}} + u = r\]
et donc tel que $(\ent{\sqrt{N}} + u)^2 - n$ soit un carré. Trouver un tel
entier $u$ donne alors la factorisation de $N$. Comme les facteurs de $N$ sont
proches l'un de l'autre, on le trouve par essais successis. \\

La méthode de Fermat n'est cependant pas du tout efficace lorsque les facteurs
de $N$ ne sont pas proches. D'après \NTS{MB}, la méthode est alors encore plus
coûteuse que la méthode des divisions successives. \\

Dans les années 1920, Maurice Kraitchik a raffiné la méthode de Fermat pour
améliorer son efficacité. Ses idées sont au cœur des algorithmes de
factorisations les plus performants en 2020. Son idée essentielle est que pour
factoriser $N$, il n'est pas \emph{nécessaire} de l'exprimer comme différence
de deux carrés ~; trouver une différence de deux carrés qui soit un multiple de
$N$ \emph{suffit}.

\begin{lemme}
	Conaître deux entiers $u, v \in \Z$ tels que $u^2 \equiv v^2 \equiv
	\pmod{N}$ et $u\not\equiv \pm v\pmod{N}$ fournit une factorisation de $N$.
\end{lemme}

\begin{proof}
	Posons $g =\pgcd(u-v, N)$ et $g' = \pgcd(u+v, N)$. Comme $u\not\equiv \pm
	v\pmod{N}$, on a $g<N$ et $g'<N$. Enfin ni $g$ et $g'$ ne sont réduits à
	$1$ : si l'un deux l'est, l'autre vaut $N$, contradiction. Donc $g$ et $g'$
	sont tous deux des facteurs non triviaux de $N$.
\end{proof}

\begin{remarque}
	Dans l'algorithme, nous nous contenterons de chercher des $u, v$ tels que
	$n$ divise la différence de leurs carrés, sans vérifier s'ils vérifient $u
	\not\equiv \pm v\pmod{N}$. Comme le polynôme $X^2 - v^2 \in \Z/N\Z$ a
	exactement quatre racines, il y a \og{} une chance sur deux \fg pour que
	$u$ et $v$ nous fournissentun facteur non trivial de $N$. \\
	\NTS{Margot tu confirmes qu'on vérifie pas si $u\equiv \pm v$ ?}
\end{remarque}

Comment trouver de tels couples $(u, v)$ ? Kraitchik a originellement utilisé
le polynôme $K := X^2 - N \in \Z[X]$. Trouver une famille d'entiers $x_1,
\cdots, x_k$ telle que $K(x_1)\cdots K(x_r)$ soit un carré fournit un tel
couple $(u, v)$. Il faut poser $u = x_1\cdots x_r$ et $v = K(x_1)\cdots K(x_r)$
: \[v^2 \equiv (x_1^2 - N)\cdots (x_k^2 - N) \equiv x_1^2 \cdots x_k^2 \equiv
u^2\pmod{N}.\] Cette méthode souffre toutefois d'un problème d'efficacité,
puisqu'il est nécessaire de calculer un grand nombre de $K(x_i)$. La croissance
de la fonction associée au polynôme $K$ étant quadratique, le coût des calculs
devient prohibitif. Nous allons maintenant voir que les fractions continues
sont une solution efficace à ce problème. \\

\subsubsection{Utilisation des fractions continues}

\subsubsection{Recherche de congruences de carrés}
