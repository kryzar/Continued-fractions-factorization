\subsection{Factorisation par fractions continues}

Dans tout le reste de cette section, $N$ désigne un entier naturel composé impair.

\subsubsection{Méthodes de Fermat et Kraitchik}

La méthode de factorisation de Fermat part du constat suivant.

\begin{lemme}
	Factoriser $N$ est équivalent à l'exprimer comme différence de deux carrés
	d'entiers.
\end{lemme}

\begin{proof}
	Si $N = u^2 - v^2, u, v\in \Z$ alors $N = (u-v)(u + v)$. Réciproquement si
	l'on a une factorisation $N = ab$, alors $N = \left(\frac{a+b}{2}\right)^2
	- \left(\frac{a-b}{2}\right)^2$.
\end{proof}

La méthode de Fermat exploite cette propriété et se montre particulièrement
efficace lorsque $N$ est le produit de deux entiers proches l'un de l'autre.
Notons $N=ab$ une factorisation de $N$, $r=\frac{a+b}{2}$, $s=\frac{a-b}{2}$ de
sorte que \[N = r^2 - s^2.\] Par hypothèse, l'entier $r$ est donc plus grand
que $\sqrt{N}$ tout en lui étant proche. Il existe donc un entier positif $u$
\emph{pas trop grand} tel que \[\ent{\sqrt{N}} + u = r\] et donc tel que
$(\ent{\sqrt{N}} + u)^2 - N$ soit un carré. Trouver un tel entier $u$ donne
alors la factorisation de $N$. Comme les facteurs de $N$ sont proches l'un de
l'autre, on le trouve par essais successifs. \\

La méthode de Fermat est cependant inefficace lorsque les facteurs de $N$ ne
sont pas proches. D'après \cite{Tale} ¶ \emph{Fermat and Kraitchik}, la méthode
est alors encore plus coûteuse que la méthode des divisions successives. Dans
les années 1920, Maurice Kraitchik a amélioré l'efficacité de la méthode de
Fermat. Son idée essentielle est que pour factoriser $N$, il \emph{suffit} de
trouver une différence de deux carrés qui soit un multiple de $N$.

\begin{lemme}
	Connaître deux entiers $u, v \in \Z$ tels que $u^2 \equiv v^2 
	\pmod{N}$ et $u\not\equiv \pm v\pmod{N}$ fournit une factorisation de $N$.
	Plus spécifiquement, les entiers $\pgcd(u-v, N)$ et $\pgcd(u+v, N)$ sont
	des facteurs non triviaux de $N$.
\end{lemme}

\begin{proof}
	Posons $g =\pgcd(u-v, N)$ et $g' = \pgcd(u+v, N)$. Comme $u\not\equiv \pm
	v\pmod{N}$, on a $g<N$ et $g'<N$. Enfin ni $g$ et $g'$ ne sont réduits à
	$1$ : si l'un deux l'est, l'autre vaut $N$, contradiction. Donc $g$ et $g'$
	sont tous deux des facteurs non triviaux de $N$.
\end{proof}

\begin{remarque}
	Dans l'algorithme, nous nous contenterons de chercher des $u, v$ tels que
	$N$ divise la différence de leurs carrés, sans vérifier s'ils vérifient $u
	\not\equiv \pm v\pmod{N}$. Comme le polynôme $X^2 - v^2 \in \Z/N\Z$ a
	exactement quatre racines, il y a \og{} une chance sur deux \fg  pour que
	$u$ et $v$ nous fournissentun facteur non trivial de $N$. \\
\end{remarque}

Pour factoriser $N$, il s'agit donc de trouver de tels couples $(u, v)$.
Kraitchik cherche pour cela des couples $(u_i, v_i)_{1\leq i \leq r}$ vérifiant
\[u_i^2 \equiv v_i \pmod{N}\] et tels que l'entier $\prod_{i=1}^r v_i$ soit un
carré (dans $\Z$). Posant $u = \prod_{i=1}^r u_i$ et $v = \sqrt{\prod_{i=1}^r
v_i}$, il vient \[v^2\equiv u^2 \pmod{N}.\] En posant $K := X^2 - N \in \Z[X]$,
alors $u_i^2 \equiv K(u_i) \pmod{N}$ pour tout $u_i\in \Z$.  Il a donc cherché
des éléments $v_i$ de la forme $K(u_i)$ dont le produit est un carré. Cette
méthode souffre toutefois d'un problème d'efficacité, puisqu'il est nécessaire
de calculer un grand nombre de $K(x_i)$. La croissance de la fonction associée
au polynôme $K$ étant quadratique, le coût des calculs devient prohibitif.

\NTS{régler la question de l'efficacité et du coût}

\subsubsection{Recherche de congruences de carrés}

Nous exposons ici une méthode (\NTS{attribution}) de recherche de couples $(u,
v)$ tels que $u^2\equiv v^2\pmod{N}$ et $u\not\equiv \pm v \pmod{N}$. Elle est
basée sur la connaissance de congruences de la forme \[u_i^2 \equiv Q_i
\pmod{N},\] où $| Q_i |$ est suffisamment petit pour être factorisé. On fixe
pour cela $B$ une base de factorisation, c'est à dire un ensemble non vide fini
de nombres premiers, et l'on dit qu'un entier $Q\in \N\setminus \{0, 1\}$ est
\emph{$B$-friable} si tous ces facteurs premiers sont dans $B$. La connaissance
de suffisamment d'entiers $Q_i$ qui sont $B$-friables permet de factoriser $N$.

\begin{proposition}
	Soit $F$ une famille d'entiers dont les valeurs absolues sont $B$-friables.
	Si \[\# F \geqslant \# B + 2\] alors on peut extraire une sous-famille de
	$F$ dont le produit des éléments est un carré.
\end{proposition}

\begin{proof}
	Posons $F = \{Q_1, \dots, Q_k\}$ (de sorte que $k = \# F)$ et $B = (p_1,
	\dots, p_m)$ (de sorte que $\# B = m$). Par hypothèse de friabilité, les
	éléments $Q_j, 1\leqslant j \leqslant k$, s'écrivent alors \[Q_j
	= (-1)^{v_0}\prod_{i=1}^m p_i^{v_{p_i}(Q_j)}.\] Fixons $j, j'\in [\![1,
	k]\!]$. Puisque les éléments de $B$ sont fixés et en nombre fini l'élément
	$Q_j$ peut être vu comme le vecteur \[v_B(Q_j) := (v_{p_m}(Q_j), \dots,
	v_{p_1}(Q_j), v_0 )\] L'entier $Q_j$ est un carré \ssi les composantes de
	son vecteur $v_B(Q_j)$ sont paires, i.e. la réduction du vecteur $v_B(Q_j)$
	modulo $2$ est nulle. Par propriété des valuations, le vecteur associé au
	produit $Q_j\cdot Q_{j'}$ est le vecteur somme $v_B(Q_j) + v_B(Q_{j'})$.
	Autrement dit, le produit d'une sous-famille $\{Q_{j_1}, \dots, Q_{j_s}\}$
	de $F$ est un carré \ssi les vecteurs $v_B(Q_{j_1}), \dots, v_B(Q_{j_s})$
	somment à $0$ modulo $2$. Soit $V$ le $\mathbb{F}_2$-espace vectoriel
	$\mathbb{F}_2^{m+1}$, qui est de dimension $m+1$. Comme $k\geqslant m+2$,
	la famille $\{v_B(Q_1), \dots, v_B(Q_k)\}$ est liée dans $V$ et il existe
	de fait des éléments $l_1,\dots, l_k\in \mathbb{F}_2$ tels que
	\[\sum_{j=1}^k l_j v_B(Q_j) = 0.\] L'élément $\prod_{j=1}^k Q_j^{l_j}$ est
	alors un carré.
\end{proof}

\begin{definition}
	Soient $B = \{p_1,\dots, p_m \}$ une base de factorisation et $Q\in \Z$ un
	entier dont la valeur absolue est $B$-friable. On appelle \emph{$B$-vecteur
	exposant de $Q$} et l'on note $v_B(Q)$ le vecteur\footnote{L'élément $v_0$
	est placé à droite et non au début car il correspondra au bit de poids
	faible du $B$-vecteur exposant le code.} des valuations \[v_B(Q) :=
	(v_{p_m}(Q), \dots, v_{p_1}(Q), v_0 ) \in \mathbb{F}_2^{m+1}.\] 
\end{definition}

Étant données des congruences de la forme $u_i^2 \equiv Q_i \pmod{N}$, la
preuve de la proposition fournit un procédé d'algèbre linéaire pour extraire
une sous-famille des $Q_i$ dont le produit des éléments est un carré. On trouve
tout d'abord des $Q_{i_1}, \dots, Q_{i_k}$ dont la valeur absolue est
$B$-friable\footnote{Nous le ferons en factorisant les $Q_i$ à disposition par
divisions successives.}. Soit $M$ la matrice \[M := \begin{pmatrix}
v_B(Q_{i_1}) \\ \vdots \\ v_B(Q_{i_k})\end{pmatrix}\in \mathcal{M}_{k,
\#B+1}(\mathbb{F}_2).\] Soient $l_{1}, \dots, l_{k}$ les éléments de
$\mathbb{F}_2$ donnés dans la preuve de la proposition tels que \[\prod_{j=1}^k
Q_{i_j}^{l_{i_j}}\] soit un carré. Le vecteur $(l_1, \dots, l_{k})$ est un
élément du noyau de la matrice transposée de $M$. Nous verrons plus tard une
adaptation de la méthode du pivot de Gau\ss{} permettant d'exhiber un tel
élément.

\subsubsection{Utilisation des fractions continues}

L'introduction des fractions continues est motivée par le constat suivant. Si
\[u_i^2 = Q_i + kNb^2, \quad u_i, Q_i, b\in \Z, k\in \N^*\] de telle sorte que
$| Q_i |$ soit petit, alors \[\left(\frac{u_i}{b}\right)^2 - kN =
\frac{Q_i}{b^2}\] est petit en valeur absolue et $\frac{u_i}{b}$ est une bonne
approximation de $\sqrt{kN}$ \NTS{réf}. Fixons $k$ un entier naturel non nul,
posons $x = \sqrt{kN}$ et reprenons les notations \ref{notations} et celles
développées à la fin de la sous-section \ref{ss-irrquad}. En vertu de
l'identité \[A_{n-1}^2 \equiv (-1)^n Q_n \pmod{N},\] nous appelons
\emph{méthode de factorisation des fractions continues} la méthode de Kraitchik
dans laquelle les entiers $u_i$ sont donnés par les $A_{i-1}$ et les $v_i$ par
les $(-1)^i Q_i$. On sait que les $Q_n$ sont majorés (\ref{inegalite}) par $2
\sqrt{kN}$ ~; l'inverse, les $x^2 - N\in \Z[X], x\in \N$ ont une croissance
linéaire de pente $2\sqrt{N}$ lorsque $x$ s'éloigne de $\sqrt{N}$. À base de
factorisation $B$ fixée, les $Q_n$ auront donc plus de chance d'être
$B$-friables que les $K(x)$ de Kraitchik. Or, l'étape la plus coûteuse de
l'algorithme est celle de la recherche des termes $B$-friables par divisions
successives. Enfin, notons qu'il est facile de générer le développement en
fraction continue de $x$ et les paires $(A, Q)$ grace à un algorithme itératif
dû à Gau\ss{} et exposé dans \NTS{réf}. \\

\begin{definition}
	Pour tout $n\in \N^*$, on appelle \emph{$n$-ième paire $(A, Q)$} le couple
	$(A_{n-1}, Q_n)$.
\end{definition}

\begin{definition}
	Un ensemble de paires $(A, Q)$ indexé par $n_1, \dots, n_k$ est dit
	\emph{valide} si le produit $\prod_{i=1}^k (-1)^{n_i} Q_{n_i}$ est un carré
	(dans $\Z$ et non uniquement dans $\Z/N\Z$).
\end{definition}

\begin{remarque}
	Par abus de langage, nous confonderons les $B$-vecteurs exposants
	$v_B(Q_n)$ et $v_B((-1)^n Q_n)$.
\end{remarque}

\NTS{Je trouve que ça sert à rien de donner à moitié l'algo à cet endroit alors
que le lecteur a très bien compris la méthode de Kraitchik et qu'on a donné nos
arguments pour les fractions continues. M'est avis que le mieux est de faire ça
en plus détaillé en tout début de la prochaine section. Si tu n'es pas d'accord
on pourra remettre ce qu'il y avait avant.} \\

\NTS{A rajouter je pense dans cette partie : critère pour sélectionner la base
de factorisation, le problème de la périodicité d'où introduction de k, l'idée
à la base de la large prime variation + dans une autre sous-section des trucs sur 
la complexité}
