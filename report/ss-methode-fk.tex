\subsection{Factorisation par fractions continues}

Dans toute ce paragraphe, $N$ désigne un entier naturel composé impair.

\subsubsection{Méthodes de Fermat et Kraitchik}

La méthode de factorisation de Fermat part du constat suivant.

\begin{lemme}
	Factoriser $N$ est équivalent à l'exprimer comme différence de deux carrés
	d'entiers.
\end{lemme}

\begin{proof}
	Si $N = u^2 - v^2, u, v\in \Z$ alors $N = (u-v)(u + v)$. Réciproquement si
	l'on a une factorisation $N = ab$, alors $N = \left(\frac{a+b}{2}\right)^2
	- \left(\frac{a-b}{2}\right)^2$.
\end{proof}

La méthode de Fermat exploite cette propriété et se montre particulièrement
efficace lorsque $N$ est le produit de deux entiers proches l'un de l'autre.
Notons $N=ab$ une factorisation de $N$, $r=\frac{a+b}{2}$, $s=\frac{a-b}{2}$ de
sorte que \[N = r^2 - s^2.\] Par hypothèse, l'entier $r$ est donc plus grand
que $\sqrt{N}$ tout en lui étant proche. Il existe donc un entier positif $u$
\emph{pas trop grand} tel que \[\ent{\sqrt{N}} + u = r\] et donc tel que
$(\ent{\sqrt{N}} + u)^2 - N$ soit un carré. Trouver un tel entier $u$ donne
alors la factorisation de $N$. Comme les facteurs de $N$ sont proches l'un de
l'autre, on le trouve par essais successifs. \\

La méthode de Fermat est cependant inefficace lorsque les facteurs de $N$ ne
sont pas proches. D'après \cite{Tale} ¶ \emph{Fermat and Kraitchik}, la méthode
est alors encore plus coûteuse que la méthode des divisions successives. Dans
les années 1920, Maurice Kraitchik a amélioré l'efficacité de la méthode de
Fermat. Son idée essentielle est que pour factoriser $N$, il \emph{suffit} de
trouver une différence de deux carrés qui soit un multiple de $N$.

\begin{lemme}
	Connaître deux entiers $u, v \in \Z$ tels que $u^2 \equiv v^2 
	\pmod{N}$ et $u\not\equiv \pm v\pmod{N}$ fournit une factorisation de $N$.
	Plus spécifiquement, les entiers $\pgcd(u-v, N)$ et $\pgcd(u+v, N)$ sont
	des facteurs non triviaux de $N$.
\end{lemme}

\begin{proof}
	Posons $g =\pgcd(u-v, N)$ et $g' = \pgcd(u+v, N)$. Comme $u\not\equiv \pm
	v\pmod{N}$, on a $g<N$ et $g'<N$. Enfin ni $g$ et $g'$ ne sont réduits à
	$1$ : si l'un deux l'est, l'autre vaut $N$, contradiction. Donc $g$ et $g'$
	sont tous deux des facteurs non triviaux de $N$.
\end{proof}

\begin{remarque}
	Dans l'algorithme, nous nous contenterons de chercher des $u, v$ tels que
	$N$ divise la différence de leurs carrés, sans vérifier s'ils vérifient $u
	\not\equiv \pm v\pmod{N}$. Comme le polynôme $X^2 - v^2 \in \Z/N\Z$ a
	exactement quatre racines, il y a \og{} une chance sur deux \fg  pour que
	$u$ et $v$ nous fournissentun facteur non trivial de $N$. \\
\end{remarque}

Pour factoriser $N$, il s'agit donc de trouver de tels couples $(u, v)$.
Kraitchik cherche pour cela des couples $(u_i, v_i)_{1\leq i \leq r}$ vérifiant
\[u_i^2 \equiv v_i \pmod{N}\] et tels que l'entier $\prod_{i=1}^r v_i$ soit un
carré (dans $\Z$). Posant $u = \prod_{i=1}^r u_i$ et $v = \sqrt{\prod_{i=1}^r
v_i}$, il vient \[v^2\equiv u^2 \pmod{N}.\] En posant $K := X^2 - N \in \Z[X]$,
alors $u_i^2 \equiv K(u_i) \pmod{N}$ pour tout $u_i\in \Z$.  Il a donc cherché
des éléments $v_i$ de la forme $K(u_i)$ dont le produit est un carré. Cette
méthode souffre toutefois d'un problème d'efficacité, puisqu'il est nécessaire
de calculer un grand nombre de $K(x_i)$. La croissance de la fonction associée
au polynôme $K$ étant quadratique, le coût des calculs devient prohibitif.

\NTS{régler la question de l'efficacité et du coût}

\subsubsection{Recherche de congruences de carrés}

Une méthode qui répond à cette question sans passer par un algorithme de recherche
exhaustive a été proposée par Kraitchik lui-même. \\ 

\NTS{Morrison et Brillart, dans leur article exposant la méthode de factorisation
avec les fractions continues, ont répondu à cette question. (garder la bonne
version, Pomerancedit M et B)}

Celle-ci nécessite de travailler avec des congruences de la forme 
\[u_i^2 \equiv Q_i \pmod{N}\] avec $\lvert Q_i \rvert$ suffisamment petit pour 
en connaître une factorisation. (La notation $Q_i$ fera échos à celle utilisée 
pour décrire la méthode de factorisation avec les fractions continues). \\

On commence par se fixer $B$ une base de factorisation, c'est à dire un 
ensemble non vide fini de nombres premiers.

\begin{definition}
	Un entier $Q\in \N\setminus \{0, 1\}$ est dit \emph{$B$-friable} si tous
	les facteurs premiers de $Q$ sont dans $B$.
\end{definition}

L'idée principale de Kraitchik \NTS{ou M et B} est que connaître suffisamment
d'entiers $Q_i \in \Z$ avec $ \lvert Q_i \rvert$ $B$-friables 
\emph{entièrement factorisés} permet de trouver la famille recherchée.

\begin{proposition}
	Soit $F$ une famille d'entiers tels que leur valeur absolue est $B$-friables.
    Si \[\# F \geqslant \# B + 2\] alors on peut extraire une sous-famille de
    $F$ dont le produit des éléments est un carré.
\end{proposition}

\begin{proof}
	Posons $F = \{Q_1, \dots, Q_k\}$ (de sorte que $k = \# F)$ et $B = (p_1,
	\dots, p_m)$ (de sorte que $\# B = m$). Les éléments $Q_j, 1\leqslant j
    \leqslant k$ s'écrivent alors, comme $\lvert Q_j \rvert$ est $B$-friable,
    \[Q_j = (-1)^{v_0}\prod_{i=1}^m p_i^{v_{p_i}(Q_j)}.\]
	Fixons $j, j'\in [\![1, k]\!]$. Puisque les éléments de $B$ sont fixés et
	en nombre fini l'élément $Q_j$ peut être vu comme le vecteur
    \[v(Q_j) := (v_{p_m}(Q_j), \dots, v_{p_1}(Q_j), v_0 )\] L'entier $Q_j$ est
    un carré \ssi les composantes de son vecteur $v(Q_j)$ sont paires, i.e. la
	réduction du vecteur $v(Q_j)$ modulo $2$ est nulle. Par propriété des
	valuations, le vecteur associé au produit $Q_j\cdot Q_{j'}$ est le vecteur
	somme $v(Q_j) + v(Q_{j'})$. Autrement dit, le produit d'une sous-famille
	$\{Q_{j_1}, \dots, Q_{j_s}\}$ de $F$ est un carré \ssi les vecteurs
	$v(Q_{j_1}), \dots, v(Q_{j_s})$ somment à $0$ modulo $2$. Soit $V$ le
    $\mathbb{F}_2$-espace vectoriel $\mathbb{F}_2^{m+1}$, qui est de dimension
    $m+1$. Comme $k\geqslant m+2$, la famille $\{v(Q_1), \dots, v(Q_k)\}$ est 
    liée dans $V$ et il existe de fait des éléments $l_1,\dots, l_k\in 
    \mathbb{F}_2$ tels que \[\sum_{j=1}^k l_j v(Q_j) = 0.\] L'élément 
    $\prod_{j=1}^k Q_j^{l_j}$ est alors un carré.
\end{proof}

\begin{definition}
    Soient $B = \{p_1,\dots, p_m \}$ une base de factorisation et $Q$ un 
    entier avec $\lvert Q \rvert$  $B$-friable. Si $Q = (-1)^{v_0} 
    \prod_{i=1}^m p_i^{v_{p_i}(Q)}$, on appellera vecteur exposant associé
    à l'entier $Q$ le vecteur \[v(Q) := (v_{p_m}(Q), \dots, v_{p_1}(Q), v_0 )
    \in \mathbb{F}_2^{m+1} \] 
\end{definition}

\begin{remarque}
    Cette définition sera réutilisée lors de la description de l'algorithme
    de factorisation avec la méthode des fractions continues. L'ordre des
    composantes du vecteur présentée ici vient du fait que le $v_0$ correspondra
    au bit de poids faible et le $v_{p_m}(Q)$ au bit de poids fort du vecteur. \\ 
\end{remarque}

Notons $B = \{p_1, \dots, p_m\}$ une base de factorisation. Etant données
des congruences $u_i^2 \equiv Q_i \pmod{N}$, la preuve de la proposition 
fournit un procédé d'algèbre linéaire pour extraire une familles $Q_1,
\cdots, Q_r$ telle que $\prod_{i=1}^r Q_i$ soit un carré. Tout d'abord,
il faut trouver des $Q_i$ tels que $ \lvert Q_i \rvert$ est $B$-friable
(ce que nous ferons par divisions successives) et retenir leur vecteur 
exposant associé. \NTS{est-ce que la description de la matrice ça va ?}. 
Disons qu'on a pu en trouver $k$, notés $Q_1, \cdots, Q_k$.
Soit ensuite $M$ la matrice $k\times(m+1)$ dont les lignes correspondent
aux vecteurs exposants de $Q_{1}, \cdots, Q_{k}$. Soient $l_{1}, \dots,
l_{k}$ les éléments de $\mathbb{F}_2$ donnés par la proposition et tels que
$\prod_{j=1}^k Q_{j}^{l_j} $ soit un carré. Le vecteur $(l_1, \dots,
l_{k})$ est un élément du noyau de la matrice transposée de $M$. Un tel
élément est facilement produit avec des algorithmes usuels d'algèbres
linéaires. Nous verrons plus tard une version adaptée du pivot
de Gau\ss{} qui nous permet d'en produire.

\subsubsection{Utilisation des fractions continues}

Pour appliquer la méthode ci-dessus, il faut arriver à générer des $Q_i$ 
avec $\lvert  Q_i \rvert$ $B$-friable pour une base de factorisation donnée,
supposée \og{} pas très grande \fg. Il faut donc chercher à avoir des
$\lvert Q_i \rvert$ \og{} petits \fg. L'introduction des fractions continues
est motivée par le constat suivant : si $u_i^2 = Q_i + kNb^2$ avec $\lvert 
Q_i \rvert$ petit, alors $\left(\frac{u_i}{b}\right)^2 - kN = \frac{Q_i}{b^2}$
est petit en valeur absolue et $\frac{u_i}{b}$ est une bonne approximation de 
$\sqrt{kN}$.

\NTS{mettre référence}

Redonnons quelques notations de la sous-section \NTS{réf} sur les fractions
continues. Nous notons $x := \sqrt{N}$, puis conformément à \NTS{réf} $x_0 :=
x$ et $x_n = \frac{1}{x_{n-1}- \ent{x_{n-1}}}$ pour tout $n\in \N^*$. Le
développement en fraction continue de l'irrationel $x$ est donc \[x = \hat{x}_0
+ \cfrac{1}{\hat{x}_1 + \cfrac{1}{\hat{x}_2 + \cfrac{1}{\hat{x}_3 + \dots}}}.\]
Sa $n$-ième réduite est pour tout $n$ un rationnel et s'exprime de fait comme
une fraction réduite $\frac{A_n}{B_n}$ où $A_n, B_n\in \Z$.  En posant
$\hat{x}_n := \ent{x_n}$ pour tout $n\in \N$, on a l'égalité \NTS{réf} \[x =
[\hat{x}_1, \dots, \hat{x}_{n-1}, x_n].\] Cet élément $x_n$ est un irrationel
quadratique et s'écrit de fait \NTS{réf} \[x_n = \frac{P_n + x}{Q_n},\quad P_n,
Q_n\in \Z,\]. 

L'algorithme de factorisation repose sur l'égalité suivante : \[ A_{n-1}^2 = 
(-1)^n Q_n + kN B_{n-1}^2 \]

On a donc : \[ A_{n-1}^2 \equiv (-1)^n Q_n \pmod{N} \] \\


Donnons à présent quelques définitions dont nous nous servirons pour décrire 
l'algorithme des fractions continues. 

\begin{definition}
	Pour tout $n\in \N^*$, on appelle \emph{$n$-ième paire $(A, Q)$} le couple
	$(A_{n-1}, Q_n)$.
\end{definition}

\begin{definition}
	Un ensemble de paires $(A, Q)$ indexé par $n_1, \dots, n_k$ est dit
	\emph{valide} si le produit $\prod_{i=1}^k (-1)^{n_i} Q_{n_i}$ est un carré
	(dans $\Z$ et non uniquement dans $\Z/N\Z$).
\end{definition}

\begin{remarque}
    Par abus de langage, nous parlerons à présent du vecteur exposant 
    associé à $Q_n$ pour désigner le vecteur exposant associé à 
    $(-1)^n Q_n = (-1)^{v_0} \prod_{i=1}^m p_i^{v_{p_i}(Q)}$,
    c'est-à-dire $(v_{p_m}(Q_n),\dots, v_{p_1}(Q_n), v_0 ) \in \mathbb{F}_2^{m+1}$
\end{remarque}

D'après la section précédente, étant donnée une base de factorisation $B$,
la méthode consiste à : 

\begin{itemize}
    \item Calculer des couples $(A,Q)$ par développement en fractions continues 
        de $\sqrt(kN)$ où k est un petit coefficient multiplicateur (voir plus loin)
    \item Sélectionner les $Q_n$ $B$-friables et leur associer un vecteur exposant
    \item Trouver un ensemble valide de paires $(A, Q)$ par pivot de Gauss 
          sur la matrice dont les lignes sont formées de ces vecteurs exposants
\end{itemize}

Nous aurons donc déterminé une famille $n_1,\dots, n_k$ d'indices tels que le 
produit $Q:=\prod_{i=1}^k (-1)^{n_i} Q_{n_i}$ soit un carré (dans $\Z$ et non 
uniquement dans $\Z/N\Z$). \\
Posons $A := \prod_{i=1}^k A_{n_i}$. Si $A\not\equiv \pm  \sqrt{Q}\pmod{N}$, 
nous aurons factorisé $N$ en vertu du lemme
\NTS{réf}.\\

Le principal avantage de l'utilisation des fractions continues plutôt que le
polynôme de Kraitchik réside dans leur croissance. L'inégalité \NTS{réf} assure
que les éléments $Q_n, n\in \N$ seront plus petits (en valeur absolu) que les
$K(x'), x'\in \Z$ (dont la croissance lorsque $x'$ s'éloigne de $\sqrt{N}$ est
approximativement linéaire de pente $2\sqrt{N}$). Les paires $(A, Q)$ seront
donc plus faciles à manier et les $Q_n$ auront plus de chance d'être $B$-friables.
Enfin, il est facile de générer le développement en fraction continue de $x$ et 
les paires $(A, Q)$ grace à un algorithme itératif dû à Gau\ss et exposé dans 
\NTS{réf}. \\

\NTS{Pour moi c'est plus ça l'avantage : } \\

Le principal avantage qu'offre l'utilisation des fractions continues par 
rapport au polynome de Kraitchik réside dans la croissance des termes $Q_n,
n\in \N$. On sait par \NTS{réf} que les $Q_n$ seront inférieurs à 
$2 \sqrt{kN}$. Les $K(x)$ donnés par le polynôme de Kratichik ont eux une croissance
lorsque $x$ s'éloigne de $\sqrt{N}$ approximativement linéaire de pente
$2 \sqrt{N}$. Les $Q_n$ auront donc plus de chance d'être  $B$-friables que les
$K(x)$ de Kraitchik. Or, l'étape la plus coûteuse de l'algorithme est celle de la
recherche des termes $B$-friables par divisions successives. Enfin, notons qu'il
est facile de générer le développement en fraction continue de $x$ et les paires
$(A, Q)$ grace à un algorithme itératif dû à Gau\ss et exposé dans \NTS{réf}. \\

\NTS{A rajouter je pense dans cette partie : critère pour sélectionner la base
de factorisation, le problème de la périodicité d'où introduction de k, l'idée
à la base de la large prime variation + dans une autre sous-section des trucs sur 
la complexité}
