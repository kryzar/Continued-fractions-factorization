\subsection{Méthodes de factorisation de Fermat-Kraitchik}

Dans toute section, $N$ désigne un entier naturel composé impair.

\subsubsection{Méthode de Fermat}

La méthode de factorisation de Fermat par du constat suivant.

\begin{lemme}
	Factoriser $N$ est équivalent à l'exprimer comme différence de deux carrés
	d'entiers.
\end{lemme}

\begin{proof}
	En effet, si $N = u^2 - v^2, u, v\in \Z$ alors $N = (u-v)(u + v)$.
	Réciproquement si l'on a une factorisation $N = ab$, alors \[N =
	\left(\frac{a+b}{2}\right)^2 - \left(\frac{a-b}{2}\right)^2.\]
\end{proof}

La méthode de Fermat cherche donc à exploiter cette propriété en exprimant $N$
comme la différence de deux carrés et en déduire une factorisation. Celle-ci se
montre particulièrement efficace lorsque $N$ est le produit de deux entiers
proches l'un de l'autre. Notons $N=ab$ une factorisation de $N$,
$r=\frac{a+b}{2}$ et $s=\frac{a-b}{2}$. On a \[N = r^2 - s^2\] et que l'entier
$r$ est donc plus grand que $\sqrt{N}$ tout en lui étant proche. Il existe donc
un entier positif $u$ \emph{pas trop grand} tel que \[\ent{\sqrt{N}} + u = r\]
et donc tel que $(\ent{\sqrt{N}} + u)^2 - n$ soit un carré. Trouver un tel
entier $u$ donne alors la factorisation de $N$. Comme les facteurs de $N$ sont
proches l'un de l'autre, on le trouve par essais successis. \\

La méthode de Fermat n'est cependant pas du tout efficace lorsque les facteurs
de $N$ ne sont pas proches. D'après \NTS{MB}, la méthode est alors encore plus
coûteuse que la méthode des divisions successives.

\subsubsection{Le raffinement de Kraitchik et l'usage des fractions continues}
