\subsection{Méthodes de factorisation de Fermat-Kraitchik et utilisation des
fractions continues}

Dans toute section, $N$ désigne un entier naturel composé impair.

\subsubsection{Méthodes de Fermat et Kraitchik}

La méthode de factorisation de Fermat par du constat suivant.

\begin{lemme}
	Factoriser $N$ est équivalent à l'exprimer comme différence de deux carrés
	d'entiers.
\end{lemme}

\begin{proof}
	En effet, si $N = u^2 - v^2, u, v\in \Z$ alors $N = (u-v)(u + v)$.
	Réciproquement si l'on a une factorisation $N = ab$, alors \[N =
	\left(\frac{a+b}{2}\right)^2 - \left(\frac{a-b}{2}\right)^2.\]
\end{proof}

La méthode de Fermat cherche donc à exploiter cette propriété en exprimant $N$
comme la différence de deux carrés et en déduire une factorisation. Celle-ci se
montre particulièrement efficace lorsque $N$ est le produit de deux entiers
proches l'un de l'autre. Notons $N=ab$ une factorisation de $N$,
$r=\frac{a+b}{2}$ et $s=\frac{a-b}{2}$. On a \[N = r^2 - s^2\] et que l'entier
$r$ est donc plus grand que $\sqrt{N}$ tout en lui étant proche. Il existe donc
un entier positif $u$ \emph{pas trop grand} tel que \[\ent{\sqrt{N}} + u = r\]
et donc tel que $(\ent{\sqrt{N}} + u)^2 - n$ soit un carré. Trouver un tel
entier $u$ donne alors la factorisation de $N$. Comme les facteurs de $N$ sont
proches l'un de l'autre, on le trouve par essais successis. \\

La méthode de Fermat n'est cependant pas du tout efficace lorsque les facteurs
de $N$ ne sont pas proches. D'après \NTS{MB}, la méthode est alors encore plus
coûteuse que la méthode des divisions successives. \\

Dans les années 1920, Maurice Kraitchik a raffiné la méthode de Fermat pour
améliorer son efficacité. Ses idées sont au cœur des algorithmes de
factorisations les plus performants en 2020. Son idée essentielle est que pour
factoriser $N$, il n'est pas \emph{nécessaire} de l'exprimer comme différence
de deux carrés ~; trouver une différence de deux carrés qui soit un multiple de
$N$ \emph{suffit}.

\begin{lemme}
	Conaître deux entiers $u, v \in \Z$ tels que $u^2 \equiv v^2 \equiv
	\pmod{N}$ et $u\not\equiv \pm v\pmod{N}$ fournit une factorisation de $N$.
\end{lemme}

\begin{proof}
	Posons $g =\pgcd(u-v, N)$ et $g' = \pgcd(u+v, N)$. Comme $u\not\equiv \pm
	v\pmod{N}$, on a $g<N$ et $g'<N$. Enfin ni $g$ et $g'$ ne sont réduits à
	$1$ : si l'un deux l'est, l'autre vaut $N$, contradiction. Donc $g$ et $g'$
	sont tous deux des facteurs non triviaux de $N$.
\end{proof}

\begin{remarque}
	Dans l'algorithme, nous nous contenterons de chercher des $u, v$ tels que
	$n$ divise la différence de leurs carrés, sans vérifier s'ils vérifient $u
	\not\equiv \pm v\pmod{N}$. Comme le polynôme $X^2 - v^2 \in \Z/N\Z$ a
	exactement quatre racines, il y a \og{} une chance sur deux \fg pour que
	$u$ et $v$ nous fournissentun facteur non trivial de $N$. \\
	\NTS{Margot tu confirmes qu'on vérifie pas si $u\equiv \pm v$ ?}
\end{remarque}

Comment trouver de tels couples $(u, v)$ ? Kraitchik a originellement utilisé
le polynôme $K := X^2 - N \in \Z[X]$. Trouver une famille d'entiers $x_1,
\cdots, x_k$ telle que $K(x_1)\cdots K(x_r)$ soit un carré fournit un tel
couple $(u, v)$. Il faut poser $u = x_1\cdots x_r$ et $v = K(x_1)\cdots K(x_r)$
: \[v^2 \equiv (x_1^2 - N)\cdots (x_k^2 - N) \equiv x_1^2 \cdots x_k^2 \equiv
u^2\pmod{N}.\] Cette méthode souffre toutefois d'un problème d'efficacité,
puisqu'il est nécessaire de calculer un grand nombre de $K(x_i)$. La croissance
de la fonction associée au polynôme $K$ étant quadratique, le coût des calculs
devient prohibitif. Nous allons maintenant voir que les fractions continues
sont une solution efficace à ce problème. \\

\subsubsection{Utilisation des fractions continues}

Redonnons quelques notations de la sous-section \NTS{réf} sur les fractions
continues. Nous notons $x := \sqrt{N}$, puis conformément à \NTS{réf} $x_0 :=
x$ et $x_n = \frac{1}{x_{n-1}- \ent{x_{n-1}}}$ pour tout $n\in \N^*$. Le
développement en fraction continue de l'irrationel $x$ est donc \[x = \hat{x}_0
+ \cfrac{1}{\hat{x}_1 + \cfrac{1}{\hat{x}_2 + \cfrac{1}{\hat{x}_3 + \dots}}}.\]
Sa $n$-ième réduite est pour tout $n$ un rationnel et s'exprime de fait comme
une fraction réduite $\frac{A_n}{B_n}$ où $A_n, B_n\in \Z$.  En posant
$\hat{x}_n := \ent{x_n}$ pour tout $n\in \N$, on a l'égalité \NTS{réf} \[x =
[\hat{x}_1, \dots, \hat{x}_{n-1}, x_n].\] Cet élément $x_n$ est un irrationel
quadratique et s'écrit de fait \NTS{réf} \[x_n = \frac{P_n + x}{Q_n},\quad P_n,
Q_n\in \Z,\] de sorte que
\begin{equation}
	A_{n-1}^2 \equiv (-1)^n Q_n \pmod{N}.
\end{equation}

\begin{definition}
	Pour tout $n\in \N^*$, on appelle \emph{$n$-ième paire $(A, Q)$} le couple
	$(A_{n-1}, Q_n)$.
\end{definition}

D'après l'identité \NTS{réf}, si l'on parvient à trouver une famille $n_1,
\dots, n_k$ d'indices tels que le produit $Q:=\prod_{i=1}^k (-1)^{n_i} Q_{n_i}$
soit un carré (dans $\Z$ et non uniquement dans $\Z/N\Z$) et $A := \prod_{i=1}^k
A_{n_i} \not\equiv \pm\sqrt{Q}$, nous aurons factorisé $N$ en vertu du lemme
\NTS{réf}.

\begin{definition}
	Un ensemble de paires $(A, Q)$ indexé par $n_1, \dots, n_k$ est dit
	\emph{valide} si le produit $\prod_{i=1}^k (-1)^{n_i} Q_{n_i}$ est un carré
	(dans $\Z$ et non uniquement dans $\Z/N\Z$).
\end{definition}

Le principal avantage de l'utilisation des fractions continues plutôt que le
polynôme de Kraitchik réside dans leur croissance. L'inégalité \NTS{réf} assure
que les éléments $Q_n, n\in \N$ seront plus petits (en valeur absolu) que les
$K(x'), x'\in \Z$ (dont la croissance lorsque $x'$ s'éloigne de $\sqrt{N}$ est
approximativement linéaire de pente $2\sqrt{N}$). Les paires $(A, Q)$ seront
donc plus faciles à manier. Enfin, il est facile de générer le développement en
fraction continue de $x$ et les paires $(A, Q)$ grace à un algorithme itératif
dû à Gau\ss et exposé dans \NTS{réf}. Nous allons à présent nous appliquer à
déterminer des paires $(A, Q)$ valides.

\subsubsection{Recherche de congruences de carrés}

La méthode présentée ici est celle de Kraitchik lui même et permet de
\emph{générer} des paires $(A, Q)$ valides sans passer par un algorithme de
recherche exhaustive. On commence par se fixier $B$ une base de factorisation,
c'est à dire un ensemble non vide fini de nombres premiers.

\begin{definition}
	Un entier $m\in \N\setminus \{0, 1\}$ est dit \emph{$B$-friable} si tous
	les facteurs premiers de $m$ sont dans $B$.
\end{definition}

L'idée principale de Kraitchik est que connaître suffisamment d'entiers $Q_n,
n\in \N$ $B$-friables \emph{entièrement factorisés} permet de construire un
ensemble valide de paires $(A, Q)$.

\begin{proposition}
	Soit $F$ une famille d'entiers $B$-friables. Si \[\# F \geqslant \# B +
	1\] alors on peut extraire une sous-famille de $F$ dont le produit des
	éléments est un carré.
\end{proposition}

\begin{proof}
	Posons $F = \{m_1, \dots, m_k\}$ (de sorte que $k = \# F)$ et $B = (p_1,
	\dots, p_r)$ (de sorte que $\# B = r$). Les éléments $m_j, 1\leqslant j
	\leqslant k$ s'écrivent alors \[m_j = \prod_{i=1}^r p_r^{v_{p_i}(m_j)}.\]
	Fixons $j, j'\in [\![1, k]\!]$. Puisque les éléments de $B$ sont fixés et
	en nombre fini, l'élément $B$-friable $m_j$ peut-être vu comme le vecteur
	\[v(m_j) := (v_{p_1}(m_j), \dots, v_{p_r}(m_j)).\] L'entier $m_j$ est un
	carré \ssi les composantes de son vecteur $v(m_j)$ sont paires, i.e. la
	réduction du vecteur $v(m_j)$ modulo $2$ est nulle. Par propriété des
	valuations, le vecteur associé au produit $m_j\cdot m_{j'}$ est le vecteur
	somme $v(m_j) + v(m_{j'})$. Autrement dit, le produit d'une sous-famille
	$\{m_{j_1}, \dots, m_{j_s}\}$ de $F$ est un carré \ssi les vecteurs
	$v(m_{j_1}), \dots, v(m_{j_s})$ somment à $0$ modulo $2$. Soit $V$ le
	$\mathbb{F}_2$-espace vectoriel de $\mathbb{F}_2^r$, qui est de
	$\mathbb{F}_2$ dimension $r$. Comme $k\geqslant r+1$, la famille $\{v(m_1),
	\dots, v(m_k)\}$ est liée dans $V$ et il existe de fait des éléments $l_1,
	\dots, l_k\in \mathbb{F}_2$ tels que \[\sum_{j=1}^r l_j v(m_j) = 0.\]
	L'élément $\prod_{j=1}^k m_j^{l_j}$ est alors un carré.
\end{proof}

\begin{corollaire}
	Il suffit de connaître $\#B + 2$ paires $(A, Q)$ pour en extraire une
	sous-famille de $Q_n$ dont les paires correspondantes soient valides.
\end{corollaire}

\begin{proof}
	Pour appliquer la proposition, il faut ajouter à la base $B$ l'élément $-1$
	qui n'est pas premier. En effet, nous ne cherchons pas à trouver des
	éléments $n_1, \dots, n_s$ pour lesquels $\prod_{i=1}^s Q_{n_i}$ est un
	carré mais $\prod_{i=1}^s (-1)^{n_i}Q_{n_i}$ l'est. On voit alors $B' :=
	B\cup \{-1\}$ comme une base de factorisation et on obtient le résultat en
	appliquant la proposition.
\end{proof}

Notons $B = \{p_1, \dots, p_r\}$. La preuve de la proposition fournit un
procédé d'algèbre linéaire pour extraire des paires $(A, Q)$ valides d'une
famille $F = Q_{n_1}, \dots, Q_{n_k}$ connue.  Tout d'abord, lesdits $Q_n$
doivent être factorisés (ce que nous ferons par divisions successives) et
$B$-friables. Soit en suite $M$ la matrice \[ M = \left(v_{p_i}(Q_{n_j})
\pmod{2}\right)_{1\leq i \leq r, 1\leq j \leq k}.\] Soient $l_{n_1}, \dots,
l_{n_k}$ les éléments de $\mathbb{F}_2$ donnés par la proposition et tels que
$\prod_{j=1}^k (-1)^{n_j} Q_{n_j}$ soit un carré. Le vecteur $(l_{n_1}, \dots,
l_{n_k})$ est un élément du noyau de la matrice transposée de $M$. Un tel
élément est facilement produit avec des algorithmes usuels d'algèbres
linéaires. Nous verrons dans la prochaine section une version adaptée du pivot
de Gau\ss{} qui nous permet d'en produire.
