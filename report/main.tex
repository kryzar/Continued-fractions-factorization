% main.tex

\documentclass[a4paper, 12pt, oneside]{article}
\usepackage[french]{babel}
\usepackage{mathtools}
\usepackage[backend=biber, style=alphabetic]{biblatex}
\usepackage{pkg}
\usepackage[linesnumbered, boxruled, vlined, onelanguage, french]{algorithm2e}
% modifier Output en Sorties dans algorithm2e
\SetKwInput{KwOut}{Sorties}

\usepackage{array}
\usepackage{graphicx}
\newcolumntype{M}[1]{>{\raggedright}m{#1}}

\addbibresource{bi.bib}

\DeclarePairedDelimiter\ent{\lfloor}{\rfloor}
\DeclareMathOperator{\e}{e} % pour l'exponentielle
\newcommand{\en}[1]{(en: \textit{#1})}

\title{Factorisation par fractions continues}
\date{Février 2021}
\author{Margot Funk, Antoine Hugounet}

\begin{document}
\maketitle

\newpage
\tableofcontents

\newpage
\section*{Introduction}
\addcontentsline{toc}{section}{Introduction}
% introduction.tex

\begin{frame}
\begin{center}
\begin{chronology}[10]{1919}{1970}{12cm}
\event{1920}{Méthode de Kraitchik}
\pause
\event{1931}{Idée des fractions continues, Lehmer et Powers}
\pause
\event{1970}{Factorisation de $F_7$, Morrison et Brillhart}
\end{chronology}
\end{center}
\end{frame}

\begin{frame}{Table des matières}
\tableofcontents Projet basé sur \textit{A Method of Factoring and the
Factorization of $F_7$}, de M. A. \textsc{Morrison} et J. \textsc{Brillhart},
dans Mathematics of Computation 29.129 (1975).
\end{frame}


\section{Théorie}
\subsection{Fraction continues}
\subsubsection{Intuition}

Intuitivement, une fraction continue est une expression — finie ou infinie — de
la forme suivante\footnote{Notez que nous ne nous autorisons que des $1$ aux
numérateurs.} : \[a_0 + \cfrac{1}{a_1 + \cfrac{1}{a_2 + \cfrac{1}{a_3 +
\dots}}}\] telle que $a_0\in \Z$ et $a_i \in \N^*$ pour tout $i\in \N^*$.
Toujours intuitivement, nous voulons affûbler cette fraciton continue d'une
valeur. Si la fraction continue est finie, cette une bonne vieille fraciton,
c'est à dire un élément du corps $\Q$~; si la fraction continue est infinie, on
calcule d'abord $a_0$, puis $a_0 + \frac{1}{a_1}$, puis $a_0 + \frac{1}{a_1
+\frac{1}{a_2}}$, et on continue une infinité de fois. La limite de la suite
générée est la \og{} valeur \fg{} de la fraction continue. Nous ferons sens
plus précis de l'intuition dans la prochaine sous-section. \\

Les fractions continues émanent de la volonté d'approcher des réels irrationels
par des fractions d'entiers. Par exemple, la fraction $\frac{103 993}{33 102}$
approche $\pi$ avec une précision meilleure que le milliardième.  Comment
générer une telle fraction continue pour un réel irrationnel $x$ ?  On part de
l'identité $x = x + \ent{x} - \ent{x}$ et l'on écrit \[x = \ent{x} +
\cfrac{1}{\cfrac{1}{x - \ent{x}}}.\] On pose $x_0 = x$ et $x_1 = \frac{1}{x_0 -
\ent{x_0}}$ (qui est bien défini par irrationalité de $x$) et l'on répète la
première étape sur $x_1$ : \[x = \ent{x_0} + \cfrac{1}{x_1} = \ent{x_0} +
\cfrac{1}{\ent{x_1} + \cfrac{1}{\cfrac{1}{x_1 - \ent{x_1}}}}.\] Comme le réel
$x$ est irrationnel, on peut répéter ce procédé indéfiniment. Nous construisons
alors la suite d'éléments \emph{irrationnels} de terme général \[x_n =
\frac{1}{x_{n-1} - \ent{x_{n-1}}}, \quad \forall n \geq 1.\] On associe alors à
\emph{l'irrationnel} $x$ la fraction continue \emph{infinie}\footnote{Lorsque
nous aurons correctement défini la notion de fraction continue, cette fraction
continue canoniquement associée à $x$ sera notée $\hat{x}$.} \[a_0 +
\cfrac{1}{a_1 + \cfrac{1}{a_2 + \cfrac{1}{a_3 + \dots}}},\] où l'on a posé
\[a_i = \ent{x_i}\] pour tout $i\in \N$. Fixons ces notations :

\begin{notation}\label{notations}
	Soit $x\in \R$ un élément irrationel. Notons $x_0 = x$ puis \[x_n :=
	\frac{1}{x_{n-1} - \ent{x_n}}, \quad \forall n \geq 1.\] Par ailleurs,
	notons \[\hat{x}_n := \ent{x_n}, \quad \forall n\in \N.\]
\end{notation}

\begin{remarque}
	La méthode de construction d'une fraction continue \emph{finie} pour un
	rationnel est la même : il faut simplement s'arrêter lorsque l'on tombe sur
	un $a_n$ vérifiant $a_n = \ent{a_n}$. Cet algorithme termine (\NTS{ref}) et
	s'exécute plus simplement en utilisant… l'algorithme d'Euclide. Par
	ailleurs, réaffirmons que la fraction continue d'un irrationel (encore une
	fois, dans un sens qui sera précisé au prochain paragraphe) est forcément
	infinie.
\end{remarque}

\subsubsection{Formalisation}

Formellement, on peut définir\footnote{La définition mathématique est
descriptive et non prescriptive.} une fraction continue ainsi :

\begin{definition}[Fraction continue]
	On appelle \emph{fraction continue} toute suite non vide (finie ou infinie)
	$(a_i)_{i\in U} \in \N^\N$, $U\subset \N$, d'entiers qui vérifie \[a_i
	\geq 1, \quad \forall i\in U\setminus\{0\}.\] Cette suite est alors notée
	\[a_0 + \cfrac{1}{a_1 + \cfrac{1}{a_2 + \cfrac{1}{a_3 + \dots}}}.\]
\end{definition}

\begin{notation}
	Soit $x\in \R$ un élément irrationel. On note $\hat{x}$ la fraction
	continue infinie canoniquement associée à $x$ par la méthode exposée dans
	le premier paragraphe. Autrement dit, $\hat{x}$ est la fraction continue
	donnée par la suite infinie (voir \ref{notations}) $(\hat{x}_i)_{i\in
	\N})$.
\end{notation}

Il est naturel d'associer à une fraction continue (finie ou infinie) une suite
(finie ou infinie) de fractions \og intermédiaires \fg{} appelées
\emph{réduites}. Pour n'avoir aucun problème de division par zéro, nous nous
plaçons temporairement dans un corps de fractions rationnelles en $\N$
indeterminées.

\begin{definition}[Réduites formelles]
	Soit $(X_i)_{i\in \N}$ une suite (infinie) d'indeterminées sur le corps
	$\Q$. On définit \[[X_0] = X_0]\] puis par récurrence \[[X_1, \dots,
	X_n] = X_0 + \frac{1}{[X_1, \dots, X_n]}.\] Ces éléments sont dans
	$\Q((X_i)_{i\in \N})$.
\end{definition}

\begin{definition}[Réduites d'une fraction continue]
	Distinguons les cas finis et infinis. Soit $f$ une fraction
	continue.
	\begin{itemize}
		\item Si $f$ est donnée par la suite finie $(a_0, \dots, a_n)$, pour
		tout $k\in [\![0, n]\!]$ on appelle \emph{$k$-ième réduite de $f$}
		l'élément $[a_0, \dots, a_k]$.
		\item Si $f$ est donnée par la suite infinie $(a_i)_{i\in \N}$, pour
		tout $k\in \N$ on appelle \emph{$k$-ième réduite de $f$} l'élément
		$[a_0, \dots, a_k]$.
	\end{itemize}
\end{definition}

\begin{exemple}
	Soit $f$ la fraction continue infinie donnée par la suite $(1)_{i\in \N}$.
	La première réduite est $[1] = 1$, la deuxième est \[[1, 1] = 1 +
	\frac{1}{[1]} = 1 + \frac{1}{1},\] la troisième est \[[1, 1, 1] = 1 +
	\frac{1}{[1, 1]} = 1 + \frac{1}{1 + \cfrac{1}{1}}.\] Plus généralement, la
	$k$-ième réduite de $f$ est de la forme \[[1, 1, \dots, 1] = 1 +
	\cfrac{1}{1 + \cfrac{1}{\dots \\ \cfrac{1}{1 +
	\cfrac{1}{1}}}}.\]
\end{exemple}

Remarquons que les réduites de toute fraction continue sont des éléments
rationnels, ce même si la fraction continue est égale à $\hat{x}$ pour un
certain irrationel $x$. De fait, $x$ n'est égal à aucune des réduites de
$\hat{x}$. Les fractions continues finies ne se comportent pas pareil puisque
toute fraction continue finie donnée par la suite $(a_1, \dots, a_n)$ est
\emph{égale}\footnote{Au sens bien connu de l'égalité dans le corps $\Q$} à la
réduite $[a_1, \dots, a_n]$. Toutefois, en reprenant les notations
\ref{notations}, on a
\begin{equation}
	x = [\hat{x}_1, \dots, \hat{x}_{n-1}, x_n], \quad \forall x\in \N.
\end{equation}
Cette égalité sera fondamentale dans notre algorithme de factorisation.

\subsubsection{Irrationels quadratiques}

\subsection{Méthodes de factorisation de Fermat-Kraitchik et utilisation des
fractions continues}

Dans toute cette section, $N$ désigne un entier naturel composé impair.

\subsubsection{Méthodes de Fermat et Kraitchik}

La méthode de factorisation de Fermat part du constat suivant.

\begin{lemme}
	Factoriser $N$ est équivalent à l'exprimer comme différence de deux carrés
	d'entiers.
\end{lemme}

\begin{proof}
	En effet, si $N = u^2 - v^2, u, v\in \Z$ alors $N = (u-v)(u + v)$.
	Réciproquement si l'on a une factorisation $N = ab$, alors \[N =
	\left(\frac{a+b}{2}\right)^2 - \left(\frac{a-b}{2}\right)^2.\]
\end{proof}

La méthode de Fermat cherche donc à exploiter cette propriété en exprimant $N$
comme la différence de deux carrés pour en déduire une factorisation. Celle-ci se
montre particulièrement efficace lorsque $N$ est le produit de deux entiers
proches l'un de l'autre. Notons $N=ab$ une factorisation de $N$,
$r=\frac{a+b}{2}$ et $s=\frac{a-b}{2}$. On a \[N = r^2 - s^2\] et que l'entier
$r$ est donc plus grand que $\sqrt{N}$ tout en lui étant proche. Il existe donc
un entier positif $u$ \emph{pas trop grand} tel que \[\ent{\sqrt{N}} + u = r\]
et donc tel que $(\ent{\sqrt{N}} + u)^2 - N$ soit un carré. Trouver un tel
entier $u$ donne alors la factorisation de $N$. Comme les facteurs de $N$ sont
proches l'un de l'autre, on le trouve par essais successifs. \\

La méthode de Fermat n'est cependant pas du tout efficace lorsque les facteurs
de $N$ ne sont pas proches. D'après \NTS{MB}, la méthode est alors encore plus
coûteuse que la méthode des divisions successives. \\

Dans les années 1920, Maurice Kraitchik a raffiné la méthode de Fermat pour
améliorer son efficacité. Ses idées sont au cœur des algorithmes de
factorisations les plus performants en 2020. Son idée essentielle est que pour
factoriser $N$, il n'est pas \emph{nécessaire} de l'exprimer comme différence
de deux carrés ~; trouver une différence de deux carrés qui soit un multiple de
$N$ \emph{suffit}.

\begin{lemme}
	Connaître deux entiers $u, v \in \Z$ tels que $u^2 \equiv v^2 
	\pmod{N}$ et $u\not\equiv \pm v\pmod{N}$ fournit une factorisation de $N$.
\end{lemme}

\begin{proof}
	Posons $g =\pgcd(u-v, N)$ et $g' = \pgcd(u+v, N)$. Comme $u\not\equiv \pm
	v\pmod{N}$, on a $g<N$ et $g'<N$. Enfin ni $g$ et $g'$ ne sont réduits à
	$1$ : si l'un deux l'est, l'autre vaut $N$, contradiction. Donc $g$ et $g'$
	sont tous deux des facteurs non triviaux de $N$.
\end{proof}

\begin{remarque}
	Dans l'algorithme, nous nous contenterons de chercher des $u, v$ tels que
	$N$ divise la différence de leurs carrés, sans vérifier s'ils vérifient $u
	\not\equiv \pm v\pmod{N}$. Comme le polynôme $X^2 - v^2 \in \Z/N\Z$ a
	exactement quatre racines, il y a \og{} une chance sur deux \fg  pour que
	$u$ et $v$ nous fournissentun facteur non trivial de $N$. \\
\end{remarque}

Comment trouver de tels couples $(u, v)$ ? Kraitchik a eu l'idée de ne chercher
dans un premier temps que des congruences de la forme : \[u_i^2 \equiv v_i 
\pmod{N}\] Il a remarqué qu'il suffisait d'exhiber une famille $v_1, \cdots,
v_r$ telle que $\prod_{i=1}^r v_i$ soit un carré pour savoir comment obtenir 
une congruence de carrés par multiplication de congruences. \\

Il originellement utilisé le polynôme $K := X^2 - N \in \Z[X]$ qui fournit la
congruence $u_i^2 \equiv K(u_i) \pmod{N}$. Si on sait que  $K(u_1)\cdots K(u_r)$ 
est un carré, en posant $u = u_1\cdots u_r$ et $v = \sqrt{ K(u_1)\cdots K(u_r)}$ 
on a bien :
\[v^2 \equiv K(u_1)\cdots K(u_r) \equiv u_1^2 \cdots u_k^2 \equiv u^2\pmod{N}\]

Cette méthode souffre toutefois d'un problème d'efficacité, puisqu'il est nécessaire 
de calculer un grand nombre de $K(x_i)$. La croissance de la fonction associée 
au polynôme $K$ étant quadratique, le coût des calculs devient prohibitif.\\

\NTS{Je suis pas vraiment d'accord que le problème vient de là, pour moi le problème
vient de la nécessité de trouver des $K(u_i)$B-friables, comme on a pas de borne c'est
moins bien que les fractions continues, mais pas du fait de calculer un
carré meme si c'est grand on sait le faire. Crible quadratique utilise le même 
polynome mais ajoute un crible bien pensé à la place des divisions successives. 
J'enleverais la remarque et garderais suelement celle qui vient après. }

Il reste à savoir comment trouver cette famille $v_1 \cdots v_r$.\\

\subsubsection{Recherche de congruences de carrés}

Une méthode qui répond à cette question sans passer par un algorithme de recherche
exhaustive a été proposée par Kraitchik lui-même. \\ 

\NTS{Morrison et Brillart, dans leur article exposant la méthode de factorisation
avec les fractions continues, ont répondu à cette question. (garder la bonne
version, Pomerancedit M et B)}

Celle-ci nécessite de travailler avec des congruences de la forme 
\[u_i^2 \equiv Q_i \pmod{N}\] avec $\lvert Q_i \rvert$ suffisamment petit pour 
en connaître une factorisation. (La notation $Q_i$ fera échos à celle utilisée 
pour décrire la méthode de factorisation avec les fractions continues). \\

On commence par se fixer $B$ une base de factorisation, c'est à dire un 
ensemble non vide fini de nombres premiers.

\begin{definition}
	Un entier $Q\in \N\setminus \{0, 1\}$ est dit \emph{$B$-friable} si tous
	les facteurs premiers de $Q$ sont dans $B$.
\end{definition}

L'idée principale de Kraitchik \NTS{ou M et B} est que connaître suffisamment
d'entiers $Q_i \in \Z$ avec $ \lvert Q_i \rvert$ $B$-friables 
\emph{entièrement factorisés} permet de trouver la famille recherchée.

\begin{proposition}
	Soit $F$ une famille d'entiers tels que leur valeur absolue est $B$-friables.
    Si \[\# F \geqslant \# B + 2\] alors on peut extraire une sous-famille de
    $F$ dont le produit des éléments est un carré.
\end{proposition}

\begin{proof}
	Posons $F = \{Q_1, \dots, Q_k\}$ (de sorte que $k = \# F)$ et $B = (p_1,
	\dots, p_m)$ (de sorte que $\# B = m$). Les éléments $Q_j, 1\leqslant j
    \leqslant k$ s'écrivent alors, comme $\lvert Q_j \rvert$ est $B$-friable,
    \[Q_j = (-1)^{v_0}\prod_{i=1}^m p_i^{v_{p_i}(Q_j)}.\]
	Fixons $j, j'\in [\![1, k]\!]$. Puisque les éléments de $B$ sont fixés et
	en nombre fini l'élément $Q_j$ peut être vu comme le vecteur
    \[v(Q_j) := (v_{p_m}(Q_j), \dots, v_{p_1}(Q_j), v_0 )\] L'entier $Q_j$ est
    un carré \ssi les composantes de son vecteur $v(Q_j)$ sont paires, i.e. la
	réduction du vecteur $v(Q_j)$ modulo $2$ est nulle. Par propriété des
	valuations, le vecteur associé au produit $Q_j\cdot Q_{j'}$ est le vecteur
	somme $v(Q_j) + v(Q_{j'})$. Autrement dit, le produit d'une sous-famille
	$\{Q_{j_1}, \dots, Q_{j_s}\}$ de $F$ est un carré \ssi les vecteurs
	$v(Q_{j_1}), \dots, v(Q_{j_s})$ somment à $0$ modulo $2$. Soit $V$ le
    $\mathbb{F}_2$-espace vectoriel $\mathbb{F}_2^{m+1}$, qui est de dimension
    $m+1$. Comme $k\geqslant m+2$, la famille $\{v(Q_1), \dots, v(Q_k)\}$ est 
    liée dans $V$ et il existe de fait des éléments $l_1,\dots, l_k\in 
    \mathbb{F}_2$ tels que \[\sum_{j=1}^k l_j v(Q_j) = 0.\] L'élément 
    $\prod_{j=1}^k Q_j^{l_j}$ est alors un carré.
\end{proof}

\begin{definition}
    Soient $B = \{p_1,\dots, p_m \}$ une base de factorisation et $Q$ un 
    entier avec $\lvert Q \rvert$  $B$-friable. Si $Q = (-1)^{v_0} 
    \prod_{i=1}^m p_i^{v_{p_i}(Q)}$, on appellera vecteur exposant associé
    à l'entier $Q$ le vecteur \[v(Q) := (v_{p_m}(Q), \dots, v_{p_1}(Q), v_0 )
    \in \mathbb{F}_2^{m+1} \] 
\end{definition}

\begin{remarque}
    Cette définition sera réutilisée lors de la description de l'algorithme
    de factorisation avec la méthode des fractions continues. L'ordre des
    composantes du vecteur présentée ici vient du fait que le $v_0$ correspondra
    au bit de poids faible et le $v_{p_m}(Q)$ au bit de poids fort du vecteur. \\ 
\end{remarque}

Notons $B = \{p_1, \dots, p_m\}$ une base de factorisation. Etant données
des congruences $u_i^2 \equiv Q_i \pmod{N}$, la preuve de la proposition 
fournit un procédé d'algèbre linéaire pour extraire une familles $Q_1,
\cdots, Q_r$ telle que $\prod_{i=1}^r Q_i$ soit un carré. Tout d'abord,
il faut trouver des $Q_i$ tels que $ \lvert Q_i \rvert$ est $B$-friable
(ce que nous ferons par divisions successives) et retenir leur vecteur 
exposant associé. \NTS{est-ce que la description de la matrice ça va ?}. 
Disons qu'on a pu en trouver $k$, notés $Q_1, \cdots, Q_k$.
Soit ensuite $M$ la matrice $k\times(m+1)$ dont les lignes correspondent
aux vecteurs exposants de $Q_{1}, \cdots, Q_{k}$. Soient $l_{1}, \dots,
l_{k}$ les éléments de $\mathbb{F}_2$ donnés par la proposition et tels que
$\prod_{j=1}^k Q_{j}^{l_j} $ soit un carré. Le vecteur $(l_1, \dots,
l_{k})$ est un élément du noyau de la matrice transposée de $M$. Un tel
élément est facilement produit avec des algorithmes usuels d'algèbres
linéaires. Nous verrons plus tard une version adaptée du pivot
de Gau\ss{} qui nous permet d'en produire.

\subsubsection{Utilisation des fractions continues}

Pour appliquer la méthode ci-dessus, il faut arriver à générer des $Q_i$ 
avec $\lvert  Q_i \rvert$ $B$-friable pour une base de factorisation donnée,
supposée \og{} pas très grande \fg. Il faut donc chercher à avoir des
$\lvert Q_i \rvert$ \og{} petits \fg. L'introduction des fractions continues
est motivée par le constat suivant : si $u_i^2 = Q_i + kNb^2$ avec $\lvert 
Q_i \rvert$ petit, alors $\left(\frac{u_i}{b}\right)^2 - kN = \frac{Q_i}{b^2}$
est petit en valeur absolue et $\frac{u_i}{b}$ est une bonne approximation de 
$\sqrt{kN}$.

\NTS{mettre référence}

Redonnons quelques notations de la sous-section \NTS{réf} sur les fractions
continues. Nous notons $x := \sqrt{N}$, puis conformément à \NTS{réf} $x_0 :=
x$ et $x_n = \frac{1}{x_{n-1}- \ent{x_{n-1}}}$ pour tout $n\in \N^*$. Le
développement en fraction continue de l'irrationel $x$ est donc \[x = \hat{x}_0
+ \cfrac{1}{\hat{x}_1 + \cfrac{1}{\hat{x}_2 + \cfrac{1}{\hat{x}_3 + \dots}}}.\]
Sa $n$-ième réduite est pour tout $n$ un rationnel et s'exprime de fait comme
une fraction réduite $\frac{A_n}{B_n}$ où $A_n, B_n\in \Z$.  En posant
$\hat{x}_n := \ent{x_n}$ pour tout $n\in \N$, on a l'égalité \NTS{réf} \[x =
[\hat{x}_1, \dots, \hat{x}_{n-1}, x_n].\] Cet élément $x_n$ est un irrationel
quadratique et s'écrit de fait \NTS{réf} \[x_n = \frac{P_n + x}{Q_n},\quad P_n,
Q_n\in \Z,\]. 

L'algorithme de factorisation repose sur l'égalité suivante : \[ A_{n-1}^2 = 
(-1)^n Q_n + kN B_{n-1}^2 \]

On a donc : \[ A_{n-1}^2 \equiv (-1)^n Q_n \pmod{N} \] \\


Donnons à présent quelques définitions dont nous nous servirons pour décrire 
l'algorithme des fractions continues. 

\begin{definition}
	Pour tout $n\in \N^*$, on appelle \emph{$n$-ième paire $(A, Q)$} le couple
	$(A_{n-1}, Q_n)$.
\end{definition}

\begin{definition}
	Un ensemble de paires $(A, Q)$ indexé par $n_1, \dots, n_k$ est dit
	\emph{valide} si le produit $\prod_{i=1}^k (-1)^{n_i} Q_{n_i}$ est un carré
	(dans $\Z$ et non uniquement dans $\Z/N\Z$).
\end{definition}

\begin{remarque}
    Par abus de langage, nous parlerons à présent du vecteur exposant 
    associé à $Q_n$ pour désigner le vecteur exposant associé à 
    $(-1)^n Q_n = (-1)^{v_0} \prod_{i=1}^m p_i^{v_{p_i}(Q)}$,
    c'est-à-dire $(v_{p_m}(Q_n),\dots, v_{p_1}(Q_n), v_0 ) \in \mathbb{F}_2^{m+1}$
\end{remarque}

D'après la section précédente, étant donnée une base de factorisation $B$,
la méthode consiste à : 

\begin{itemize}
    \item Calculer des couples $(A,Q)$ par développement en fractions continues 
        de $\sqrt(kN)$ où k est un petit coefficient multiplicateur (voir plus loin)
    \item Sélectionner les $Q_n$ $B$-friables et leur associer un vecteur exposant
    \item Trouver un ensemble valide de paires $(A, Q)$ par pivot de Gauss 
          sur la matrice dont les lignes sont formées de ces vecteurs exposants
\end{itemize}

Nous aurons donc déterminé une famille $n_1,\dots, n_k$ d'indices tels que le 
produit $Q:=\prod_{i=1}^k (-1)^{n_i} Q_{n_i}$ soit un carré (dans $\Z$ et non 
uniquement dans $\Z/N\Z$). \\
Posons $A := \prod_{i=1}^k A_{n_i}$. Si $A\not\equiv \pm  \sqrt{Q}\pmod{N}$, 
nous aurons factorisé $N$ en vertu du lemme
\NTS{réf}.\\

Le principal avantage de l'utilisation des fractions continues plutôt que le
polynôme de Kraitchik réside dans leur croissance. L'inégalité \NTS{réf} assure
que les éléments $Q_n, n\in \N$ seront plus petits (en valeur absolu) que les
$K(x'), x'\in \Z$ (dont la croissance lorsque $x'$ s'éloigne de $\sqrt{N}$ est
approximativement linéaire de pente $2\sqrt{N}$). Les paires $(A, Q)$ seront
donc plus faciles à manier et les $Q_n$ auront plus de chance d'être $B$-friables.
Enfin, il est facile de générer le développement en fraction continue de $x$ et 
les paires $(A, Q)$ grace à un algorithme itératif dû à Gau\ss et exposé dans 
\NTS{réf}. \\

\NTS{Pour moi c'est plus ça l'avantage : } \\

Le principal avantage qu'offre l'utilisation des fractions continues par 
rapport au polynome de Kraitchik réside dans la croissance des termes $Q_n,
n\in \N$. On sait par \NTS{réf} que les $Q_n$ seront inférieurs à 
$2 \sqrt{kN}$. Les $K(x)$ donnés par le polynôme de Kratichik ont eux une croissance
lorsque $x$ s'éloigne de $\sqrt{N}$ approximativement linéaire de pente
$2 \sqrt{N}$. Les $Q_n$ auront donc plus de chance d'être  $B$-friables que les
$K(x)$ de Kraitchik. Or, l'étape la plus coûteuse de l'algorithme est celle de la
recherche des termes $B$-friables par divisions successives. Enfin, notons qu'il
est facile de générer le développement en fraction continue de $x$ et les paires
$(A, Q)$ grace à un algorithme itératif dû à Gau\ss et exposé dans \NTS{réf}. \\

\NTS{A rajouter je pense dans cette partie : critère pour sélectionner la base
de factorisation, le problème de la périodicité d'où introduction de k, l'idée
à la base de la large prime variation + dans une autre sous-section des trucs sur 
la complexité}


\section{Explication du programme}
Dans cette §, nous décrivons notre implémentation de la méthode de
factorisation des fractions continues.
\subsection{Architecture du programme}

\subsubsection{Terminologie}

Enonçons pour commencer quelques définitions qui seront utiles pour décrire le code.


\begin{definition}
	On dira qu'un couple $(A_{n-1}, Q_n)$ est une \emph{paire $(A, Q)$}.
\end{definition}

\begin{definition}
	Un ensemble de paires $(A, Q)$ indexé par $n_1, \dots, n_r$ est dit
	\emph{valide} si le produit $\prod_{i=1}^r (-1)^{n_i} Q_{n_i}$ est un carré
	(dans $\Z$ et non uniquement dans $\Z/N\Z$).
\end{definition}

\begin{definition}
    Si $B$ est la base de factorisation utilisée par le programme, on 
    désignera par l'expression \emph{vecteur exposant associé à $Q_n$} 
    le $B$-vecteur exposant $v_B((-1)^n Q_n)$.
\end{definition}

\subsubsection{Structure générale}

Notre programme comprend deux étapes principales. La première consiste à générer, 
à partir du développement en fractions continues de $ \sqrt{kN} $, des paires 
$(A, Q)$ avec $Q_n$ friable pour une base de factorisation préalablement déterminée.
On associe à chaque $Q_n$ ainsi produit son vecteur exposant \texttt{mpz\_t exp\_vect}.
Ce vecteur permet de retenir les nombres premiers qui interviennent dans la 
factorisation de $Q_n$ avec une valuation impaire. Dans le but d'augmenter le nombre
de paires $(A,Q) $ acceptées lors de cette étape, nous avons implémenté la 
\textit{large prime variation}. Celle-ci permet d'accepter une paire si $Q_n$ se
factorise  grâce aux premiers de la base de factorisation et à un grand facteur
premier supplémentaire. Les fonctions de cette phase de collecte sont rassemblées
dans le fichier \texttt{step\_1.c}. Elles font appel, pour mettre en oeuvre la 
\textit{large prime variation}, aux fonctions du fichier \texttt{lp\_var.c}. \\
 
Ces données sont traitées lors de la seconde phase dans l'espoir de trouver un 
facteur non trivial de $N$. Il s'agit de trouver des ensembles valides de paires
$(A, Q)$ par pivot de Gauss sur la matrice dont les lignes sont formées des 
vecteurs exposants. Chaque ensemble valide est à l'origine d'une congruence de 
la forme $A^2 \equiv Q^2$ (mod $N$) permettant potentiellement de trouver un
facteur non trivial de $N$. Les fonctions de cette phase sont regroupées dans 
le fichier \texttt{step\_2.c}. \\

Avant d'effectuer la première étape, il convient de se doter d'une base de 
factorisation. Ceci est permis par une des fonctions de \texttt{init\_algo.c}.
Ces dernières se chargent plus généralement de l'initialisation et du choix par
défaut des paramètres. \\

Finalement, en mettant bout à bout les deux étapes, la fonction 
\texttt{contfract\_factor} du fichier \texttt{fact.c} recherche un facteur
non trivial de $N$ et \texttt{print\_results} affiche les résultats. 

\subsubsection{Entrées et sorties}
Nous avons regroupé dans une structure \texttt{Params} les paramètres d'entrée 
de la fonction de factorisation, à savoir :

\begin{itemize}
    \item \texttt{N} : le nombre à factoriser, supposé produit de deux grands
                       nombres premiers.
    \item \texttt{k} : le coefficient multiplicateur.
    \item \texttt{n\_lim} : le nombre maximal de paires $(A,Q)$ que l'on 
                             s'autorise à calculer. Ce nombre prend en compte
                             toutes les paires produites et non uniquement les
                             paires avec $Q_n$ friable ou résultant de la  
                             \textit{large prime variation}.
    \item \texttt{s\_fb} : la taille de la base de factorisation. 
    \item \texttt{nb\_want\_AQp} : le nombre désiré de paires $(A,Q)$ avec $Q_n$ 
                                    friable ou résultant de la \textit{large prime 
                                   variation}.
                               \item des booléens indiquant si la \textit{early 
                                   abort strategy} ou la \textit{large prime variation}
                                   doivent être utilisées et des paramètres s'y rapportant.

\end{itemize}

Le programme stocke dans une structure \texttt{Results} un facteur non trivial
de \texttt{N} trouvé (si tel est le cas) ainsi que des données permettant 
l'analyse des performances de la méthode.  

\begin{remarque}
L'efficatité de la méthode dépend du choix des paramètres ci-dessus. Pour avoir
plus de latitude dans les tests, nous les considérons comme des paramètres 
d'entrée du programme. C'est pourquoi notre programme ne s'attèle pas à la
factorisation complète d'un entier, qui aurait nécessité une sous-routine 
déterminant des paramètres optimaux en fonction de la taille de l'entier 
dont on cherche un facteur. 
   
\end{remarque}

\begin{remarque}
Notre programme n'est pas supposé prendre en entrée un nombre admettant un petit
facteur premier (inférieur aux premiers de la base de factorisation par exemple).
En effet, comme il ne teste pas au préalable si \texttt{N} est divisible par de 
petits facteurs, il mettra autant de temps à trouver un petit facteur qu'un
grand facteur.
\end{remarque}

\subsection{Pivot de Gauss et recherche d'un facteur non trivial :
\texttt{step\_2.c}}

Avant de nous pencher sur les détails de la phase de collecte, regardons 
l'implémentation de la seconde phase, qui aide à mieux comprendre la 
forme sous laquelle nous collectons les données.

\subsubsection{Utilisation des données collectées}

A l'issue de la première phase, on espère avoir collecté \texttt{nb\_want\_AQp}
paires $(A,Q)$ avec $Q_n$ friable\footnote{Les paires peuvent aussi résulter de
la \textit{large prime variation}, cela n'a aucune incidence sur les fonctions
de cette partie.}. Le nombre réel de telles paires est stocké dans le champ
\texttt{nb\_AQp} d'une structure \texttt{Results} (voir sous-§ précédente). Une
paire $(A,Q)$ collectée est caractérisée par :

\begin{itemize}
    \item la valeur $A_{n-1}$,
    \item la valeur $Q_n$,
    \item le vecteur exposant associé à $Q_n$,
    \item un vecteur historique (voir ci-contre).
\end{itemize}

Les données de ces \texttt{nb\_AQp} paires sont stockées dans quatre tableaux :
\texttt{mpz\_t *Ans}, \texttt{mpz\_t *Qns}, \texttt{mpz\_t *exp\_vects} et
\texttt{mpz\_t *hist\_vects}. À un indice correspond une paire $(A,Q)$ donnée.
Le vecteur historique sert à indexer les paires collectées pour former un
analogue de la matrice identité utilisée pendant le pivot de Gauss. Plus
précisément, \texttt{hist\_vects[i]} est, avant pivot de Gauss, le vecteur
$(e_{l-1}, \cdots, e_0)$ où $l=\texttt{nb\_AQp}$ et $e_j = \delta_{ij}$. À
partir de ces quatre tableaux, la fonction \texttt{find\_factor} cherche un
facteur de \texttt{N} selon la méthode des congruences de carrés. Elle utilise
pour cela les fonctions auxiliaires \texttt{gauss\_elimination} et
\texttt{calculate\_A\_Q}. 

\subsubsection{La fonction \texttt{gauss\_elimination}}

La fonction \texttt{gauss\_elimination} effectue un pivot de Gauss sur les
éléments de \texttt{mpz\_t *exp\_vects}, vus comme les vecteurs-lignes d'une
matrice. Comme pour un pivot de Gauss classique, les calculs effectués sur les
vecteurs exposants sont reproduits en parallèle sur la matrice identité,
c'est-à-dire sur les éléments de \texttt{mpz\_t *hist\_vects}. Si le
\textit{xor} de deux vecteurs exposants donne le vecteur nul, cela signifie
qu'une relation de dépendance a été trouvée. On inscrit alors dans un tableau
l'indice de ce vecteur nul. Le vecteur historique dudit indice indique les
paires $(A,Q)$ de l'ensemble valide trouvé. La procédure que nous avons 
implémentée est décrite ci-dessous. 

\vspace{1em}
\begin{algorithm}[H]
\DontPrintSemicolon
\caption{\sc{Pivot de Gauss}}
\KwIn{tableau $\mathrm{exp\_vects}[0 \cdots nb\_AQp -1 ] $ des vecteurs
	exposants, tableau $\mathrm{hist\_vects}[0 \cdots nb\_AQp -1 ]$ des
	vecteurs historiques}
\vspace{0.5em}
\KwOut{$\mathrm{hist\_vects}[0 \cdots nb\_AQp -1]$ après le pivot, le nombre
	$nb\_lin\_rel$ de relations linéaires trouvées, $\mathrm{lin\_rel\_ind}[0
	\cdots nb\_lin\_rel-1]$ contenant les indices des lignes où une relation
	linéaire a été trouvée}
\vspace{0.5em}
créer tableau $\mathrm{msb\_ind}[0 \cdots nb\_AQp - 1]$\;  
créer tableau $\mathrm{lin\_rel\_ind}$\; 
$nb\_lin\_rel \gets 0$\; 
\vspace{0.5em}
\tcc{Initialisation du tableau \textsc{msb\_ind} : \textsc{Msb}(x) renvoie $0$
	si x est nul, l'indice du bit de poids fort de x sinon. Les indices des
	bits sont  numérotés de 1 à l'indice du bit de poids fort.}
\vspace{0.5em}
\For{$i \gets 0 \textbf{ à } nb\_AQp -1 $}{
    $\mathrm{msb\_ind}[i] \gets \textsc{MSB} (\mathrm{exp\_vects}[i])$  \; 
}
\vspace{0.5em}

\For{$ j \gets \textsc{MAX} (\mathrm{msb\_ind}) \textbf{ à } 1 $}{
    $pivot \gets \begin{cases}
		\min \big\{ i \in [\![ 0, nb\_AQp - 1 ]\!] \big\vert \mathrm{msb\_ind}[i]
		= j \big\}\\
		\varnothing \text{ si pour tout }i \in [\![ 0, nb\_AQp - 1 ]\!],
		\mathrm{msb\_ind}[i] \neq j  
   \end{cases}$\;
    \If{$pivot \neq \varnothing$}{
        \For{$i \gets pivot + 1 \textbf{ à } nb\_AQp -1 $}{
            \If{$\mathrm{msb\_ind}[i] = j$}{
                $\mathrm{exp\_vects}[i]  \gets \mathrm{exp\_vects}[i] \oplus
				\mathrm{exp\_vects}[pivot]$ \; 
                $\mathrm{hist\_vects}[i] \gets \mathrm{hist\_vects}[i] \oplus
				\mathrm{hist\_vects}[pivot]$\; 
                $\mathrm{msb\_ind}[i] \gets  \textsc{MSB}(\mathrm{exp\_vects}
				[i])$\; 
                \If{$\mathrm{exp\_vects}[i] = 0 $}{
                    ajouter $i$ au tableau $\mathrm{lin\_rel\_ind}$\; 
                    $nb\_lin\_rel \gets nb\_lin\_rel + 1 $\; 
                }
            }
        }
    }
}
\Return{$\mathrm{hist\_vects}[0 \cdots nb\_AQp -1]$, $\mathrm{lin\_rel\_ind}
    [0 \cdots nb\_lin\_rel-1]$, $nb\_lin\_rel$}
\end{algorithm}
\vspace{1em}

\subsubsection{La fonction \texttt{calculate\_A\_Q}}

Une fois les indices des vecteurs historiques indiquant un ensemble valide de
paires $(A, Q)$ récupérés, la fonction \texttt{find\_factor} appelle la
fonction \texttt{calculate\_A\_Q} pour calculer des entiers $A$ et $Q$
vérifiant $A^2 \equiv Q^2 \pmod{N}$. Elle lui donne en argument un de ces
vecteurs historiques et les données des tableaux \texttt{Ans} et \texttt{Qns}.
\\

Notons $l$ l'entier \texttt{nb\_AQp} et $(e_{l-1}, \cdots , e_0) \in
\mathbb{F}_2^{l}$ le vecteur historique donné en argument de la fonction. Le
calcul de \[A:= \prod_{n=0}^{l-1} Ans[i]^{e_i} \pmod{N} \] ne pose pas de
difficulté. Pour le calcul de \[Q:= \sqrt{\prod_{n=0}^{l-1 } Qns[i] ^{e_i}}
\pmod{N},\] on utilise l'algorithme proposé par Morrison et Brillhart.

\vspace{1em}
\begin{algorithm}[H]
\DontPrintSemicolon
\caption{\sc Extraction de racine carrée}
\KwIn{Des entiers $Q_1,\cdots, Q_r \in \Z$ tels que $\prod_{i=1}^r Q_{i}$ est
	un carré}
\KwOut{$\sqrt{\prod_{i=1}^r Q_{i}} \pmod{N}$}
$Q \gets 1$\;
$R \gets Q_1$\;
    \For{$i \gets 2$ $\textbf{à } r$}{
    $X \gets \pgcd(R, Q_i)$\; 
    $Q \gets XQ \pmod{N}$\;
    $R \gets \frac{R}{X} \cdot \frac{Q_i}{X}$\;
}
$X \gets \sqrt{R}$\;
    $Q \gets XQ \pmod{N}$\;
\Return{Q}\;
\end{algorithm}
\vspace{1em}

\begin{remarque}
    Pour démontrer la correction de l'algorithme, on peut utiliser l'invariant
    de boucle \[Q\sqrt{R.Q_i\cdots Q_r} \equiv \sqrt{\prod_{i=1}^r Q_{i}}
    \pmod{N},\] dont la conservation découle de l'égalité \[Q\sqrt{R.Q_i\cdots
    Q_r} \equiv QX \sqrt{\frac{R}{X}\frac{Q_i}{X}Q_{i+1} \cdots Q_r}
    \pmod{N}.\]
\end{remarque}

\subsection{Collecte des paires $(A,Q)$ : \texttt{step\_1.c et lp\_var.c}}

Décrivons à présent la phase de collecte des données. Concernant les vecteurs
historiques, il suffit d'initialiser à la fin de la 
collecte \texttt{hist\_vects[i]} pour \texttt{0 <= i < nb\_AQp}. C'est ce que fait 
la fonction \texttt{init\_hist\_vects}. La collecte des autres données 
requiert un peu plus d'explications. 

\subsubsection{La fonction \texttt{create\_AQ\_pairs}}

Sachant que seules les paires $(A,Q)$ dont on a pu factoriser $Q_n$ nous
intéressent pour la seconde phase, nous avons décidé de ne stocker que celles-ci.
Ce choix a en outre un avantage : étant donné un nombre \texttt{nb\_want\_AQp}
représentant le nombre voulu de telles paires, il est possible d'arrêter le 
développement en fraction continue dès que ce nombre est atteint. Cela évite
d'avoir à stocker toutes les paires $(A,Q)$, pour ensuite sélectionner celles 
qui nous intéressent, en courant le risque d'en avoir trop ou pas assez. \\

Ce choix amène à avoir une grande fonction, en l'occurence 
\texttt{create\_AQ\_pairs}, qui au fur à mesure du développement de $\sqrt{kN}$
en fraction continue, teste si le $Q_n$ qui vient d'être calculé est factorisable.
Si c'est le cas, on crée son vecteur exposant et ajoute les données de la paire
aux tableaux \texttt{Ans}, \texttt{Qns} et \texttt{exp\_vects}. Pour ce faire, la
fonction utilise les sous-routines \texttt{is\_Qn\_factorisable} et 
\texttt{init\_exp\_vect}.

\subsubsection{La \textit{early abort strategy}}

La fonction \texttt{is\_Qn\_factorisable} teste si un $Q_n$ est friable 
\footnote {ou presque friable, voir paragraphe suivant.} par divisions successives
avec les premiers de la base de factorisation.  Un moyen d'améliorer les 
performances de la méthode est de décider de ne pas poursuivre les divisions 
successives si après un nombre \texttt{eas\_cut} de divisions la partie non
factorisée de $Q_n$ est trop grande (supérieure à une borne 
\texttt{eas\_bound\_div} proportionnelle à la borne déjà connue $\sqrt{kN}$). 

\subsubsection{La \textit{large prime variation }}

Etant donnée une base de factorisation $B = \{ p_1, \cdots, p_m\}$, la \textit{large
prime variation } consiste à accepter lors de la collecte, non seulement des 
$Q_n$ $B$-friables mais aussi des $Q_n$ produits d'un entier $B$-friable et d'un
entier $lp_n$ inférieur à $p_m^2$. On dira que $Q_n$ est \emph{presque friable}
et l'on appelera \emph{grand premier (large prime)} le premier $lp_n$ en question. \\

Pour que des $Q_n$ presque friables soient exploitables, il faut qu'ils aient
un grand premier $lp$ en commun. En effet, si on trouve deux entiers presque
friables $Q_{n_1} = X_{n_1}lp $ et $Q_{n_2} =  X_{n_2}lp $, on peut former une
nouvelle paire $(A,Q)$ avec laquelle on peut travailler pour chercher une 
congruence de carrés. \\ 

Remarquons pour cela qu'on a les conguences :
\begin{equation*}
  \left\{
      \begin{aligned}
          A_{n_1 -1}^2 &\equiv (-1)^{n_1} X_{n_1}lp\pmod{N} \\
          A_{n_2 -1}^2 &\equiv (-1)^{n_2} X_{n_2}lp\pmod{N}\\
        \end{aligned}
    \right.
\end{equation*}

En les multipliant, on obtient : 
\[  (A_{n_1 -1} A_{n_2 -1})^2 \equiv 
     \underbrace{(-1)^{n_1 + n_2} X_{n_1} X_{n_2}}_
            {\begin{subarray}{c}\text{associé au vecteur exposant}\\
             v_{B} \big( (-1 )^{n_1} X_{n_1} \big)
             + v_{B} \big( (-1 )^{n_2} X_{n_2} \big) \end{subarray}
             }
    \underbrace{lp^2}_{\text{carré qui ne pose pas problème}}
    \pmod{N}
 \]
  
On forme donc la nouvelle paire $ (A_{n_1-1}A_{n_2 -1} \pmod{N}, Q_{n_1}Q_{n_2}) $
associée au vecteur exposant $v_{B} \big( (-1 )^{n_1} X_{n_1} \big)+
v_{B} \big( (-1 )^{n_2} X_{n_2} \big) $.  Elle sera traitée lors de la 
deuxième phase exactement de la même manière que les paires \og classiques \fg{}.\\

En pratique, pour repérer les paires qui ont le même grand premier, nous
constituons au fur et à mesure de la collecte une liste chainée dont les noeuds
stockent les données d'une paire dont le $Q_n$ est presque friable (les entiers
$Q_n$, $A_{n-1}$, le vecteur exposant et le grand premier associé à $Q_n$). Nous
maintenons cette liste triée par taille des grands premiers. Lorsque que survient
un $Q_n$ presque friable, il est repéré par la fonction
\texttt{is\_Qn\_factorisable} qui fournit également son grand premier $lp$. La 
liste chainée est alors parcourue pour savoir si l'on a déjà rencontré ce $lp$.
Deux cas se présentent alors. Si $lp$ est absent de la liste, on crée à la bonne
place un noeud. Si $lp$ est déjà présent dans la liste, au lieu de rajouter un
noeud, on utilise le noeud possédant ce $lp$ pour obtenir une nouvelle paire
$(A,Q)$ selon la méthode énoncée plus haut et ajoute ses composantes aux tableaux
\texttt{Ans}, \texttt{Qns} et \texttt{exp\_vects}. La fonction
\texttt{insert\_or\_elim\_lp} se charge de cela. \\



\newpage

\section{Résultats expérimentaux}

Les fonctions dont nous nous sommes servi pour tester le programme se trouvent
dans le fichier \texttt{test.c}. Nous avons effectué nos tests sur des entiers
générés par la fonction \texttt{rand\_N}. Ces entiers sont de la forme $N = pq$
avec $p$ et $q$ des premiers aléatoires et de taille semblable. 

\subsection{Choix de la taille de la base de factorisation}

Nous avons cherché à déterminer expérimentalement, selon la taille de l'entier
$N$ à factoriser, une bonne taille pour la base de factorisation lorsque la \textit{large
prime variation} et la \textit{early abort strategy} sont utilisées. Pour une
taille d'entier fixée, nous avons inscrit dans un fichier le temps que met le
programme pour plusieurs tailles de bases de factorisation (voir les fichiers
*.txt). Nous donnons ci-dessous des tailles appropriées (il ne s'agit que d'un
ordre de grandeur, la taille optimale varie d'un entier à l'autre). Ce sont
elles qui sont renvoyées par la fonction \texttt{choose\_s\_fb}.

\begin{center}
    \begin{tabular}{ |l |l | }
        \hline
         nombre de bits de $N$ & taille de la base de factorisation \\ \hline
         70                    & truc                               \\ \hline
         80                    & truc                               \\ \hline
         90                    & truc                               \\ \hline 
         100                   & truc                               \\ \hline 
         110                   & ttuc                               \\ \hline 
         120                   & truc                               \\ \hline 
         130                   & truc                               \\ \hline 
         140                   & truc                               \\ \hline 
         150                   & truc                               \\ \hline
    \end{tabular}
\end{center}

\subsection{Factorisation de $F_7$}

Notre programme permet de factoriser $F_7 = 2^{128} + 1$. La fonction 
\texttt{fact\_F7} prend en entrée des booléens indiquant si la
\textit{large\_prime variation (lp)} et la \textit{early abort strategy (eas)}
doivent être utilisées et lance cette factorisation. Avec le paramètre \texttt{k}
retourné par la fonction \texttt{choose\_k} - soit $k=38$ - et les paramètres 
\texttt{eas\_cut} et \texttt{eas\_coeff} valant respectivement $50$ et $1000000$,
nous obtenons les résultats suivants :  

\begin{center}
     \begin{tabular}{| l | l| l | l | l | l |}
     \hline
         variantes                    & \texttt{s\_fb} & \texttt{nb\_want\_AQp} &\texttt{nb\_AQp}& \texttt{last\_n}  & temps de calcul  \\ \hline
     \textit{lp} et \textit{eas}      &                &                        &                &                   &                  \\ \hline
     \textit{lp}                      &                &                        &                &                   &                  \\ \hline
     \textit{eas}                     &                &                        &                &                   &                  \\ \hline
     sans variante                    &                &                        &                &                   &                  \\ \hline
    \end{tabular}
\end{center}

\quad \footnotesize{\texttt{last\_n} est l'indice de la dernière paire $(A,Q)$ calculée.}

\subsection{Temps de calcul (avec les deux variantes) }

Ce graphique représente le temps moyen mis par le programme ( sur ... tests)
en fonction du nombre de bits de l'entier que l'on souhaite factoriser. La
\textit{large prime variation} et la \textit{early abort strategy} ont été 
utilisées et les paramètres ont été choisis par les fonctions 
\texttt{choose\_s\_fb} et \texttt{choose\_k}.\\


comparer quand k choisi par la fonction et k = 1: prévu de calculer le rapport des 
deux temps 


\addcontentsline{toc}{section}{Bibliographie}
\printbibliography

\end{document}
