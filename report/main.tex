% main.tex

\documentclass[a4paper, 12pt, oneside]{article}
\usepackage[french]{babel}
\usepackage{mathtools}
\usepackage[backend=biber, style=alphabetic]{biblatex}
\usepackage{pkg}
\usepackage[linesnumbered, boxruled, vlined, onelanguage, french]{algorithm2e}
% modifier Output en Sorties dans algorithm2e
\SetKwInput{KwOut}{Sorties}

\usepackage{array}
\newcolumntype{M}[1]{>{\raggedright}m{#1}}

\addbibresource{bi.bib}

\DeclarePairedDelimiter\ent{\lfloor}{\rfloor}
\newcommand{\en}[1]{(en: \textit{#1})}

\title{Factorisation par fractions continues}
\date{Février 2021}
\author{Margot Funk, Antoine Hugounet}

\begin{document}
\maketitle

\tableofcontents

\section*{Introduction}
\addcontentsline{toc}{section}{Introduction}
% introduction.tex

La méthode de factorisation par fractions continues est une variante de la
méthode de Fermat-Kraitchik dont les idées (recherche de congruences de carrés
$\pmod{N}$ — $N$ étant le nombre à factoriser — et recherche de relations
linéaires sur $\mathbb{F}_2$) demeurent centrales dans les algorithmes de
factorisation actuels. Cette méthode fut suggérée en 1931 par D. H. Lehmer et
R. E. Powers et permet de manier des quantités plus petites que dans la méthode
de Kraitchik originelle. C'est aussi l'une des premières méthodes de
factorisation dont la complexité est sous exponentielle. Bien que la méthode
de crible quadratique lui soit aujourd'hui préférée, la méthode des fractions
continues obtint des succès d'importance, notamment la première factorisation
du nombre de Fermat $F_7 = 2^{2^7} + 1$ le 13 septembre 1970. Cet exploit est
dû à M. A. Morrison et J. Brillhart qui implémentèrent la méthode pour la
première fois. Dans ce projet, nous implémentons à notre tour la méthode des
fractions continues, en langage C, et dans une version très proche de celle de
Morrison et Brillhart. La première section de ce texte est donc consacrée à la
théorie mathématique sous-jacente ; la deuxième présente l'implémentation en
détails ; la troisième expose les résultats numériques obtenus.


\section{Théorie}
\subsection{Fractions continues}
\subsubsection{Intuition}

Intuitivement, une fraction continue est une expression — finie ou infinie — de
la forme suivante\footnote{Notez que nous ne nous autorisons que des $1$ aux
numérateurs.} : \[a_0 + \cfrac{1}{a_1 + \cfrac{1}{a_2 + \cfrac{1}{a_3 +
\dots}}}\] telle que $a_0\in \Z$ et $a_i \in \N^*$ pour tout $i\in \N^*$.
Toujours intuitivement, nous voulons affûbler cette fraction continue d'une
valeur. Si la fraction continue est finie, c'est une bonne vieille fraction,
c'est à dire un élément du corps $\Q$~; si la fraction continue est infinie, on
calcule d'abord $a_0$, puis $a_0 + \frac{1}{a_1}$, puis $a_0 + \frac{1}{a_1
+\frac{1}{a_2}}$, et on continue une infinité de fois. La limite de la suite
générée est la \og{} valeur \fg{} de la fraction continue. Nous ferons sens
plus précis de l'intuition dans la prochaine sous-section. \\

Les fractions continues émanent de la volonté d'approcher des réels irrationels
par des fractions d'entiers. Par exemple, la fraction $\frac{103 993}{33 102}$
approche $\pi$ avec une précision meilleure que le milliardième.  Comment
générer une telle fraction continue pour un réel irrationnel $x$ ?  On part de
l'identité $x = x + \ent{x} - \ent{x}$ et l'on écrit \[x = \ent{x} +
\cfrac{1}{\cfrac{1}{x - \ent{x}}}.\] On pose $x_0 = x$ et $x_1 = \frac{1}{x_0 -
\ent{x_0}}$ (qui est bien défini par irrationalité de $x$) et l'on répète la
première étape sur $x_1$ : \[x = \ent{x_0} + \cfrac{1}{x_1} = \ent{x_0} +
\cfrac{1}{\ent{x_1} + \cfrac{1}{\cfrac{1}{x_1 - \ent{x_1}}}}.\] Comme le réel
$x$ est irrationnel, on peut répéter ce procédé indéfiniment. Nous construisons
alors la suite d'éléments \emph{irrationnels} de terme général \[x_n =
\frac{1}{x_{n-1} - \ent{x_{n-1}}}, \quad \forall n \geq 1.\] On associe alors à
\emph{l'irrationnel} $x$ la fraction continue \emph{infinie}\footnote{Lorsque
nous aurons correctement défini la notion de fraction continue, cette fraction
continue canoniquement associée à $x$ sera notée $\hat{x}$.} \[\hat{x}_0 +
\cfrac{1}{\hat{x}_1 + \cfrac{1}{\hat{x}_2 + \cfrac{1}{\hat{x}_3 + \dots}}},\]
où l'on a posé \[\hat{x}_i = \ent{x_i}\] pour tout $i\in \N$. Fixons ces
notations :

\begin{notation}\label{notations}
	Soit $x\in \R$ un élément irrationel. Notons $x_0 = x$ puis \[x_n :=
	\frac{1}{x_{n-1} - \ent{x_n}}, \quad \forall n \geq 1.\] Par ailleurs,
	notons \[\hat{x}_n := \ent{x_n}, \quad \forall n\in \N.\]
\end{notation}

\begin{remarque}
	La méthode de construction d'une fraction continue \emph{finie} pour un
	rationnel est la même : il faut simplement s'arrêter lorsque l'on tombe sur
	un $\hat{x}_n$ vérifiant $\hat{x}_n = \ent{\hat{x}_n}$. Cet algorithme
	termine (\NTS{ref}) et s'exécute plus simplement en utilisant l'algorithme
	d'Euclide.
\end{remarque}

\subsubsection{Formalisation}

Formellement, on peut définir\footnote{La définition mathématique est
descriptive et non prescriptive.} une fraction continue ainsi :

\begin{definition}[Fraction continue]\label{def-fracont}
	On appelle \emph{fraction continue} toute suite non vide (finie ou infinie)
	$(a_i)_{i\in U} \in \N^\N$, $U\subset \N$, d'entiers qui vérifie \[a_i
	\geq 1, \quad \forall i\in U\setminus\{0\}.\] Cette suite est alors notée
	\[a_0 + \cfrac{1}{a_1 + \cfrac{1}{a_2 + \cfrac{1}{a_3 + \dots}}}.\]
\end{definition}

\begin{notation}
	Soit $x\in \R$ un élément irrationel. On note $\hat{x}$ la fraction
	continue infinie canoniquement associée à $x$ par la méthode exposée dans
	le premier paragraphe. Autrement dit, $\hat{x}$ est la fraction continue
	donnée par la suite infinie (voir \ref{notations}) $(\hat{x}_i)_{i\in
	\N}$.
\end{notation}

Il est naturel d'associer à une fraction continue (finie ou infinie) une suite
(finie ou infinie) de fractions \og intermédiaires \fg{} appelées
\emph{réduites}. Pour n'avoir aucun problème de division par zéro, nous nous
plaçons temporairement dans un corps de fractions rationnelles en $\N$
indeterminées.

\begin{definition}[Réduites formelles]
	Soit $(X_i)_{i\in \N}$ une suite (infinie) d'indeterminées sur le corps
	$\Q$. On définit \[[X_0] = X_0\] puis par récurrence \[[X_0, \dots,
	X_n] = X_0 + \frac{1}{[X_1, \dots, X_n]}.\] Ces éléments sont dans
	$\Q((X_i)_{i\in \N})$.
\end{definition}

\begin{definition}[Réduites d'une fraction continue]
	Distinguons les cas finis et infinis. Soit $f$ une fraction
	continue.
	\begin{itemize}
		\item Si $f$ est donnée par la suite finie $(a_0, \dots, a_n)$, pour
		tout $k\in [\![0, n]\!]$ on appelle \emph{$k$-ième réduite de $f$}
		l'élément $[a_0, \dots, a_k]$.
		\item Si $f$ est donnée par la suite infinie $(a_i)_{i\in \N}$, pour
		tout $k\in \N$ on appelle \emph{$k$-ième réduite de $f$} l'élément
		$[a_0, \dots, a_k]$.
	\end{itemize}
\end{definition}

\begin{exemple}
	Soit $f$ la fraction continue infinie donnée par la suite $(1)_{i\in \N}$.
	La première réduite est $[1] = 1$, la deuxième est \[[1, 1] = 1 +
	\frac{1}{[1]} = 1 + \frac{1}{1},\] la troisième est \[[1, 1, 1] = 1 +
	\frac{1}{[1, 1]} = 1 + \frac{1}{1 + \cfrac{1}{1}}.\] Plus généralement, la
	$k$-ième réduite de $f$ est de la forme \[[1, 1, \dots, 1] = 1 +
	\cfrac{1}{1 + \cfrac{1}{\dots \\ \cfrac{1}{1 +
	\cfrac{1}{1}}}}.\]
\end{exemple}

Remarquons que les réduites de toute fraction continue sont des éléments
rationnels, ce même si la fraction continue est égale à $\hat{x}$ pour un
certain irrationel $x$. De fait, $x$ n'est égal à aucune des réduites de
$\hat{x}$. Mais en reprenant les notations \ref{notations}, on a toutefois
\NTS{pourquoi ça fait ça ? avec le 1.1.3 ?}
\begin{equation} \ref{egalite-reduite}
	x = [\hat{x}_1, \dots, \hat{x}_{n-1}, x_n], \quad \forall x\in \N.
\end{equation}
Cette égalité sera cruciale dans notre algorithme de factorisation. \\

Même si les fractions continues finie restent des suites (déf.
\ref{def-fracont}), leur représentation graphique permet de les voir
trivialement comme des éléments du corps $\Q$. En effet, en représentant \[a_0
+ \cfrac{1}{a_1 + \cfrac{1}{a_2 + \dots}}\] la fraction continue finie $f$
associée à la suite finie $(a_0, \dots, a_n)$, on peut la voir comme l'élément
rationnel \[a_0 + \cfrac{1}{a_1 + \cfrac{1}{a_2 + \dots}}.\] Cet élément n'est
autre que sa dernière réduite $[a_0, \dots, a_n]$ et on dit que $f$ est égale à
l'élément rationnel $[a_0, \dots, a_n]$. Pour les fractions continues infinies,
ce n'est pas aussi simple.

\begin{definition}
	Soient $l$ un réel et $f$ une fraction continue donnée par la suite infinie
	$(a_i)_{i\in \N}$. On dit que \emph{$f$ est égale à $l$}, que \emph{$f$
	converge vers $l$}, ou encore que \emph{$f$ est le développement en
	fraction continue de $l$} et l'on note $f = l$ si la suite des réduites de
	$f$ converge vers $l$. Si une fraction continue infinie est égale à un
	certain réel, on dit qu'elle converge.
\end{definition}

\begin{exemple}[Nombre d'or]
	On appelle \emph{nombre d'or} et l'on note $\varphi$ l'unique racine réelle
	positive du polynôme $X^2 - X - 1 \in \Z[X]$. On a $\varphi =\frac{1 +
	\sqrt{5}}{2} \simeq 1, 618$. Comme $\varphi^2 = \varphi + 1$ et que
	$\varphi \neq 0$, on a $\varphi = 1 + \frac{1}{\varphi} = 1 + \cfrac{1}{1 +
	\cfrac{1}{\varphi}}$. En réalité, $\varphi$ est égal à une fraction
	continue : \[\varphi = 1 + \cfrac{1}{1 + \cfrac{1}{1 + \cfrac{1}{1 +
	\dots}}}.\]
\end{exemple}

Dans quelle mesure une fraction continue converge-t-elle ? Des raisonnements
d'analyse élémentaire (\NTS{réf}) permettent de montrer que toute fraction
continue infinie converge, et qu'elle converge vers un irrationnel !

\begin{theoreme}
	La fonction canonique \[x \mapsto \cfrac{1}{\hat{x}_0 +
	\cfrac{1}{\hat{x}_1 + \cfrac{1}{\hat{x}_2 + \dots}}}\] est une
	\emph{bijection} entre l'ensemble des nombres réels irrationnels et des
	fractions continues infinies.
\end{theoreme}

En particulier, un réel $x$ et une fraction continue $f$ sont égaux \ssi $f$
est la fraction continue donnée par la suite $(\hat{x}_{i\in \N})$. Attention,
les réels tout entier ne sont pas en bijection avec les fractions continues
(finies ou infinies). En effet, un rationnel est égal (au sens donné dans les
paragraphes précédents) à exactement deux fractions continues : si un rationnel
est égal à $[a_0, \dots, a_n]$, il est aussi égal à $[a_0, \dots, a_n - 1, 1]$
et n'est égal à aucune autre fraction continue (\NTS{ref}).

\subsubsection{Irrationels quadratiques}

L'adaptation de l'algorithme de Fermat-Kraitchik avec les fractions continues
utilise crucialement le développement en fraction continue de $\sqrt{kN}$, où
$N$ est le nombre à factoriser et $k\in \N*$ un entier arbitraire. Intéréssons
nous aux fractions continues de ces nombres.

\begin{definition}[Irrationel quadratique]
	On appelle \emph{irrationel quadratique} tout nombre réel, algébrique sur
	$\Q$, de degré $2$. Un irrationel quadratique est dit \emph{réduit} si son
	conjugué est dans l'intervalle $]\!]-1, 0[\![$.
\end{definition}

Les fractions continues d'irrationnels quaratiques sont sujettes à des
phénomènes de périodicité.

\begin{definition}
	Soit $f$ la fraction continue donnée par une suite $(a_i)_{i\in \N}$. On
	dit que $f$ est \emph{périodique} si la suite l'est à partir d'un certain
	rang.  Autrement dit, il existe un rang $n_0\in \N$ et une période $p\in
	\N^*$ tels que \[a_{i} = a_{i+p}, \quad \forall i\geq n_0.\] On note alors
	\[f = [a_0, \dots, a_{n_0 - 1}, \overline{a_{n_0}, \dots, a_{n_0 + p
	-1}}].\] On dit enfin que $f$ est \emph{purement périodique} si $n_0 = 0$.
\end{definition}

\begin{exemple}
	La fraction continue du nombre d'or est purement périodique de période $1$.
	La fraction continue de l'irrationel $\sqrt{14}$ vaut \[\sqrt{14} = [4,
	\overline{2, 1, 3, 1, 2, 8}].\]
\end{exemple}

Nous disposons des deux résultats fondamentaux suivants.

\begin{theoreme}[Lagrange, 1770]
	Un réel irrationel est un irrationel quadratique \ssi son développement en
	fraction continue est périodique.
\end{theoreme}

\begin{theoreme}[Galois, 1829]
	Un irrationnel quadratique est réduit \ssi son développement en fraction
	continue est purement périodique.
\end{theoreme}

Un troisième résultat donne encore plus d'informations dans le cas où
l'irrationel quadratique considéré est une racine carrée.

\begin{theoreme}[Legendre, 1798]
	Un réel irrationel est la racine carrée d'un entier $>1$ \ssi son
	développement en fraction continue est de la forme \[[a_0, \overline{a_1,
	a_2, \dots, a_2, a_1, 2a_0}].\]
\end{theoreme}

Ces phénomènes de périodicité devront être pris en compte dans les paramètres
d'entrée de l'algorithme de factorisation, voir \NTS{réf}. En plus de la suite
des réduites, nous aurons besoin d'une autre suite importante.

\begin{lemme}
	Soit $x$ un irrationel quadratique. Alors l'élément $\frac{1}{x - \ent{x}}$
	est lui aussi un irrationel quadratique. \NTS{réf}
\end{lemme}

Fixons $N$ l'entier à factoriser et $k\in \N*$ tel que $\sqrt{kN}$ est un
irrationel quadratique. Posons alors $x := \sqrt{kN}$. D'après l'identité
\label{egalite-reduite} et en reprenant les notations \label{notations}, nous
pouvons donner un développement partiel (jusqu'à un rang donné $n\in
\N$) de $x$ en fraction continue \[x = [\hat{x}_1, \dots, \hat{x}_{n-1},
x_n].\] D'après le lemme précédent, $x_n$ est lui aussi un irrationel
quadratique, i.e. il est solution d'une équation quadratique. On peut
(\NTS{réf, dém. de 2.5.8}) de fait l'écrire de manière unique 

\begin{equation}
	x_n = \frac{P_n + x}{Q_n}, \quad P_n, Q_n \in \Z.
\end{equation}

Ces notations $P_n$ et $Q_n$ seront réutilisées plus tard. La $n$-ième réduite
de $x$ étant un nombre rationnel, on peut l'écrire sous la forme
$\frac{A_n}{B_n}$, où $A_n, B_n$ sont entiers et la fraction est irréductible
bien définie. Les entiers $A_n$, $B_n$, $Q_n$ et l'irrationel $x$ se
rencontrent dans un grand nombre d'égalités numériques. Nous choisissons de
n'exposer que les plus utiles à notre propos\footnote{Pour plus de contenu et
de détails, la lectrice pourra se référer à \NTS{réf}.}. Pour tout $n\geqslant
1$, on a \[A_{n-1}^2 - kN B_{n-1}^2 = (-1)^n Q_n\] et donc
\begin{equation}
	A_{n-1}^2 \equiv (-1)^n Q_n.
\end{equation}
On a également
\begin{equation}
	\begin{cases}
		P_n < \sqrt{kN} \\
		Q_n < 2\sqrt{kN}.
	\end{cases}
\end{equation}

\subsection{Factorisation par fractions continues}

Dans tout le reste de cette section, $N$ désigne un entier naturel composé impair.

\subsubsection{Méthodes de Fermat et Kraitchik}

La méthode de factorisation de Fermat part du constat suivant.

\begin{lemme}
	Factoriser $N$ est équivalent à l'exprimer comme différence de deux carrés
	d'entiers.
\end{lemme}

\begin{proof}
	Si $N = u^2 - v^2, u, v\in \Z$ alors $N = (u-v)(u + v)$. Réciproquement si
	l'on a une factorisation $N = ab$, alors $N = \left(\frac{a+b}{2}\right)^2
	- \left(\frac{a-b}{2}\right)^2$.
\end{proof}

La méthode de Fermat exploite cette propriété et se montre particulièrement
efficace lorsque $N$ est le produit de deux entiers proches l'un de l'autre.
Notons $N=ab$ une factorisation de $N$ avec $a$ et $b$ proches, $r=\frac{a+b}{2}$,
$s=\frac{a-b}{2}$ de sorte que \[N = r^2 - s^2.\] Comme $s$ est petit en valeur
absolue par hypothèse, l'entier $r$ est donc plus grand que $\sqrt{N}$ tout en
lui étant proche. Il existe donc un entier positif $u$ \emph{pas trop grand} tel
que \[\ent{\sqrt{N}} + u = r\] et donc tel que $(\ent{\sqrt{N}} + u)^2 - N$ soit
un carré. Trouver un tel entier $u$ donne alors la factorisation de $N$. Comme 
les facteurs de $N$ sont proches l'un de l'autre, on le trouve par essais 
successifs. \\

La méthode de Fermat est cependant inefficace lorsque les facteurs de $N$ ne
sont pas proches. D'après \cite{Tale} ¶ \emph{Fermat and Kraitchik}, la méthode
est alors encore plus coûteuse que la méthode des divisions successives. Dans
les années 1920, Maurice Kraitchik a amélioré l'efficacité de la méthode de
Fermat. Son idée essentielle est que pour factoriser $N$, il \emph{suffit} de
trouver une différence de deux carrés qui soit un multiple de $N$.

\begin{lemme}
	Connaître deux entiers $u, v \in \Z$ tels que $u^2 \equiv v^2 
	\pmod{N}$ et $u\not\equiv \pm v\pmod{N}$ fournit une factorisation de $N$.
	Plus spécifiquement, les entiers $\pgcd(u-v, N)$ et $\pgcd(u+v, N)$ sont
	des facteurs non triviaux de $N$.
\end{lemme}

\begin{proof}
	Posons $g =\pgcd(u-v, N)$ et $g' = \pgcd(u+v, N)$. Comme $u\not\equiv \pm
	v\pmod{N}$, on a $g<N$ et $g'<N$. Enfin ni $g$ et $g'$ ne sont réduits à
	$1$ : si l'un deux l'est, l'autre vaut $N$, contradiction. Donc $g$ et $g'$
	sont tous deux des facteurs non triviaux de $N$.
\end{proof}

\begin{remarque}
	Dans l'algorithme, nous nous contenterons de chercher des $u, v$ tels que
	$N$ divise la différence de leurs carrés, sans vérifier s'ils vérifient $u
	\not\equiv \pm v\pmod{N}$. Comme le polynôme $X^2 - v^2 \in \Z/N\Z$ a
    exactement quatre racines, il y a \og{} une chance sur deux \fg{}  pour que
	$u$ et $v$ nous fournissent un facteur non trivial de $N$.
\end{remarque}

La factorisation de $N$ se fait donc en trouvant de tels couples $(u, v)$.
Kraitchik cherche dans un premier temps des couples $(u_i, Q_i)_{1\leq i \leq
r}$ vérifiant \[u_i^2 \equiv Q_i \pmod{N}\] et tels que l'entier $\prod_{i=1}^r
Q_i$ soit un carré (dans $\Z$). Posant $u = \prod_{i=1}^r u_i$ et $v =
\sqrt{\prod_{i=1}^rQ_i}$, il vient \[u^2\equiv v^2 \pmod{N}.\] Obtenir des
congruences de la forme $u_i^2 \equiv Q_i \pmod{N}$ ne pose pas de difficulté.
Kraitchik propose d'utiliser le polynôme\footnote{Ce qui n'est pas sans défaut,
nous verrons plus tard que l'utilisation des fractions continues en lieu et
place de ce polynôme est plus efficace.} \[K:= X^2 - N \in \Z[X].\] Se donner
des éléments $u_1,\dots, u_s\in \Z$ et poser $Q_1 = K(u_1), \cdots, Q_s =
K(u_s)$ fournit bien des congruences de la forme souhaitée. Il reste à donner
un algorithme efficace permettant d'exhiber une sous-famille de $\{Q_1, \cdots,
Q_s\}$ dont le produit est un carré.

\subsubsection{Recherche de congruences de carrés}

Morrison et Brillhart proposent un tel algorithme dans \cite{MB}. Leur méthode
est basée sur la connaissance de congruences de la forme \[u_i^2 \equiv Q_i
\pmod{N},\] où $| Q_i |$ est suffisamment petit pour être factorisé. On fixe
pour cela $B = \{p_1, \dots, p_m\}$ une base de factorisation, c'est à dire un
ensemble non vide fini de nombres premiers, et l'on dit qu'un entier $Q\in
\N\setminus \{0, 1\}$ est \emph{$B$-friable} si tous ses facteurs premiers sont
dans $B$. Soit $Q\in \Z$ un entier dont la valeur absolue est $B$-friable. Il
s'écrit alors \[Q = (-1)^{v_0}\prod_{i=1}^m p_i^{v_{p_i}(Q)}.\] Puisque les
éléments de $B$ sont fixés et en nombre fini, l'élément $Q$ peut être vu comme
le vecteur des valuations $(v_{p_m}(Q), \dots, v_{p_1}(Q), v_0 ) \in
\mathbb{N}^{m+1}$.

\begin{definition}
	On appelle \emph{$B$-vecteur exposant de $Q$}, ou simplement \emph{vecteur
	exposant de $B$} si aucune confusion n'est possible, et l'on note $v_B(Q)$
	le vecteur \footnote{Notez qu'il s'agit d'un élément de
	$\mathbb{F}_2^{m+1}$ : seule la parité des valuations nous intéresse.
	L'élément $v_0$ est placé à droite et non au début car il correspondra au
	bit de poids faible du $B$-vecteur exposant dans le code.} \[v_B(Q) :=
	(v_{p_m}(Q), \dots, v_{p_1}(Q), v_0 ) \in \mathbb{F}_2^{m+1}.\] 
\end{definition}

Connaissant suffisamment d'entiers $Q_i$ de valeur absolue $B$-friable, on peut
en extraire une sous-famille dont le produit est un carré.

\begin{proposition}
	Soit $F$ une famille d'entiers dont les valeurs absolues sont $B$-friables.
	Si \[\# F \geqslant \# B + 2\] alors on peut extraire une sous-famille de
	$F$ dont le produit des éléments est un carré.
\end{proposition}

\begin{proof}
	Posons $F = \{Q_1, \dots, Q_k\}$ (de sorte que $k = \# F)$ et $B = (p_1,
	\dots, p_m)$ (de sorte que $\# B = m$). Par hypothèse de friabilité, on
	peut associer à chaque $Q_j$, $1\leqslant j \leqslant k$, un $B$-vecteur
	exposant. Fixons $j, j'\in [\![1,k]\!]$. L'entier $Q_j$ est un carré \ssi
	les composantes de son vecteur de valuations \[(v_{p_m}(Q), \dots,
	v_{p_1}(Q), v_0 ) \in \mathbb{N}^{m+1}\] sont paires, i.e. si son
	$B$-vecteur exposant est nul. Par propriété des valuations, le $B$-vecteur
	exposant associé au produit $Q_j\cdot Q_{j'}$ est le vecteur somme
	$v_B(Q_j) + v_B(Q_{j'})$. Autrement dit, le produit d'une sous-famille
	$\{Q_{j_1}, \dots, Q_{j_s}\}$ de $F$ est un carré \ssi la somme des
	$B$-vecteurs exposants $v_B(Q_{j_1}), \dots, v_B(Q_{j_s})$ est nulle. Soit
	$V$ le $\mathbb{F}_2$-espace vectoriel $\mathbb{F}_2^{m+1}$, qui est de
	dimension $m+1$. Comme $k\geqslant m+2$, la famille \[\{v_B(Q_1), \dots,
	v_B(Q_k)\}\] est liée dans $V$ et il existe de fait des éléments $l_1,\dots,
	l_k\in \mathbb{F}_2$ tels que \[\sum_{j=1}^k l_j v_B(Q_j) = 0.\] L'élément
	$\prod_{j=1}^k Q_j^{l_j}$ est alors un carré.
\end{proof}

\label{matrice-gauss}
Étant données des congruences de la forme $u_i^2 \equiv Q_i \pmod{N}$, la
preuve de la proposition fournit un procédé d'algèbre linéaire pour extraire
une sous-famille des $Q_i$ dont le produit des éléments est un carré. On trouve
tout d'abord des $Q_{i_1}, \dots, Q_{i_k}$ dont la valeur absolue est
$B$-friable\footnote{Nous le ferons en factorisant les $Q_i$ à disposition par
divisions successives.}. Soit $M$ la matrice \[M := \begin{pmatrix}
v_B(Q_{i_1}) \\ \vdots \\ v_B(Q_{i_k})\end{pmatrix}\in \mathcal{M}_{k,
\#B+1}(\mathbb{F}_2),\] soient $l_{1}, \dots, l_{k}$ les éléments de
$\mathbb{F}_2$ donnés dans la preuve de la proposition tels que \[\prod_{j=1}^k
Q_{i_j}^{l_{i_j}}\] est un carré. Le vecteur $(l_1, \dots, l_{k})$ est un
élément du noyau de la matrice transposée de $M$. Il peut donc être exhibé 
par pivot de Gau\ss{}.

\subsubsection{Utilisation des fractions continues}

L'introduction des fractions continues est motivée par le constat suivant. Si
\[u_i^2 = Q_i + kNb^2, \quad u_i, Q_i, b\in \Z, k\in \N^*\] de telle sorte que
$| Q_i |$ soit petit, alors \[\left(\frac{u_i}{b}\right)^2 - kN =
\frac{Q_i}{b^2}\] est petit en valeur absolue et $\frac{u_i}{b}$ est une bonne
approximation de $\sqrt{kN}$ (\cite{Cohen}, p. 478). Fixons $k$ un entier
naturel non nul, posons $x = \sqrt{kN}$ et reprenons les notations
\ref{notations} et celles développées à la fin de la sous-section
\ref{ss-irrquad}. En vertu de l'identité \[A_{n-1}^2 \equiv (-1)^n Q_n
\pmod{N},\] nous appelons \emph{méthode de factorisation des fractions
continues} la méthode de Kraitchik dans laquelle les entiers $u_i$ sont donnés
par les $A_{i-1}$ et les $Q_i$ par les $(-1)^i Q_i$. \\

L'intérêt de la méthode des fractions continues réside en la majoration $Q_n
\leq 2\sqrt{kN}$ (\ref{inegalite}). À l'inverse, les $x^2 - N, x\in \N$ de
Kraitchik ont une croissance linéaire de pente $2\sqrt{N}$ lorsque $x$
s'éloigne de $\sqrt{N}$. Pour une base de factorisation $B$ fixée, les $Q_n$
auront donc plus de chance d'être $B$-friables. Or, l'étape la plus coûteuse de
l'algorithme est celle de la recherche des termes $B$-friables par divisions
successives. Notons d'autre part qu'il est facile de générer le développement
en fraction continue de $x$ et les paires $(A_{n-1}, Q_n)$ grace à un
algorithme itératif dû à Gau\ss{} et exposé dans \cite{MB}, p. 185. Qui plus
est, on a le résultat suivant (\cite{MB}, p. 191).

\begin{proposition}
	Les diviseurs premiers $p$ de $Q_n$ vérifient nécessairement
	\[\left(\frac{kN}{p} \right) \in \{0, 1\}.\]
\end{proposition}

\begin{proof}
	Soit $p$ un diviseur premier de $Q_n$. Alors $ A_{n-1}^2 \equiv kNB_{n-1}^2
	\pmod{p}$. Comme $\pgcd(A_{n-1},B_{n-1}) = 1$, $p$ ne peut pas diviser
	$B_{n-1}$ (sinon $A_{n-1}^2 \equiv 0 \pmod{p}$ et $p$ diviserait aussi
	$A_{n-1}$). L'entier $B_{n-1}$ est donc inversible modulo $p$ et $\left(
	\frac{A_{n-1}}{B_{n-1}}\right )^2 \equiv kN \pmod{p}$.
\end{proof}

\subsubsection{Quelques éléments pour appréhender la complexité de la méthode}


\section{Explication du programme}
Dans cette §, nous décrivons notre implémentation de la méthode de
factorisation des fractions continues.
\subsection{Architecture du programme}

\subsubsection{Terminologie}

Énonçons pour commencer quelques définitions qui seront utiles pour décrire le
code. \NTS{réf vers 1.2.3 pour les notations}

\begin{definition}
	On nomme \emph{paire $(A, Q)$} tout couple $(A_{n-1}, Q_n), n\geq 1$.
\end{definition}

\begin{definition}
	Un ensemble de paires $(A, Q)$ indexé par $n_1, \dots, n_r$ est dit
	\emph{valide} si le produit $\prod_{i=1}^r (-1)^{n_i} Q_{n_i}$ est un carré
	(dans $\Z$).
\end{definition}

\begin{definition}
    Si $B$ est la base de factorisation utilisée par le programme, on 
    désignera par l'expression \emph{vecteur exposant associé à $Q_n$} 
    le $B$-vecteur exposant $v_B((-1)^n Q_n)$.
\end{definition}

\subsubsection{Structure générale}

Le programme comprend deux étapes principales. La première consiste à générer,
à partir du développement en fractions continues de $\sqrt{kN}$ \NTS{réf
section}, des paires $(A, Q)$ avec $Q_n$ friable pour une base de factorisation
préalablement déterminée et fixée. On associe à chaque $Q_n$ ainsi produit son
vecteur exposant \texttt{mpz\_t exp\_vect}. Ce vecteur permet de retenir les
nombres premiers qui interviennent dans la factorisation de $Q_n$ avec une
valuation impaire. Dans le but d'augmenter le nombre de paires $(A,Q)$
acceptées lors de cette étape, nous avons implémenté la \textit{large prime
variation}. Cette variante permet d'accepter une paire si son $Q_n$ se
factorise grâce aux premiers de la base de factorisation fixée et à un nombre
premier supplémentaire. Les fonctions de cette phase de collecte sont rassemblées dans
le fichier \texttt{step\_1.c}. Elles font appel, pour mettre en oeuvre la
\textit{large prime variation}, aux fonctions du fichier \texttt{lp\_var.c}. \\
 
Ces données sont traitées lors de la seconde phase dans l'espoir de trouver un
facteur non trivial de $N$. Il s'agit de trouver des ensembles valides de
paires $(A, Q)$ parmi les paires trouvées en première phase. Cela se fait par
pivot de Gau\ss{} sur la matrice dont les lignes sont formées des vecteurs
exposants \NTS{réf à avant}. Chaque ensemble de paires $(A, Q)$ valide trouvé
fournit une congruence de la forme $A^2 \equiv Q^2$ (mod $N$). Cette dernière
permet éventuellement de trouver un facteur non trivial de $N$. Les fonctions de 
cette phase sont regroupées dans le fichier \texttt{step\_2.c}. \\

Avant d'effectuer la première étape, il convient de se doter d'une base de
factorisation, ce qui est fait avec les fonctions de \texttt{init\_algo.c}.
Ces dernières se chargent plus généralement de l'initialisation et du choix par
défaut des paramètres, comme la taille de la base de factorisation ou l'entier
$k$. \\

Finalement, en mettant bout à bout les deux étapes, la fonction 
\texttt{contfract\_factor} du fichier \texttt{fact.c} recherche un facteur
non trivial de $N$ et \texttt{print\_results} affiche les résultats. 

\subsubsection{Entrées et sorties}

Nous avons regroupé dans une structure \texttt{Params} les paramètres d'entrée
de la fonction de factorisation, à savoir :

\begin{itemize}
	\item \texttt{N} : le nombre à factoriser, supposé produit de deux grands
	nombres premiers.
    \item \texttt{k} : le coefficient multiplicateur.
	\item \texttt{n\_lim} : le nombre maximal de paires $(A,Q)$ que l'on
	s'autorise à calculer. Ce nombre prend en compte toutes les paires
	produites, non uniquement les paires dont le $Q_n$ est friable ou résultant
	de la \textit{large prime variation}.
    \item \texttt{s\_fb} : la taille de la base de factorisation. 
	\item \texttt{nb\_want\_AQp} : le nombre désiré de paires $(A,Q)$ avec
	$Q_n$ friable ou résultant de la \textit{large prime variation}.
	\item des booléens indiquant si la \textit{early abort strategy} ou la
	\textit{large prime variation} doivent être utilisées et des paramètres s'y
	rapportant.
\end{itemize}

Le programme stocke dans une structure \texttt{Results} un facteur non trivial
de \texttt{N} trouvé (si tel est le cas) ainsi que des données permettant 
l'analyse des performances de la méthode.  

\begin{remarque}
	L'efficatité de la méthode dépend du choix des paramètres ci-dessus. Pour
	avoir plus de latitude dans les tests, nous les considérons comme des
	paramètres d'entrée du programme. C'est pourquoi notre programme ne
	s'attèle pas à la factorisation complète d'un entier, qui aurait nécessité
	une sous-routine déterminant des paramètres optimaux en fonction de la
	taille de l'entier dont on cherche un facteur. 
\end{remarque}

\begin{remarque}
	Notre programme n'est pas supposé prendre en entrée un nombre admettant un
	petit facteur premier (inférieur aux premiers de la base de factorisation
	par exemple).  En effet, comme il ne teste pas au préalable si \texttt{N}
	est divisible par de petits facteurs, il mettra autant de temps à trouver
	un petit facteur qu'un grand facteur.
\end{remarque}

\subsection{Pivot de Gauss et recherche d'un facteur non trivial :
\texttt{step\_2.c}}

Avant de nous pencher sur les détails de la phase de collecte, regardons 
l'implémentation de la seconde phase, qui aide à mieux comprendre la 
forme sous laquelle nous collectons les données.

\subsubsection{Utilisation des données collectées}

A l'issue de la première phase, on espère avoir collecté \texttt{nb\_want\_AQp}
paires $(A,Q)$ avec $Q_n$ friable\footnote{Les paires peuvent aussi résulter de
la \textit{large prime variation}, cela n'a aucune incidence sur les fonctions
de cette partie.}. Le nombre réel de telles paires est stocké dans le champ
\texttt{nb\_AQp} d'une structure \texttt{Results} (voir sous-§ précédente). Une
paire $(A,Q)$ collectée est caractérisée par :

\begin{itemize}
    \item la valeur $A_{n-1}$,
    \item la valeur $Q_n$,
    \item le vecteur exposant associé à $Q_n$,
    \item un vecteur historique (voir ci-contre).
\end{itemize}

Les données de ces \texttt{nb\_AQp} paires sont stockées dans quatre tableaux :
\texttt{mpz\_t *Ans}, \texttt{mpz\_t *Qns}, \texttt{mpz\_t *exp\_vects} et
\texttt{mpz\_t *hist\_vects}. À un indice correspond une paire $(A,Q)$ donnée.
Le vecteur historique sert à indexer les paires collectées pour former un
analogue de la matrice identité utilisée pendant le pivot de Gauss. Plus
précisément, \texttt{hist\_vects[i]} est, avant pivot de Gauss, le vecteur
$(e_{l-1}, \cdots, e_0)$ où $l=\texttt{nb\_AQp}$ et $e_j = \delta_{ij}$. À
partir de ces quatre tableaux, la fonction \texttt{find\_factor} cherche un
facteur de \texttt{N} selon la méthode des congruences de carrés. Elle utilise
pour cela les fonctions auxiliaires \texttt{gauss\_elimination} et
\texttt{calculate\_A\_Q}. 

\subsubsection{La fonction \texttt{gauss\_elimination}}

La fonction \texttt{gauss\_elimination} effectue un pivot de Gauss sur les
éléments de \texttt{mpz\_t *exp\_vects}, vus comme les vecteurs-lignes d'une
matrice. Comme pour un pivot de Gauss classique, les calculs effectués sur les
vecteurs exposants sont reproduits en parallèle sur la matrice identité,
c'est-à-dire sur les éléments de \texttt{mpz\_t *hist\_vects}. Si le
\textit{xor} de deux vecteurs exposants donne le vecteur nul, cela signifie
qu'une relation de dépendance a été trouvée. On inscrit alors dans un tableau
l'indice de ce vecteur nul. Le vecteur historique dudit indice indique les
paires $(A,Q)$ de l'ensemble valide trouvé. La procédure que nous avons 
implémentée est décrite ci-dessous. 

\vspace{1em}
\begin{algorithm}[H]
\DontPrintSemicolon
\caption{\sc{Pivot de Gauss}}
\KwIn{tableau $\mathrm{exp\_vects}[0 \cdots nb\_AQp -1 ] $ des vecteurs
	exposants, tableau $\mathrm{hist\_vects}[0 \cdots nb\_AQp -1 ]$ des
	vecteurs historiques}
\vspace{0.5em}
\KwOut{$\mathrm{hist\_vects}[0 \cdots nb\_AQp -1]$ après le pivot, le nombre
	$nb\_lin\_rel$ de relations linéaires trouvées, $\mathrm{lin\_rel\_ind}[0
	\cdots nb\_lin\_rel-1]$ contenant les indices des lignes où une relation
	linéaire a été trouvée}
\vspace{0.5em}
créer tableau $\mathrm{msb\_ind}[0 \cdots nb\_AQp - 1]$\;  
créer tableau $\mathrm{lin\_rel\_ind}$\; 
$nb\_lin\_rel \gets 0$\; 
\vspace{0.5em}
\tcc{Initialisation du tableau \textsc{msb\_ind} : \textsc{Msb}(x) renvoie $0$
	si x est nul, l'indice du bit de poids fort de x sinon. Les indices des
	bits sont  numérotés de 1 à l'indice du bit de poids fort.}
\vspace{0.5em}
\For{$i \gets 0 \textbf{ à } nb\_AQp -1 $}{
    $\mathrm{msb\_ind}[i] \gets \textsc{MSB} (\mathrm{exp\_vects}[i])$  \; 
}
\vspace{0.5em}

\For{$ j \gets \textsc{MAX} (\mathrm{msb\_ind}) \textbf{ à } 1 $}{
    $pivot \gets \begin{cases}
		\min \big\{ i \in [\![ 0, nb\_AQp - 1 ]\!] \big\vert \mathrm{msb\_ind}[i]
		= j \big\}\\
		\varnothing \text{ si pour tout }i \in [\![ 0, nb\_AQp - 1 ]\!],
		\mathrm{msb\_ind}[i] \neq j  
   \end{cases}$\;
    \If{$pivot \neq \varnothing$}{
        \For{$i \gets pivot + 1 \textbf{ à } nb\_AQp -1 $}{
            \If{$\mathrm{msb\_ind}[i] = j$}{
                $\mathrm{exp\_vects}[i]  \gets \mathrm{exp\_vects}[i] \oplus
				\mathrm{exp\_vects}[pivot]$ \; 
                $\mathrm{hist\_vects}[i] \gets \mathrm{hist\_vects}[i] \oplus
				\mathrm{hist\_vects}[pivot]$\; 
                $\mathrm{msb\_ind}[i] \gets  \textsc{MSB}(\mathrm{exp\_vects}
				[i])$\; 
                \If{$\mathrm{exp\_vects}[i] = 0 $}{
                    ajouter $i$ au tableau $\mathrm{lin\_rel\_ind}$\; 
                    $nb\_lin\_rel \gets nb\_lin\_rel + 1 $\; 
                }
            }
        }
    }
}
\Return{$\mathrm{hist\_vects}[0 \cdots nb\_AQp -1]$, $\mathrm{lin\_rel\_ind}
    [0 \cdots nb\_lin\_rel-1]$, $nb\_lin\_rel$}
\end{algorithm}
\vspace{1em}

\subsubsection{La fonction \texttt{calculate\_A\_Q}}

Une fois les indices des vecteurs historiques indiquant un ensemble valide de
paires $(A, Q)$ récupérés, la fonction \texttt{find\_factor} appelle la
fonction \texttt{calculate\_A\_Q} pour calculer des entiers $A$ et $Q$
vérifiant $A^2 \equiv Q^2 \pmod{N}$. Elle lui donne en argument un de ces
vecteurs historiques et les données des tableaux \texttt{Ans} et \texttt{Qns}.
\\

Notons $l$ l'entier \texttt{nb\_AQp} et $(e_{l-1}, \cdots , e_0) \in
\mathbb{F}_2^{l}$ le vecteur historique donné en argument de la fonction. Le
calcul de \[A:= \prod_{n=0}^{l-1} Ans[i]^{e_i} \pmod{N} \] ne pose pas de
difficulté. Pour le calcul de \[Q:= \sqrt{\prod_{n=0}^{l-1 } Qns[i] ^{e_i}}
\pmod{N},\] on utilise l'algorithme proposé par Morrison et Brillhart.

\vspace{1em}
\begin{algorithm}[H]
\DontPrintSemicolon
\caption{\sc Extraction de racine carrée}
\KwIn{Des entiers $Q_1,\cdots, Q_r \in \Z$ tels que $\prod_{i=1}^r Q_{i}$ est
	un carré}
\KwOut{$\sqrt{\prod_{i=1}^r Q_{i}} \pmod{N}$}
$Q \gets 1$\;
$R \gets Q_1$\;
    \For{$i \gets 2$ $\textbf{à } r$}{
    $X \gets \pgcd(R, Q_i)$\; 
    $Q \gets XQ \pmod{N}$\;
    $R \gets \frac{R}{X} \cdot \frac{Q_i}{X}$\;
}
$X \gets \sqrt{R}$\;
    $Q \gets XQ \pmod{N}$\;
\Return{Q}\;
\end{algorithm}
\vspace{1em}

Pour démontrer la correction de l'algorithme, on peut utiliser l'invariant de
boucle \[Q\sqrt{R.Q_i\cdots Q_r} \equiv \sqrt{\prod_{i=1}^r Q_{i}} \pmod{N},\]
dont la conservation découle de l'égalité \[Q\sqrt{R.Q_i\cdots
Q_r} \equiv QX \sqrt{\frac{R}{X}\frac{Q_i}{X}Q_{i+1} \cdots Q_r} \pmod{N}.\]

\subsection{Collecte des paires $(A,Q)$ : \texttt{step\_1.c et lp\_var.c}}

Décrivons à présent la phase de collecte des données. Concernant les vecteurs
historiques, il suffit d'initialiser à la fin de la collecte
\texttt{hist\_vects[i]} pour \texttt{0 <= i < nb\_AQp}. C'est ce que fait la
fonction \texttt{init\_hist\_vects}. La collecte des autres données requiert un
peu plus d'explications. 

\subsubsection{La fonction \texttt{create\_AQ\_pairs}}

Sachant que seules les paires $(A,Q)$ dont on a pu factoriser $Q_n$ nous
intéressent pour la seconde phase, nous avons décidé de ne stocker que
celles-ci. Ce choix a un autre avantage : ayant fixé \texttt{nb\_want\_AQp} le
nombre de paires $(A, Q)$ factorisées désirées, on peut arrêter le
développement en fraction continue dès que ce nombre est atteint. Au fur à
mesure du développement de $\sqrt{kN}$ en fraction continue, il faut donc
tester si le $Q_n$ qui vient d'être calculé est factorisable. Si c'est le cas,
on crée son vecteur exposant et ajoute les données de la paire aux tableaux
\texttt{Ans}, \texttt{Qns} et \texttt{exp\_vects}. Tout ceci est géré dans la
longue fonction \texttt{create\_AQ\_pairs}, qui utilise les sous-routines
\texttt{is\_Qn\_factorisable} et \texttt{init\_exp\_vect}.


\subsubsection{La \textit{early abort strategy}}

La fonction \texttt{is\_Qn\_factorisable} teste si un $Q_n$ est friable
\footnote{Ou presque friable, voir ¶ suivant.} par divisions successives avec
les premiers de la base de factorisation. Il est possible d'améliorer les
performances de la fonction en l'empêchant de poursuivre les divisions 
successives si, après un certain nombre de tentatives, on juge la partie non 
factorisée de $Q_n$ trop grande. On fixe pour cela une borne
\texttt{eas\_bound\_div} (proportionnelle à la borne déjà connue $2\sqrt{kN}$),
un palier \texttt{eas\_cut} et l'on arrête les divisions successives sur $Q_n$ 
si $Q_n$ reste plus grand que \texttt{eas\_bound\_div} après \texttt{eas\_cut}
tentatives de divisions.

\subsubsection{La \textit{large prime variation }}

Étant donnée une base de factorisation $B = \{ p_1, \cdots, p_m\}$, la
\textit{large prime variation } consiste à accepter lors de la collecte, non
seulement des $Q_n$ $B$-friables mais aussi des $Q_n$ produits d'un entier
$B$-friable et d'un entier $lp_n$ inférieur à $p_m^2$. On dira que $Q_n$ est
\emph{presque friable} et l'on appellera \emph{grand premier (large prime)} le
premier $lp_n$ en question. \\

Pour que des $Q_n$ presque friables soient exploitables, il faut qu'ils aient
un grand premier $lp$ en commun. En effet, si on trouve deux entiers presque
friables $Q_{n_1} = X_{n_1}lp $ et $Q_{n_2} =  X_{n_2}lp $, on peut former une
nouvelle paire $(A,Q)$ avec laquelle on peut travailler pour chercher une 
congruence de carrés. Remarquons pour cela que :
\[\begin{cases}
	A_{n_1 -1}^2 \equiv (-1)^{n_1} X_{n_1}lp\pmod{N}, \\
	A_{n_2 -1}^2 \equiv (-1)^{n_2} X_{n_2}lp\pmod{N}.
\end{cases}\]

Multiplier entre elles ces deux identités donne :
\[(A_{n_1 -1} A_{n_2 -1})^2 \equiv 
     \underbrace{(-1)^{n_1 + n_2} X_{n_1} X_{n_2}}_
            {\begin{subarray}{c}\text{associé au vecteur exposant}\\
             v_{B} \big( (-1 )^{n_1} X_{n_1} \big)
             + v_{B} \big( (-1 )^{n_2} X_{n_2} \big) \end{subarray}
             }
    \underbrace{lp^2}_{\text{carré qui ne pose pas problème}}
    \pmod{N}.
 \]
  
On forme donc la nouvelle paire $ (A_{n_1-1}A_{n_2 -1} \pmod{N},
Q_{n_1}Q_{n_2}) $ associée au vecteur exposant $v_{B} \big( (-1 )^{n_1} X_{n_1}
\big)+ v_{B} \big( (-1 )^{n_2} X_{n_2} \big) $.  Elle sera traitée lors de la
deuxième phase exactement de la même manière que les paires \og classiques
\fg{}.\\

En pratique, pour repérer les paires qui ont le même grand premier, nous
constituons au fur et à mesure de la collecte une liste chainée dont les nœuds
stockent les données d'une paire dont le $Q_n$ est presque friable (les entiers
$Q_n$, $A_{n-1}$, le vecteur exposant et le grand premier associé à $Q_n$).
Nous maintenons cette liste triée par taille des grands premiers. Lorsque que
survient un $Q_n$ presque friable, il est repéré par la fonction
\texttt{is\_Qn\_factorisable} qui fournit également son grand premier $lp$. La
liste chainée est alors parcourue pour savoir si l'on a déjà rencontré ce $lp$.
Deux cas se présentent alors. Si $lp$ est absent de la liste, on crée à la
bonne place un nœud. Si $lp$ est déjà présent dans la liste, au lieu de
rajouter un nœud, on utilise le nœud possédant ce $lp$ pour obtenir une
nouvelle paire $(A,Q)$ selon la méthode énoncée plus haut et ajoute ses
composantes aux tableaux \texttt{Ans}, \texttt{Qns} et \texttt{exp\_vects}. La
fonction \texttt{insert\_or\_elim\_lp} se charge de cela. \\



\newpage

\section{Résultats expérimentaux}

Les fonctions dont nous nous sommes servi pour tester le programme se trouvent
dans le fichier \texttt{test.c}. Nous avons effectué nos tests sur des entiers
générés par la fonction \texttt{rand\_N}. Ces entiers sont de la forme $N = pq$
avec $p$ et $q$ des premiers aléatoires et de taille semblable. 

\subsection{Choix de la taille de la base de factorisation}

Nous avons cherché à déterminer expérimentalement, selon la taille de l'entier
$N$ à factoriser, une bonne taille pour la base de factorisation lorsque la \textit{large
prime variation} et la \textit{early abort strategy} sont utilisées. Pour une
taille d'entier fixée, nous avons inscrit dans un fichier le temps que met le
programme pour plusieurs tailles de bases de factorisation (voir les fichiers
*.txt). Nous donnons ci-dessous des tailles appropriées (il ne s'agit que d'un
ordre de grandeur, la taille optimale varie d'un entier à l'autre). Ce sont
elles qui sont renvoyées par la fonction \texttt{choose\_s\_fb}.

\begin{center}
    \begin{tabular}{ |l |l | }
        \hline
         nombre de bits de $N$ & taille de la base de factorisation \\ \hline
         70                    & truc                               \\ \hline
         80                    & truc                               \\ \hline
         90                    & truc                               \\ \hline 
         100                   & truc                               \\ \hline 
         110                   & ttuc                               \\ \hline 
         120                   & truc                               \\ \hline 
         130                   & truc                               \\ \hline 
         140                   & truc                               \\ \hline 
         150                   & truc                               \\ \hline
    \end{tabular}
\end{center}

\subsection{Factorisation de $F_7$}

Notre programme permet de factoriser $F_7 = 2^{128} + 1$. La fonction 
\texttt{fact\_F7} prend en entrée des booléens indiquant si la
\textit{large\_prime variation (lp)} et la \textit{early abort strategy (eas)}
doivent être utilisées et lance cette factorisation. Avec le paramètre \texttt{k}
retourné par la fonction \texttt{choose\_k} - soit $k=38$ - et les paramètres 
\texttt{eas\_cut} et \texttt{eas\_coeff} valant respectivement $50$ et $1000000$,
nous obtenons les résultats suivants :  

\begin{center}
     \begin{tabular}{| l | l| l | l | l | l |}
     \hline
         variantes                    & \texttt{s\_fb} & \texttt{nb\_want\_AQp} &\texttt{nb\_AQp}& \texttt{last\_n}  & temps de calcul  \\ \hline
     \textit{lp} et \textit{eas}      &                &                        &                &                   &                  \\ \hline
     \textit{lp}                      &                &                        &                &                   &                  \\ \hline
     \textit{eas}                     &                &                        &                &                   &                  \\ \hline
     sans variante                    &                &                        &                &                   &                  \\ \hline
    \end{tabular}
\end{center}

\quad \footnotesize{\texttt{last\_n} est l'indice de la dernière paire $(A,Q)$ calculée.}

\subsection{Temps de calcul (avec les deux variantes) }

Ce graphique représente le temps moyen mis par le programme ( sur ... tests)
en fonction du nombre de bits de l'entier que l'on souhaite factoriser. La
\textit{large prime variation} et la \textit{early abort strategy} ont été 
utilisées et les paramètres ont été choisis par les fonctions 
\texttt{choose\_s\_fb} et \texttt{choose\_k}.\\


comparer quand k choisi par la fonction et k = 1: prévu de calculer le rapport des 
deux temps 


\addcontentsline{toc}{section}{Bibliographie}
\printbibliography

\end{document}
