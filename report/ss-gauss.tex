\subsection{Pivot de Gauss et recherche d'un facteur non trivial :
\texttt{step\_2.c}}

Avant de nous pencher sur les détails de la phase de collecte, regardons 
l'implémentation de la seconde phase, qui aide à mieux comprendre la 
forme sous laquelle nous collectons les données.

\subsubsection{Utilisation des données collectées}

A l'issue de la première phase, on espère avoir collecté \texttt{nb\_want\_AQp}
paires $(A,Q)$ avec $Q_n$ friable\footnote{Les paires peuvent aussi résulter de
la \textit{large prime variation}, cela n'a aucune incidence sur les fonctions
de cette partie.}. Le nombre réel de telles paires est stocké dans le champ
\texttt{nb\_AQp} d'une structure \texttt{Results} (voir sous-§ précédente). Une
paire $(A,Q)$ collectée est caractérisée par :

\begin{itemize}
    \item la valeur $A_{n-1}$,
    \item la valeur $Q_n$,
    \item le vecteur exposant associé à $Q_n$,
    \item un vecteur historique (voir ci-contre).
\end{itemize}

Les données de ces \texttt{nb\_AQp} paires sont stockées dans quatre tableaux :
\texttt{mpz\_t *Ans}, \texttt{mpz\_t *Qns}, \texttt{mpz\_t *exp\_vects} et
\texttt{mpz\_t *hist\_vects}. À un indice correspond une paire $(A,Q)$ donnée.
Le vecteur historique sert à indexer les paires collectées pour former un
analogue de la matrice identité utilisée pendant le pivot de Gauss. Plus
précisément, \texttt{hist\_vects[i]} est, avant pivot de Gauss, le vecteur
$(e_{l-1}, \cdots, e_0)$ où $l=\texttt{nb\_AQp}$ et $e_j = \delta_{ij}$. À
partir de ces quatre tableaux, la fonction \texttt{find\_factor} cherche un
facteur de \texttt{N} selon la méthode des congruences de carrés. Elle utilise
pour cela les fonctions auxiliaires \texttt{gauss\_elimination} et
\texttt{calculate\_A\_Q}. 

\subsubsection{La fonction \texttt{gauss\_elimination}}

La fonction \texttt{gauss\_elimination} effectue un pivot de Gauss sur les
éléments de \texttt{mpz\_t *exp\_vects}, vus comme les vecteurs-lignes d'une
matrice. Comme pour un pivot de Gauss classique, les calculs effectués sur les
vecteurs exposants sont reproduits en parallèle sur la matrice identité,
c'est-à-dire sur les éléments de \texttt{mpz\_t *hist\_vects}. Si le
\textit{xor} de deux vecteurs exposants donne le vecteur nul, cela signifie
qu'une relation de dépendance a été trouvée. On inscrit alors dans un tableau
l'indice de ce vecteur nul. Le vecteur historique dudit indice indique les
paires $(A,Q)$ de l'ensemble valide trouvé. La procédure que nous avons 
implémentée est décrite ci-dessous. 

\vspace{1em}
\begin{algorithm}[H]
\DontPrintSemicolon
\caption{\sc{Pivot de Gauss}}
\KwIn{tableau $\mathrm{exp\_vects}[0 \cdots nb\_AQp -1 ] $ des vecteurs
	exposants, tableau $\mathrm{hist\_vects}[0 \cdots nb\_AQp -1 ]$ des
	vecteurs historiques}
\vspace{0.5em}
\KwOut{$\mathrm{hist\_vects}[0 \cdots nb\_AQp -1]$ après le pivot, le nombre
	$nb\_lin\_rel$ de relations linéaires trouvées, $\mathrm{lin\_rel\_ind}[0
	\cdots nb\_lin\_rel-1]$ contenant les indices des lignes où une relation
	linéaire a été trouvée}
\vspace{0.5em}
créer tableau $\mathrm{msb\_ind}[0 \cdots nb\_AQp - 1]$\;  
créer tableau $\mathrm{lin\_rel\_ind}$\; 
$nb\_lin\_rel \gets 0$\; 
\vspace{0.5em}
\tcc{Initialisation du tableau \textsc{msb\_ind} : \textsc{Msb}(x) renvoie $0$
	si x est nul, l'indice du bit de poids fort de x sinon. Les indices des
	bits sont  numérotés de 1 à l'indice du bit de poids fort.}
\vspace{0.5em}
\For{$i \gets 0 \textbf{ à } nb\_AQp -1 $}{
    $\mathrm{msb\_ind}[i] \gets \textsc{MSB} (\mathrm{exp\_vects}[i])$  \; 
}
\vspace{0.5em}

\For{$ j \gets \textsc{MAX} (\mathrm{msb\_ind}) \textbf{ à } 1 $}{
    $pivot \gets \begin{cases}
		\min \big\{ i \in [\![ 0, nb\_AQp - 1 ]\!] \big\vert \mathrm{msb\_ind}[i]
		= j \big\}\\
		\varnothing \text{ si pour tout }i \in [\![ 0, nb\_AQp - 1 ]\!],
		\mathrm{msb\_ind}[i] \neq j  
   \end{cases}$\;
    \If{$pivot \neq \varnothing$}{
        \For{$i \gets pivot + 1 \textbf{ à } nb\_AQp -1 $}{
            \If{$\mathrm{msb\_ind}[i] = j$}{
                $\mathrm{exp\_vects}[i]  \gets \mathrm{exp\_vects}[i] \oplus
				\mathrm{exp\_vects}[pivot]$ \; 
                $\mathrm{hist\_vects}[i] \gets \mathrm{hist\_vects}[i] \oplus
				\mathrm{hist\_vects}[pivot]$\; 
                $\mathrm{msb\_ind}[i] \gets  \textsc{MSB}(\mathrm{exp\_vects}
				[i])$\; 
                \If{$\mathrm{exp\_vects}[i] = 0 $}{
                    ajouter $i$ au tableau $\mathrm{lin\_rel\_ind}$\; 
                    $nb\_lin\_rel \gets nb\_lin\_rel + 1 $\; 
                }
            }
        }
    }
}
\Return{$\mathrm{hist\_vects}[0 \cdots nb\_AQp -1]$, $\mathrm{lin\_rel\_ind}
    [0 \cdots nb\_lin\_rel-1]$, $nb\_lin\_rel$}
\end{algorithm}
\vspace{1em}

\subsubsection{La fonction \texttt{calculate\_A\_Q}}

Une fois les indices des vecteurs historiques indiquant un ensemble valide de
paires $(A, Q)$ récupérés, la fonction \texttt{find\_factor} appelle la
fonction \texttt{calculate\_A\_Q} pour calculer des entiers $A$ et $Q$
vérifiant $A^2 \equiv Q^2 \pmod{N}$. Elle lui donne en argument un de ces
vecteurs historiques et les données des tableaux \texttt{Ans} et \texttt{Qns}.
\\

Notons $l$ l'entier \texttt{nb\_AQp} et $(e_{l-1}, \cdots , e_0) \in
\mathbb{F}_2^{l}$ le vecteur historique donné en argument de la fonction. Le
calcul de \[A:= \prod_{n=0}^{l-1} Ans[i]^{e_i} \pmod{N} \] ne pose pas de
difficulté. Pour le calcul de \[Q:= \sqrt{\prod_{n=0}^{l-1 } Qns[i] ^{e_i}}
\pmod{N},\] on utilise l'algorithme proposé par Morrison et Brillhart.

\vspace{1em}
\begin{algorithm}[H]
\DontPrintSemicolon
\caption{\sc Extraction de racine carrée}
\KwIn{Des entiers $Q_1,\cdots, Q_r \in \Z$ tels que $\prod_{i=1}^r Q_{i}$ est
	un carré}
\KwOut{$\sqrt{\prod_{i=1}^r Q_{i}} \pmod{N}$}
$Q \gets 1$\;
$R \gets Q_1$\;
    \For{$i \gets 2$ $\textbf{à } r$}{
    $X \gets \pgcd(R, Q_i)$\; 
    $Q \gets XQ \pmod{N}$\;
    $R \gets \frac{R}{X} \cdot \frac{Q_i}{X}$\;
}
$X \gets \sqrt{R}$\;
    $Q \gets XQ \pmod{N}$\;
\Return{Q}\;
\end{algorithm}
\vspace{1em}

Pour démontrer la correction de l'algorithme, on peut utiliser l'invariant de
boucle \[Q\sqrt{R.Q_i\cdots Q_r} \equiv \sqrt{\prod_{i=1}^r Q_{i}} \pmod{N},\]
dont la conservation découle de l'égalité \[Q\sqrt{R.Q_i\cdots
Q_r} \equiv QX \sqrt{\frac{R}{X}\frac{Q_i}{X}Q_{i+1} \cdots Q_r} \pmod{N}.\]
