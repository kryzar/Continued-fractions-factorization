\subsection{La stucture générale du programme}

Notre programme comprend deux étapes principales. La première consiste à générer, 
à partir du développement en fractions continues de $ \sqrt{kN} $, des paires 
$(A, Q)$ avec $Q$ friable pour une base de factorisation préalablement déterminée.
On associe à chaque $Q$ ainsi produit un vecteur exposant \texttt{mpz\_t exp\_vect}.
Ce vecteur permet de retenir les nombres premiers qui interviennent dans la 
factorisation de $Q$ avec une valuation impaire. Dans le but d'augmenter le nombre
de paires $(A,Q) $ acceptées lors de cette étape, nous avons implémenté la "large 
prime variation". Celle-ci permet d'accepter une paire si $Q$ se factorise  grâce
aux premiers de la base de factorisation et à un grand facteur premier 
supplémentaire. Les fonctions de cette phase de collecte sont rassemblées dans les
fichiers \texttt{step\_1.h} et \texttt{step\_1.c}. Elles font appel, pour mettre
en oeuvre la "large prime variation", aux fonctions des fichiers \texttt{lp\_var.h}
et \texttt{lp\_var.c}. \\
 
Ces données sont traitées lors de la seconde phase dans l'espoir de trouver un 
facteur non trivial de $N$. Il s'agit de trouver des ensembles valides de paires
$(A, Q)$ par pivot de Gauss sur la matrice dont les lignes sont formées des 
vecteurs exposants. Chaque ensemble valide est à l'origine d'une congruence
$A^2 \equiv Q^2$ (mod $N$) permettant potentiellement de trouver un facteur non
trivial de $N$. Les fonctions de cette phase sont regroupées dans les fichiers
\texttt{step\_2.h} et \texttt{step\_2.c}. \\

Avant d'effectuer la première étape, il convient de se doter d'une base de 
factorisation. Ceci est permis par une fonction de fonction \NTS{trouver nom}
qui gèrent plus généralement le choix par défaut des paramètres. \\

explique \texttt{fact.c} \texttt{fact.h} \\ 
proposition : renommer stepA stepB stepC fact.h et changer les fonctions de 
fact.h et stepA.h


\newpage

\section{Entrées et sorties du programme}

Nous avons regroupé dans une structures \texttt{Params} les paramètres d'entrée 
de la fonction de factorisation, à savoir :

\begin{itemize}
    \item \texttt{N} : le nombre à factoriser, supposé produit de deux grands
                       nombres premiers.
    \item \texttt{k} : le coefficient multiplicateur.
    \item \texttt{n\_lim} : le nombre maximal de paires $(A,Q)$ que l'on 
                             s'autorise à calculer. Ce nombre prend en compte
                             toutes les paires produites et non uniquement les
                             paires avec $Q$ friable ou accepté par la "large
                             prime variation"
    \item \texttt{s\_fb} : la taille de la base de factorisation. 
    \item \texttt{nb\_want\_AQp} : le nombre désiré de paires $(A,Q)$ avec $Q$ 
                                 friable ou accepté par la "large prime variation"
    \item des booléens indiquant si la "early abort strategy" ou la "large prime 
          variation" doivent être utilisées et des paramètres s'y rapportant.

\end{itemize}

Le programme stocke dans une structure \texttt{Results} le facteur non trivial
de \texttt{N} trouvé (si tel est le cas) ainsi que des données permettant 
l'analyse des performances de la méthode.

\begin{remarque}
Nous avons décidé de ne pas implémenter un programme réalisant la factorisation
complète de $N$. En effet, l'efficatité de la méthode de factorisation dépend
du choix de certains paramètres : la taille de la base de factorisation et le
coefficient \texttt{k}. Nous avons préféré les considérer comme paramètres 
d'entrée du programme plutôt que comme paramètres devant être déterminés par 
une sous-routine en fonction du nombre à factoriser pou avoir plus de latitude 
dans les tests. 
\end{remarque}

\begin{remarque}
Notre programme n'est pas supposé prendre en entrée un nombre admettant un petit
facteur premier (inférieur aux premiers de la base de factorisation par exemple).
En effet, il ne test pas au préalable si $N$ est divisible par des petits facteurs, 
et mettra donc autant de temps à trouver un petit facteur qu'un grand facteur.
\end{remarque}
