% ss-fractions-continues.tex

\subsection{Fractions continues}

\begin{frame}{Intuition, définition}

Intuitivement, une fraction continue est une expression de la forme \[a_0 +
\cfrac{1}{a_1 + \cfrac{1}{a_2 + \cfrac{1}{a_3 + \dots}}}, \quad a_0\in \Z,
\forall i \in \N^* a_i \in \N^*.\]

\pause

\begin{definition}[Fraction continue]
	On appelle \emph{fraction continue} toute suite non vide (finie ou infinie)
	$(a_i)_{i\in U} \in \Z^U$, $U\subset \N$, d'entiers qui vérifie \[a_i
	\geq 1, \quad \forall i\in U\setminus\{0\}.\]
\end{definition}
\end{frame}

\begin{frame}{Génération (1/2)}

On génère un développement en fraction continue ainsi. Soit $x\in
\R\setminus{\Z}$.

\begin{itemize}
	\item On écrit $x = x + \ent{x} - \ent{x}$ et \[x = \ent{x} +
	\cfrac{1}{\cfrac{1}{x - \ent{x}}}.\] On pose $x_0 = x$ et $x_1 =
	\frac{1}{x_0 - \ent{x_0}}$.

	\pause

	\item On recommence sur $x_1$ : \[x = \ent{x_0} + \cfrac{1}{x_1} =
	\ent{x_0} + \cfrac{1}{\ent{x_1} + \cfrac{1}{\cfrac{1}{x_1 - \ent{x_1}}}}.\] 
\end{itemize}
\end{frame}

\begin{frame}{Génération (2/2)}
\begin{itemize}
	\item Si l'on peut continuer, on construit la suite d'éléments
	\emph{irrationnels} de terme général \[x_n = \frac{1}{x_{n-1} -
	\ent{x_{n-1}}}, \quad \forall n \geq 1.\]
\end{itemize}

\pause
\vspace{1em}
La suite est finie si $x$ est rationnel, infinie sinon. \vspace{1em}

On note $\hat{x}_n = \ent{x_n}$ pour tout $n$, de sorte que si $x$ est
irrationnel, on a \[x = \hat{x_0} + \cfrac{1}{\hat{x}_1 + \cfrac{1}{\hat{x}_2 +
\cfrac{1}{\hat{x}_3 + \dots}}}.\]
\end{frame}

\begin{frame}{Réduites (1/2)}

\begin{definition}[Réduites formelles]
	Soit $(X_i)_{i\in \N}$ une suite (infinie) d'indeterminées sur le corps
	$\Q$. On définit $[X_0] = X_0$ puis par récurrence \[[X_0, \dots,
	X_n] = X_0 + \frac{1}{[X_1, \dots, X_n]}.\]
\end{definition}

\pause

\begin{definition}
Soient $x\in \R$ et $f$ une fraction continue. On dit que \emph{$f$ est un
développement en fraction continue de $x$} si la suite des réduites de $f$
converge vers $x$.
\end{definition}

\end{frame}

\begin{frame}{Réduites (2/2)}

Si $x = (\hat{x_i})_{i\in \N}$, on a \[[\hat{x}_0, \dots, \hat{x}_n] =
\hat{x_0} + \cfrac{1}{\hat{x}_1 + \cfrac{1}{\dots + \cfrac{1}{\hat{x}_n}}}\]
et \[x = [\hat{x_0}, \dots, \hat{x_n}, x_{n+1}].\]

\end{frame}

\begin{frame}{Irrationels quadratiques}

\begin{definition}[Irrationel quadratique]
	On appelle \emph{irrationel quadratique} tout nombre réel, algébrique sur
	$\Q$, de degré $2$.
\end{definition}

Si $M\in \Z$ est sans facteurs carrés, $\sqrt{M}$ est un irrationel
quadratique. \vspace{1em}

\pause

Théorie des fractions continues très poussée pour les irrationels quadratiques
(Lagrange, Galois, Legendre). Il y a une bijection entre les irrationels et les
fractions continues infinies ; on peut donc parler \textit{du} développement en
fraction continue d'un irrationel.

\end{frame}

\begin{frame}{Identités}
Soit $x\in \R$ un irrationel.
\begin{itemize}
	\item $x_n$ est irrationel quadratique et s'écrit $x_n = \frac{P_n +
	x}{Q_n}$,
	\item la $n$-ième réduite du développement de $x$ est rationnelle et
	s'écrit $\frac{A_n}{B_n}, A_n, B_n \in \Z$.
\end{itemize}

\pause
\vspace{1em}

On a
\[\begin{cases}
	A_{n-1}^2 \equiv (-1)^n Q_n \pmod{N}, \\
	Q_n < 2\sqrt{kN}.
\end{cases}\]

\end{frame}
